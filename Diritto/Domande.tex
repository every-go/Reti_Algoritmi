\documentclass[10pt,oneside,a4paper]{article}
\usepackage{ragged2e,graphicx,tocloft,hyperref}
\title{Diritto informatica e società}
\author{Matteo Mazzaretto}
\date{2024/2025}
\hypersetup{
	colorlinks=true,   % Abilita il colore dei link (senza box intorno)
	linkcolor=blue,    % Colore per i link interni (come quelli nell'indice)
	urlcolor=red,      % Colore per i link URL esterni
	filecolor=magenta, % Colore per i link ai file locali
	citecolor=green,   % Colore per i riferimenti bibliografici
	pdfborder={0 0 0}  % Disabilita i bordi intorno ai link
}
\begin{document}
	% tolgo la numerazione in modo che la lunghezza dell'indice non incida sulla lunghezza del documento
	\pagenumbering{gobble}
	\maketitle
	\begin{center}
		%do un nuovo nome alla tabella degli indici e la inizializzo
		\renewcommand{\contentsname}{Indice}
		\tableofcontents
	\end{center}
	\newpage
	%inizio l'effettivo conteggio delle pagine
	\pagenumbering{arabic}
	\setcounter{page}{1}
	\section{Panoramica delle fonti del diritto, definizione, produzione e cognizione, gerarchia}
	Le fonti del diritto sono tutti gli atti e fatti che l'ordinamento giuridico riconosce come idonei a produrre norme giuridiche\\
	In particolare, si distinguono in fonti normative e fonti non normative\\
	\textbf{Fonti normative:} atti che producono direttamente norme giuridiche, tra cui la Costituzione, le leggi ordinarie, i regolamenti e le consuetudini\\
	\textbf{Fonti non normative:} atti che non producono direttamente norme giuridiche, ma sono utili per la conoscenza delle stesse, come la pubblicazione sulla Gazzetta Ufficiale, in questo caso si parla di fatti di cognizione\\
	Atti: Costituzione, legge, regolamento.\\
	Fatti: consuetudine (comportamento posto in essere dalla generalità di un'organizzazione per un periodo di tempo indefinito, in grado di acquisire valore normativo).\\
	La gerarchia delle fonti del diritto in Italia presenta cinque livelli principali:\\
	1) Costituzione\\
	2) Trattati UE e legislazione comunitaria\\
	3) Leggi statali, decreti legislativi, decreti-legge, leggi regionali\\
	4) Regolamenti\\
	5) Usi e consuetudini
	\section{Cosa succede in caso di contrasti fra diverse leggi}
	Nel caso di contrasti fra leggi rimane:
	\begin{enumerate}
		\item quella più in alto nella gerarchia (quando due norme contrastano, si applica il criterio gerarchico: la norma superiore prevale su quella inferiore. Tuttavia, quando il contrasto avviene tra fonti di pari grado, si applicano altri criteri come la cronologia o la competenz)
		\item fra stesse fonti, la più recente (stesso legislatore)
		\item fra leggi statali/regionali, ci si basa sul criterio di competenza
	\end{enumerate}
	Il principio generale è che una nuova legge disciplina solo i fatti successivi alla sua entrata in vigore (principio di irretroattività)\\
	Tuttavia, in diritto penale si applica sempre la norma più favorevole al reo (principio del favor rei)
	Se una norma viene dichiarata invalida, si considera nulla sin dall'origine (efficacia retroattiva), ma questo avviene solo in determinati casi, ad esempio quando viene dichiarata incostituzionale dalla Corte Costituzionale
	\section{Le leggi: chi le fa?}
	In Italia, il potere legislativo è esercitato dal Parlamento, composto dalla Camera dei deputati e dal Senato della Repubblica.  \\
	Le leggi possono essere proposte da:  
	\begin{itemize}
		\item Governo  
		\item Parlamentari  
		\item Consigli regionali
	\end{itemize}
	Il procedimento legislativo segue il principio del bicameralismo perfetto: una legge deve essere approvata nello stesso testo da entrambe le Camere\\  
	Ogni Camera approva il testo con maggioranza semplice (50\%+1 dei presenti, con almeno il 50\% dei componenti presenti)\\
	Se il Senato modifica il testo, la legge torna alla Camera per una nuova approvazione\\
	Le leggi di revisione costituzionale e altre leggi particolari richiedono una doppia approvazione da parte delle Camere, con almeno tre mesi di intervallo tra le due votazioni. 
	\begin{itemize}
		\item Se approvate con una maggioranza di almeno i \(\frac{2}{3}\) dei membri di ciascuna Camera, entrano in vigore direttamente.  
		\item Se approvate con una maggioranza compresa tra \(\frac{1}{2}\) e \(\frac{2}{3}\), possono essere sottoposte a referendum confermativo se richiesto da almeno 500.000 elettori, 5 Consigli regionali o 1/5 dei membri di una Camera.  
	\end{itemize}
	Oltre alle leggi ordinarie, il Governo può emanare atti con forza di legge in due casi particolari:  
	\begin{itemize}
		\item \textbf{Decreto legislativo (d.lgs.)}: il Parlamento delega il Governo a legiferare su una materia specifica tramite una \textit{legge delega}, che stabilisce principi e criteri direttivi. Il Governo, seguendo tali direttive, emana il decreto legislativo.  
		\item \textbf{Decreto-legge (d.l.)}: il Governo, in casi straordinari di necessità e urgenza, può emanare un decreto con forza di legge, che entra in vigore immediatamente. Tuttavia, deve essere convertito in legge dal Parlamento entro 60 giorni, altrimenti perde efficacia.  
	\end{itemize}
	\section{Come si interpretano le leggi? E le analogie?}
	Chiunque lavori con le norme giuridiche svolge necessariamente un ruolo interpretativo, ossia deve attribuire un significato preciso alle disposizioni legislative\\
	Poiché l'interpretazione delle leggi potrebbe essere influenzata dalla soggettività, l'ordinamento giuridico stabilisce criteri per garantire coerenza nell'interpretazione\\
	Questi criteri sono stabiliti nell'articolo 12 delle \textit{Disposizioni sulla legge in generale} (Preleggi)\\
	L'articolo 12 delle preleggi stabilisce che:  
	\begin{enumerate}
		\item \textbf{Interpretazione letterale e logica}: nell'applicare la legge, non si può attribuirle un significato diverso da quello reso evidente dal senso proprio delle parole, tenendo conto della loro connessione logica e dell'intenzione del legislatore
		\item \textbf{Interpretazione sistematica e teleologica}: se il significato letterale non è sufficiente, si considera l'intenzione del legislatore al momento della promulgazione della norma e la coerenza della norma con il sistema giuridico complessivo
	\end{enumerate}
	A volte, la legge può risultare imprecisa:  
	\begin{itemize}
		\item può esprimere più di quanto il legislatore intendeva dire
		\item può dire meno di quanto necessario per regolare un caso concreto  
	\end{itemize}
	I principali criteri interpretativi sono:  
	\begin{enumerate}
		\item \textbf{Interpretazione letterale}: analisi del significato proprio delle parole
		\item \textbf{Interpretazione logica}: considerazione del contesto logico della norma
		\item \textbf{Interpretazione sistematica}: valutazione della connessione della norma con altre disposizioni
		\item \textbf{Interpretazione teleologica}: ricerca dell'intenzione del legislatore al momento della promulgazione della legge
	\end{enumerate}Se manca una norma specifica per un caso concreto, si ricorre all'analogia, che può essere di due tipi:  
	\begin{enumerate}
		\item \textbf{Analogia legis}: si applica al caso non regolato una norma prevista per un caso simile
		\item \textbf{Analogia iuris}: se non esiste una norma specifica, si ricavano principi generali dall'intero ordinamento giuridico
	\end{enumerate}
	L'analogia non è ammessa in diritto penale e in altre materie di stretta interpretazione, poiché potrebbe violare il principio di legalità. 
	\section{Soggetti di diritto}
	I soggetti di diritto sono coloro che possono essere titolari di diritti e doveri giuridici\\
	Si distinguono in:  
	\begin{itemize}
		\item \textbf{Persone fisiche}: ogni essere umano è un soggetto di diritto sin dalla nascita
		\item \textbf{Enti collettivi}: comprendono sia le persone giuridiche che altri soggetti privi di personalità giuridica  
	\end{itemize}
	Le persone giuridiche sono enti ai quali l'ordinamento attribuisce una soggettività autonoma rispetto ai singoli individui che ne fanno parte\\
	Si distinguono in:  
	\begin{itemize}
		\item \textbf{Persone giuridiche} (soggetti con autonoma personalità giuridica):  
		\begin{itemize}
			\item Associazioni riconosciute  
			\item Fondazioni  
			\item Società di capitali (S.p.A., S.r.l.)  
		\end{itemize}
		\item \textbf{Gruppi organizzati senza personalità giuridica}, che comunque operano nel mondo giuridico:  
		\begin{itemize}
			\item Associazioni non riconosciute  
			\item Comitati  
			\item Società di persone (S.n.c., S.a.s.)  
		\end{itemize}
	\end{itemize}
	Il concetto di diritto può essere distinto in due accezioni principali:  
	\begin{enumerate}
		\item \textbf{Diritto soggettivo}: è la posizione giuridica attiva di un soggetto, che può vantare un interesse protetto dall'ordinamento. Esempi: diritto di proprietà, diritto di credito 
		\item \textbf{Diritto oggettivo}: è l'insieme delle norme giuridiche che regolano la società e disciplinano i diritti soggettivi
	\end{enumerate}
	\section{Capacità}
	Esistono due principali tipi di capacità: \textbf{capacità giuridica} e \textbf{capacità di agire}\\
	La capacità giuridica è l'attitudine a essere titolari di diritti e doveri\\
	Si acquisisce al momento della nascita e spetta a chiunque senza distinzioni\\
	In via eccezionale, l'ordinamento attribuisce alcuni diritti patrimoniali anche a soggetti non ancora nati (\textit{nascituri}) nei casi di:  
	\begin{itemize}
		\item Testamento  
		\item Donazione  
	\end{itemize}
	In questi casi, i diritti sono condizionati alla nascita del soggetto\\
	La capacità di agire è la capacità di esercitare personalmente i diritti di cui si è titolari, disponendone mediante atti giuridici come vendita, donazione o contratti\\
	Poiché gli atti possono avere conseguenze giuridiche rilevanti, la capacità di agire si acquista al compimento del \textbf{18° anno di età}\\
	Dai 18 anni si possono compiere validamente atti giuridici, ma in alcuni casi si possono compiere anche da minorenni (es. apertura di un conto corrente bancario può richiedere autorizzazione)\\
	Gli atti compiuti da un minore possono essere annullabili su richiesta del rappresentante legale\\
	La capacità di agire si distingue in:  
	\begin{enumerate}
		\item \textbf{Capacità naturale}: è la capacità di intendere e di volere, ossia di comprendere le proprie azioni e assumerne la responsabilità. È rilevante per determinare la validità di alcuni atti e la responsabilità civile o penale. 
		\item \textbf{Capacità legale}: è il riconoscimento formale della capacità di agire. I maggiorenni sono titolari di capacità legale, ma in alcuni casi possono essere dichiarati incapaci o sottoposti a limitazioni
	\end{enumerate}
	L'ordinamento prevede misure di protezione per i soggetti incapaci:  
	\begin{enumerate}
		\item \textbf{Minore di età}: non può compiere atti giuridici autonomamente, salvo eccezioni (es. acquisti di modesta entità). Gli atti sono compiuti dai genitori o da chi esercita la responsabilità genitoriale  
		\item \textbf{Inabilitazione}: misura che limita parzialmente la capacità di agire. Il soggetto inabilitato può compiere alcuni atti da solo, ma per quelli più rilevanti è affiancato da un \textbf{curatore}
		\item \textbf{Interdizione}: comporta la totale incapacità di agire. Gli atti giuridici vengono compiuti per suo conto da un \textbf{tutore} 
		\item \textbf{Amministrazione di sostegno}: misura più flessibile, introdotta per assistere persone con ridotta autonomia. L’amministratore di sostegno opera secondo le disposizioni stabilite dal giudice nel provvedimento di nomina
	\end{enumerate}
	La legge disciplina i poteri e i doveri di tutori, curatori e amministratori di sostegno, imponendo controlli e limiti per tutelare gli interessi dell’incapace.  
	\section{Diritto della personalità}
	Il diritto della personalità è disciplinato dall'articolo 2 della Costituzione italiana, il quale afferma:
	\begin{quote}
		"La Repubblica riconosce e garantisce i diritti inviolabili dell'uomo, sia come singolo, sia nelle formazioni sociali ove si svolge la sua personalità, e richiede l'adempimento dei doveri inderogabili di solidarietà politica, economica e sociale"
	\end{quote}
	I principali diritti della personalità sono:  
	\begin{enumerate}
		\item \textbf{Diritto al nome}  
		\item \textbf{Diritto all'immagine}  
		\item \textbf{Diritto morale d'autore}  
		\item \textbf{Diritto all'onore, reputazione e decoro}  
		\item \textbf{Diritto all'identità personale}  
		\item \textbf{Diritto all'integrità fisica}  
		\item \textbf{Diritto alla salute (e autodeterminazione terapeutica)}  
		\item \textbf{Diritto alla privacy}  
		\item \textbf{Diritto alla dignità}  
	\end{enumerate}
	Presentano le seguenti caratteristiche:  
	\begin{enumerate}
		\item \textbf{Innati e connaturati alla persona}: spettano all'individuo sin dalla nascita
		\item \textbf{Non patrimoniali}: non hanno contenuto economico diretto
		\item \textbf{Indisponibili}: non possono essere trasmessi, rinunciati o alienati
		\item \textbf{Imprescrittibili}: non si estinguono con il passare del tempo
	\end{enumerate}
	Le situazioni giuridiche soggettive possono acquisirsi in due modi:  
	\begin{itemize}
		\item \textbf{Acquisto originario}: il diritto nasce in capo al soggetto senza che vi sia un precedente titolare (es. diritto alla vita, alla salute)
		\item \textbf{Acquisto derivativo}: il diritto si trasferisce da un individuo all'altro attraverso atti giuridici come compravendita, successione ereditaria, donazione
	\end{itemize}
	\section{L'autonomia patrimoniale}
	L'autonomia patrimoniale è il principio secondo cui un ente collettivo dispone di un proprio patrimonio separato da quello delle persone fisiche che lo compongono\\
	Essa determina in che misura i beni dell'ente e quelli dei soci o membri restano distinti in caso di obbligazioni e responsabilità\\
	L'autonomia patrimoniale può essere:  
	\begin{enumerate}
		\item \textbf{Perfetta}: il patrimonio dell'ente è completamente separato da quello dei soci o membri. Ciò significa che i creditori dell'ente possono soddisfarsi solo sui beni dell'ente stesso, senza intaccare il patrimonio personale dei soci. Questo vale, ad esempio, per le società di capitali (S.p.A., S.r.l.)
		\item \textbf{Imperfetta}: i soci rispondono anche con il proprio patrimonio personale per le obbligazioni dell'ente. Questo accade nelle società di persone (S.n.c., S.a.s. per i soci accomandatari), dove almeno alcuni soci hanno responsabilità illimitata per i debiti sociali
	\end{enumerate}
	La distinzione tra autonomia patrimoniale perfetta e imperfetta è fondamentale per comprendere il regime di responsabilità in caso di insolvenza:  
	\begin{itemize}
		\item Se una persona fisica ha debiti personali, i creditori non possono aggredire il patrimonio dell'ente di cui è socio o membro.  
		\item Se invece è l'ente collettivo ad avere debiti, i creditori possono soddisfarsi sul suo patrimonio. Tuttavia, in presenza di autonomia patrimoniale imperfetta, possono rivalersi anche sui beni personali dei soci responsabili illimitatamente.  
	\end{itemize}
	\section{Le fonti europee quali sono?}
	La comunità internazionale è formata da Stati sovrani posti in posizione di reciproca parità\\
	L'ordinamento internazionale è costituito da consuetudini, convenzioni e organizzazioni degli Stati\\
	Una particolare comunità internazionale a cui appartiene l'Italia è il Consiglio d'Europa, che ha come obiettivo principale la tutela dei diritti umani\\
	L'organo giudiziario del Consiglio d'Europa è la Corte europea dei diritti dell’uomo (CEDU), che si occupa di garantire il rispetto della Convenzione europea dei diritti dell'uomo (CEDU) da parte degli Stati membri\\
	Le fonti normative dell'Unione Europea derivano principalmente da:  
	\begin{enumerate}
		\item \textbf{Trattati}:  
		\begin{itemize}
			\item \textbf{Trattati istitutivi}, che creano e regolano il funzionamento dell’Unione Europea (ad esempio, il Trattato di Roma, il Trattato di Maastricht, il Trattato di Lisbona)
			\item \textbf{Trattati di modifica}, che aggiornano o integrano i trattati istitutivi
		\end{itemize}
		\item \textbf{Principi generali del diritto dell'UE}, ricavati dalle tradizioni giuridiche comuni degli Stati membri e dal diritto dell’Unione
		\item \textbf{Carta dei diritti fondamentali dell'UE}, che raccoglie i diritti e le libertà fondamentali riconosciuti nell’ordinamento UE
	\end{enumerate}
	Gli atti adottati dalle istituzioni dell'Unione Europea si distinguono in:  
	\begin{enumerate}
		\item \textbf{Atti tipici}:
		\begin{enumerate}
			\item \textbf{Vincolanti}:
			\begin{itemize}
				\item \textbf{Regolamento}: ha effetto immediato negli Stati membri senza necessità di recepimento nazionale.
				\item \textbf{Direttiva}: vincola gli Stati membri al raggiungimento di un obiettivo, ma lascia loro la libertà di scegliere i mezzi più idonei per attuarla attraverso la legislazione nazionale.
				\item \textbf{Decisione}: obbligatoria per i destinatari (che possono essere Stati o soggetti specifici).
			\end{itemize}
			\item \textbf{Non vincolanti}:
			\begin{itemize}
				\item \textbf{Raccomandazione}: suggerisce un comportamento agli Stati membri senza obbligo giuridico.
				\item \textbf{Parere}: esprime una valutazione su una questione, senza effetti vincolanti.
			\end{itemize}
		\end{enumerate}
		\item \textbf{Atti atipici}:  
		\begin{itemize}
			\item Accordi interistituzionali
			\item Dichiarazioni comuni
			\item Comunicazioni
			\item Codici di condotta
			\item Libri verdi e libri bianchi
		\end{itemize}
	\end{enumerate}
	\section{Diritti dei beni e dell'ingegno}
	I beni possono essere:
	\begin{enumerate}
		\item mobili: tutto ciò che non rientra nelle successive definizioni
		\item immobili: beni con vincolo naturale o artificiale che li collega al suolo (fiumi, torrenti, edifici)
		\item beni mobili registrati: beni mobili soggetti a registrazione in particolare registri (motoveicoli)
	\end{enumerate}
	Costituiscono beni anche quelli privi di materialità come software, energia elettrica, brevetto, registrazione marchio\\
	Si può distinguere ulteriormente i beni fra:
	\begin{enumerate}
		\item beni fungibili: sostituibili (denaro per ecccellenza)
		\item beni infungibili: non sostituibili
	\end{enumerate}
	I beni \textit{fruttiferi} sono beni che producono altri beni, i quali vengono chiamati \textit{frutti}, che possono essere naturali o civili\\
	I diritti dell'ingegno sono quelli descritti dall'art 2575 cc:\\
	\begin{quote}
	Formano oggetto del diritto di autore le opere dell'ingegno di carattere creativo che appartengono alle scienze, alla letteratura, alla musica, alle 
	arti figurative, all'architettura, al teatro e alla cinematografia, qualunque ne sia il modo o la forma di espressione\\
	\end{quote}
	Ad essere protetti non sono le idee base delle opere bensì la forma di espressione di tale idee, questo perché il modo in cui l'artista esprime è ciò che viene proiettato\\
	Si può suddividire in:
	\begin{enumerate}
		\item opere d'ingegno: fa parte il software che è quindi soggetto a diritto d'autore e copyright, possono anche essere brevettati come invenzioni industriali 
		\item invenzioni industriali
		\item modelli industriali
	\end{enumerate}
	\section{Copyright/Brevetto/Diritto d'autore}
	Il diritto d'autore deriva dall'art. 2575 del Codice Civile:
	\begin{quote}
		Formano oggetto del diritto di autore le opere dell'ingegno di carattere creativo che appartengono alle scienze, alla letteratura, alla musica, alle arti figurative, all'architettura, al teatro e alla cinematografia, qualunque ne sia il modo o la forma di espressione.
	\end{quote}
	Il diritto d'autore si compone di due elementi principali:
	\begin{enumerate}
		\item \textbf{Diritto morale}:  
		\begin{itemize}
			\item È il diritto della personalità dell'autore, riconosciuto come irrinunciabile
			\item Permette di difendere l'integrità e la paternità dell'opera, anche dopo la morte dell'autore
		\end{itemize}
		\item \textbf{Diritto patrimoniale}:  
		\begin{itemize}
			\item Consente lo sfruttamento economico dell'opera, tutelando lo sforzo creativo dell'autore
			\item Comprende la facoltà di autorizzare o vietare la riproduzione, distribuzione, noleggio e comunicazione al pubblico dell'opera
			\item Ha una durata di protezione che si estende fino a 70 anni dopo la morte dell'autore
			\item Può essere ceduto per contratto o per successione (mortis causa)
			\item Il principio di \textit{esaurimento} limita il diritto patrimoniale: una volta venduto un prodotto, il titolare non può controllarne il successivo commercio, fatta eccezione per la regolamentazione dei download digitali
		\end{itemize}
	\end{enumerate}
	Il brevetto, disciplinato dall'art. 2585 del Codice Civile, protegge le nuove invenzioni destinate ad avere un'applicazione industriale. In particolare:
	\begin{quote}
		Possono costituire oggetto di brevetto le nuove invenzioni atte ad avere un'applicazione industriale, quali un metodo o un processo di lavorazione industriale, una macchina, uno strumento, un utensile o un dispositivo meccanico, un prodotto o un risultato industriale e l'applicazione tecnica di un principio scientifico, purché essa dia immediati risultati industriali.
	\end{quote}
	Le caratteristiche principali del brevetto sono:
	\begin{itemize}
		\item \textbf{Sfruttamento economico esclusivo}: Il titolare del brevetto può escludere terzi dallo sfruttamento dell'invenzione
		\item \textbf{Durata}:
		\begin{itemize}
			\item Invenzioni industriali: durata massima di 20 anni, non rinnovabile
			\item Modelli d'utilità: durata di 10 anni
			\item Disegni (2D) e modelli (3D): durata iniziale di 5 anni, rinnovabile fino a un massimo di 5 volte
		\end{itemize}
		\item \textbf{Requisiti di brevettabilità}:  
		Per essere brevettabile, un'invenzione deve essere:
		\begin{enumerate}
			\item Innovativa
			\item Dotata di applicazione industriale
			\item Non divulgata al pubblico prima della presentazione della domanda di brevetto (novità)
		\end{enumerate}
	\end{itemize}
	Il termine \textit{copyright} è spesso utilizzato in senso intercambiabile con il diritto d'autore, sebbene in alcuni contesti vi siano delle distinzioni:
	\begin{itemize}
		\item \textbf{Ambito di applicazione}:  
		Il copyright, in particolare, si riferisce all'insieme dei diritti patrimoniali che tutelano l'opera e ne regolano l'utilizzo economico, come la riproduzione, la distribuzione e la comunicazione al pubblico
		\item \textbf{Tutela internazionale}:  
		Il concetto di copyright è ampiamente riconosciuto a livello internazionale e assume sfumature specifiche a seconda delle normative dei vari paesi
		\item \textbf{Differenza concettuale}:  
		Mentre il diritto d'autore comprende sia la componente morale (irrinunciabile e legata alla personalità dell'autore) sia quella patrimoniale, il copyright è spesso inteso come l'insieme delle tutele economiche derivanti dal diritto patrimoniale
	\end{itemize}
	\section{Diritto di proprietà}
	La proprietà è disciplinata dall'articolo 832 del Codice Civile:
	\begin{quote}
		Il proprietario ha diritto di godere e disporre delle cose in modo pieno ed esclusivo, entro i limiti e con l'osservanza degli obblighi stabiliti dall'ordinamento giuridico.
	\end{quote}
	Nel diritto privato sussiste il divieto di compiere atti emulativi, ovvero il proprietario non può compiere azioni con il solo scopo di arrecare molestie ad altri\\
	Inoltre, il diritto di proprietà può essere limitato per motivi di pubblica utilità, ad esempio mediante l'espropriazione
	Esistono due modi di acquisizione del diritto di proprietà:
	\begin{enumerate}
		\item \textbf{Titolo derivativo}: il contratto trasla il diritto di proprietà dal vecchio al nuovo proprietario (ad esempio, in caso di successione ereditaria)
		\item \textbf{Titolo originario}: il diritto di proprietà sorge direttamente, ad esempio, attraverso l'invenzione, la creazione o altri istituti giuridici
	\end{enumerate}
	Sebbene la proprietà non si acquisisca per prescrizione ordinaria, essa può essere oggetto di \textit{usucapione}: se un soggetto esercita il possesso continuativo e in buona fede per il periodo previsto dalla legge, e se il proprietario non rivendica il proprio diritto, quest'ultimo può venir meno\\
	Perché l'usucapione sia possibile è necessario il concetto di \textit{possesso}, così definito dall'articolo 1140 del Codice Civile:
	\begin{quote}
		Il possesso è il potere sulla cosa che si manifesta in un'attività corrispondente all'esercizio della proprietà o di altro diritto reale. Si può possedere direttamente o per mezzo di altra persona, che ha la detenzione della cosa.
	\end{quote}
	\section{Cosa dice la regola del possesso vale titolo?}
	La regola del possesso vale titolo è descritta dall'art 1153 cc:
	\begin{quote}
		Colui al quale sono alienati beni mobili da parte di chi non è proprietario, ne acquista la proprietà mediante il possesso, purché sia in buona fede al momento della consegna e sussista un titolo idoneo al trasferimento della proprietà
	\end{quote}
	Praticamente, se al momento dell'acquisto l'articolo era stato rubato/venduto in maniera illecita ma l'acquirente non ne aveva idea l'acquisto rimane valido perché in buona fede e non voleva compiere l'illecito\\
	Infatti, la regola mira a tutelare il possessore in buona fede
	\section{GDPR e Diritti dell'interessato (Cos'è anche il principio di minimizzazione)}
	g
	\section{Diritto di informazione del GDPR}
	g
	\section{Software e Protezione del Software}
	g
	\section{Privacy by design/by default}
	g
	\section{Contratti digitali}
	g
	\section{Contratti (definizione) e tipi di contratto}
	g
	\section{Differenza tra diritto di protezione alla privacy e del trattamento del dato personale}
	g
	\section{Contratti stipulati dai minorenni}
	g
	\section{GDPR e soggetti coinvolti}
	g
	\section{Vari tipi di contratti}
	g
	\section{Successione del patrimonio digitale e il consenso come base giuridica (gdpr)}
	g
	\section{Open Software}
	g
	\section{Tipi di dato}
	g
	\section{Gamification}
	g
	\section{Illecito civile in particolare illecito extracontrattuale}
	g
	\section{Quali elementi ci devono essere per poter parlare di un illecito extracontrattuale}
	g
	\section{Condizioni generali di contratto}
	g
	\section{Contratto collettivo}
	g
	\section{La definizione del titolare (data controller)}
	g
	\section{Elementi del contratto}
	g
	\section{Quando un contratto è invalido e quando è inefficace}
	g
	\section{Come si interpreta il contratto?}
	g
	\section{Cos'è la causa?}
	g
	\section{Principio trasparenza GDPR}
	g
	\section{Cos'è la subordinazione? Contratto di lavoro subordinato}
	g\\
	-----------------------------------------------------------
	\section{Poteri direttivi e di controllo}
	g
	\section{Quali sono i compiti del DPO e la responsabilità civile in generale e del service provider}
	g
	\section{Privacy sul posto di lavoro}
	g
	\section{Provider}
	g
	\section{Domande su RRI (concezione procedurale/normativa)}
	g
	\section{Esempio di concorrenza di pasticcerie}
	g
	\section{Il lavoro su piattaforma è regolato dal nostro ordinamento. La normativa li considera lavori autonomi o subordinati?}
	g
	\section{Cosa sono i sistemi decisionali di monitoraggio automatizzati?}
	g
	\section{Potere di controllo}
	g
	\section{Cos'è una base giuridia e quali sono le possibili basi giuridiche per trattamento dei dati}
	g
	\section{Controlli occulti}
	g
	\section{Il potere disciplinare del datore}
	g
	\section{Contratto d'appalto e contratto d'opera}
\end{document}