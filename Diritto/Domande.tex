\documentclass[8pt,oneside,a4paper]{article}
\usepackage{ragged2e,graphicx,tocloft,hyperref}
\title{Diritto informatica e società}
\author{Matteo Mazzaretto}
\date{2024/2025}
\hypersetup{
	colorlinks=true,   % Abilita il colore dei link (senza box intorno)
	linkcolor=blue,    % Colore per i link interni (come quelli nell'indice)
	urlcolor=red,      % Colore per i link URL esterni
	filecolor=magenta, % Colore per i link ai file locali
	citecolor=green,   % Colore per i riferimenti bibliografici
	pdfborder={0 0 0}  % Disabilita i bordi intorno ai link
}
\begin{document}
	% tolgo la numerazione in modo che la lunghezza dell'indice non incida sulla lunghezza del documento
	\pagenumbering{gobble}
	\maketitle
	\begin{center}
		%do un nuovo nome alla tabella degli indici e la inizializzo
		\renewcommand{\contentsname}{Indice}
		\tableofcontents
	\end{center}
	\newpage
	%inizio l'effettivo conteggio delle pagine
	\pagenumbering{arabic}
	\setcounter{page}{1}
	\section{Panoramica delle fonti del diritto, definizione, produzione e cognizione, gerarchia}
	Le fonti del diritto sono tutti gli atti e fatti che l'ordinamento giuridico riconosce come idonei a produrre norme giuridiche\\
	In particolare, si distinguono in fonti normative e fonti non normative\\
	\textbf{Fonti normative:} atti che producono direttamente norme giuridiche, tra cui la Costituzione, le leggi ordinarie, i regolamenti e le consuetudini\\
	\textbf{Fonti non normative:} atti che non producono direttamente norme giuridiche, ma sono utili per la conoscenza delle stesse, come la pubblicazione sulla Gazzetta Ufficiale, in questo caso si parla di fatti di cognizione\\
	Atti: Costituzione, legge, regolamento.\\
	Fatti: consuetudine (comportamento posto in essere dalla generalità di un'organizzazione per un periodo di tempo indefinito, in grado di acquisire valore normativo).\\
	La gerarchia delle fonti del diritto in Italia presenta cinque livelli principali:\\
	1) Costituzione\\
	2) Trattati UE e legislazione comunitaria\\
	3) Leggi statali, decreti legislativi, decreti-legge, leggi regionali\\
	4) Regolamenti\\
	5) Usi e consuetudini
	\section{Cosa succede in caso di contrasti fra diverse leggi}
	Nel caso di contrasti fra leggi rimane:
	\begin{enumerate}
		\item quella più in alto nella gerarchia (quando due norme contrastano, si applica il criterio gerarchico: la norma superiore prevale su quella inferiore. Tuttavia, quando il contrasto avviene tra fonti di pari grado, si applicano altri criteri come la cronologia o la competenz)
		\item fra stesse fonti, la più recente (stesso legislatore)
		\item fra leggi statali/regionali, ci si basa sul criterio di competenza
	\end{enumerate}
	Il principio generale è che una nuova legge disciplina solo i fatti successivi alla sua entrata in vigore (principio di irretroattività)\\
	Tuttavia, in diritto penale si applica sempre la norma più favorevole al reo (principio del favor rei)
	Se una norma viene dichiarata invalida, si considera nulla sin dall'origine (efficacia retroattiva), ma questo avviene solo in determinati casi, ad esempio quando viene dichiarata incostituzionale dalla Corte Costituzionale
	\section{Le leggi: chi le fa?}
	In Italia, il potere legislativo è esercitato dal Parlamento, composto dalla Camera dei deputati e dal Senato della Repubblica.  \\
	Le leggi possono essere proposte da:  
	\begin{itemize}
		\item Governo  
		\item Parlamentari  
		\item Consigli regionali
	\end{itemize}
	Il procedimento legislativo segue il principio del bicameralismo perfetto: una legge deve essere approvata nello stesso testo da entrambe le Camere\\  
	Ogni Camera approva il testo con maggioranza semplice (50\%+1 dei presenti, con almeno il 50\% dei componenti presenti)\\
	Se il Senato modifica il testo, la legge torna alla Camera per una nuova approvazione\\
	Le leggi di revisione costituzionale e altre leggi particolari richiedono una doppia approvazione da parte delle Camere, con almeno tre mesi di intervallo tra le due votazioni. 
	\begin{itemize}
		\item Se approvate con una maggioranza di almeno i \(\frac{2}{3}\) dei membri di ciascuna Camera, entrano in vigore direttamente.  
		\item Se approvate con una maggioranza compresa tra \(\frac{1}{2}\) e \(\frac{2}{3}\), possono essere sottoposte a referendum confermativo se richiesto da almeno 500.000 elettori, 5 Consigli regionali o 1/5 dei membri di una Camera.  
	\end{itemize}
	Oltre alle leggi ordinarie, il Governo può emanare atti con forza di legge in due casi particolari:  
	\begin{itemize}
		\item \textbf{Decreto legislativo (d.lgs.)}: il Parlamento delega il Governo a legiferare su una materia specifica tramite una \textit{legge delega}, che stabilisce principi e criteri direttivi. Il Governo, seguendo tali direttive, emana il decreto legislativo.  
		\item \textbf{Decreto-legge (d.l.)}: il Governo, in casi straordinari di necessità e urgenza, può emanare un decreto con forza di legge, che entra in vigore immediatamente. Tuttavia, deve essere convertito in legge dal Parlamento entro 60 giorni, altrimenti perde efficacia.  
	\end{itemize}
	\section{Come si interpretano le leggi? E le analogie?}
	Chiunque lavori con le norme giuridiche svolge necessariamente un ruolo interpretativo, ossia deve attribuire un significato preciso alle disposizioni legislative\\
	Poiché l'interpretazione delle leggi potrebbe essere influenzata dalla soggettività, l'ordinamento giuridico stabilisce criteri per garantire coerenza nell'interpretazione\\
	Questi criteri sono stabiliti nell'articolo 12 delle \textit{Disposizioni sulla legge in generale} (Preleggi)\\
	L'articolo 12 delle preleggi stabilisce che:  
	\begin{enumerate}
		\item \textbf{Interpretazione letterale e logica}: nell'applicare la legge, non si può attribuirle un significato diverso da quello reso evidente dal senso proprio delle parole, tenendo conto della loro connessione logica e dell'intenzione del legislatore
		\item \textbf{Interpretazione sistematica e teleologica}: se il significato letterale non è sufficiente, si considera l'intenzione del legislatore al momento della promulgazione della norma e la coerenza della norma con il sistema giuridico complessivo
	\end{enumerate}
	A volte, la legge può risultare imprecisa:  
	\begin{itemize}
		\item può esprimere più di quanto il legislatore intendeva dire
		\item può dire meno di quanto necessario per regolare un caso concreto  
	\end{itemize}
	I principali criteri interpretativi sono:  
	\begin{enumerate}
		\item \textbf{Interpretazione letterale}: analisi del significato proprio delle parole
		\item \textbf{Interpretazione logica}: considerazione del contesto logico della norma
		\item \textbf{Interpretazione sistematica}: valutazione della connessione della norma con altre disposizioni
		\item \textbf{Interpretazione teleologica}: ricerca dell'intenzione del legislatore al momento della promulgazione della legge
	\end{enumerate}Se manca una norma specifica per un caso concreto, si ricorre all'analogia, che può essere di due tipi:  
	\begin{enumerate}
		\item \textbf{Analogia legis}: si applica al caso non regolato una norma prevista per un caso simile
		\item \textbf{Analogia iuris}: se non esiste una norma specifica, si ricavano principi generali dall'intero ordinamento giuridico
	\end{enumerate}
	L'analogia non è ammessa in diritto penale e in altre materie di stretta interpretazione, poiché potrebbe violare il principio di legalità. 
	\section{Soggetti di diritto}
	I soggetti di diritto sono coloro che possono essere titolari di diritti e doveri giuridici\\
	Si distinguono in:  
	\begin{itemize}
		\item \textbf{Persone fisiche}: ogni essere umano è un soggetto di diritto sin dalla nascita
		\item \textbf{Enti collettivi}: comprendono sia le persone giuridiche che altri soggetti privi di personalità giuridica  
	\end{itemize}
	Le persone giuridiche sono enti ai quali l'ordinamento attribuisce una soggettività autonoma rispetto ai singoli individui che ne fanno parte\\
	Si distinguono in:  
	\begin{itemize}
		\item \textbf{Persone giuridiche} (soggetti con autonoma personalità giuridica):  
		\begin{itemize}
			\item Associazioni riconosciute  
			\item Fondazioni  
			\item Società di capitali (S.p.A., S.r.l.)  
		\end{itemize}
		\item \textbf{Gruppi organizzati senza personalità giuridica}, che comunque operano nel mondo giuridico:  
		\begin{itemize}
			\item Associazioni non riconosciute  
			\item Comitati  
			\item Società di persone (S.n.c., S.a.s.)  
		\end{itemize}
	\end{itemize}
	Il concetto di diritto può essere distinto in due accezioni principali:  
	\begin{enumerate}
		\item \textbf{Diritto soggettivo}: è la posizione giuridica attiva di un soggetto, che può vantare un interesse protetto dall'ordinamento. Esempi: diritto di proprietà, diritto di credito 
		\item \textbf{Diritto oggettivo}: è l'insieme delle norme giuridiche che regolano la società e disciplinano i diritti soggettivi
	\end{enumerate}
	\section{Capacità}
	Esistono due principali tipi di capacità: \textbf{capacità giuridica} e \textbf{capacità di agire}\\
	La capacità giuridica è l'attitudine a essere titolari di diritti e doveri\\
	Si acquisisce al momento della nascita e spetta a chiunque senza distinzioni\\
	In via eccezionale, l'ordinamento attribuisce alcuni diritti patrimoniali anche a soggetti non ancora nati (\textit{nascituri}) nei casi di:  
	\begin{itemize}
		\item Testamento  
		\item Donazione  
	\end{itemize}
	In questi casi, i diritti sono condizionati alla nascita del soggetto\\
	La capacità di agire è la capacità di esercitare personalmente i diritti di cui si è titolari, disponendone mediante atti giuridici come vendita, donazione o contratti\\
	Poiché gli atti possono avere conseguenze giuridiche rilevanti, la capacità di agire si acquista al compimento del \textbf{18° anno di età}\\
	Dai 18 anni si possono compiere validamente atti giuridici, ma in alcuni casi si possono compiere anche da minorenni (es. apertura di un conto corrente bancario può richiedere autorizzazione)\\
	Gli atti compiuti da un minore possono essere annullabili su richiesta del rappresentante legale\\
	La capacità di agire si distingue in:  
	\begin{enumerate}
		\item \textbf{Capacità naturale}: è la capacità di intendere e di volere, ossia di comprendere le proprie azioni e assumerne la responsabilità. È rilevante per determinare la validità di alcuni atti e la responsabilità civile o penale. 
		\item \textbf{Capacità legale}: è il riconoscimento formale della capacità di agire. I maggiorenni sono titolari di capacità legale, ma in alcuni casi possono essere dichiarati incapaci o sottoposti a limitazioni
	\end{enumerate}
	L'ordinamento prevede misure di protezione per i soggetti incapaci:  
	\begin{enumerate}
		\item \textbf{Minore di età}: non può compiere atti giuridici autonomamente, salvo eccezioni (es. acquisti di modesta entità). Gli atti sono compiuti dai genitori o da chi esercita la responsabilità genitoriale  
		\item \textbf{Inabilitazione}: misura che limita parzialmente la capacità di agire. Il soggetto inabilitato può compiere alcuni atti da solo, ma per quelli più rilevanti è affiancato da un \textbf{curatore}
		\item \textbf{Interdizione}: comporta la totale incapacità di agire. Gli atti giuridici vengono compiuti per suo conto da un \textbf{tutore} 
		\item \textbf{Amministrazione di sostegno}: misura più flessibile, introdotta per assistere persone con ridotta autonomia. L’amministratore di sostegno opera secondo le disposizioni stabilite dal giudice nel provvedimento di nomina
	\end{enumerate}
	La legge disciplina i poteri e i doveri di tutori, curatori e amministratori di sostegno, imponendo controlli e limiti per tutelare gli interessi dell’incapace.  
	\section{Diritto della personalità}
	Il diritto della personalità è disciplinato dall'articolo 2 della Costituzione italiana, il quale afferma:
	\begin{quote}
		"La Repubblica riconosce e garantisce i diritti inviolabili dell'uomo, sia come singolo, sia nelle formazioni sociali ove si svolge la sua personalità, e richiede l'adempimento dei doveri inderogabili di solidarietà politica, economica e sociale"
	\end{quote}
	I principali diritti della personalità sono:  
	\begin{enumerate}
		\item \textbf{Diritto al nome}  
		\item \textbf{Diritto all'immagine}  
		\item \textbf{Diritto morale d'autore}  
		\item \textbf{Diritto all'onore, reputazione e decoro}  
		\item \textbf{Diritto all'identità personale}  
		\item \textbf{Diritto all'integrità fisica}  
		\item \textbf{Diritto alla salute (e autodeterminazione terapeutica)}  
		\item \textbf{Diritto alla privacy}  
		\item \textbf{Diritto alla dignità}  
	\end{enumerate}
	Presentano le seguenti caratteristiche:  
	\begin{enumerate}
		\item \textbf{Innati e connaturati alla persona}: spettano all'individuo sin dalla nascita
		\item \textbf{Non patrimoniali}: non hanno contenuto economico diretto
		\item \textbf{Indisponibili}: non possono essere trasmessi, rinunciati o alienati
		\item \textbf{Imprescrittibili}: non si estinguono con il passare del tempo
	\end{enumerate}
	Le situazioni giuridiche soggettive possono acquisirsi in due modi:  
	\begin{itemize}
		\item \textbf{Acquisto originario}: il diritto nasce in capo al soggetto senza che vi sia un precedente titolare (es. diritto alla vita, alla salute)
		\item \textbf{Acquisto derivativo}: il diritto si trasferisce da un individuo all'altro attraverso atti giuridici come compravendita, successione ereditaria, donazione
	\end{itemize}
	\section{L'autonomia patrimoniale}
	L'autonomia patrimoniale è il principio secondo cui un ente collettivo dispone di un proprio patrimonio separato da quello delle persone fisiche che lo compongono\\
	Essa determina in che misura i beni dell'ente e quelli dei soci o membri restano distinti in caso di obbligazioni e responsabilità\\
	L'autonomia patrimoniale può essere:  
	\begin{enumerate}
		\item \textbf{Perfetta}: il patrimonio dell'ente è completamente separato da quello dei soci o membri. Ciò significa che i creditori dell'ente possono soddisfarsi solo sui beni dell'ente stesso, senza intaccare il patrimonio personale dei soci. Questo vale, ad esempio, per le società di capitali (S.p.A., S.r.l.)
		\item \textbf{Imperfetta}: i soci rispondono anche con il proprio patrimonio personale per le obbligazioni dell'ente. Questo accade nelle società di persone (S.n.c., S.a.s. per i soci accomandatari), dove almeno alcuni soci hanno responsabilità illimitata per i debiti sociali
	\end{enumerate}
	La distinzione tra autonomia patrimoniale perfetta e imperfetta è fondamentale per comprendere il regime di responsabilità in caso di insolvenza:  
	\begin{itemize}
		\item Se una persona fisica ha debiti personali, i creditori non possono aggredire il patrimonio dell'ente di cui è socio o membro.  
		\item Se invece è l'ente collettivo ad avere debiti, i creditori possono soddisfarsi sul suo patrimonio. Tuttavia, in presenza di autonomia patrimoniale imperfetta, possono rivalersi anche sui beni personali dei soci responsabili illimitatamente.  
	\end{itemize}
	\section{Le fonti europee quali sono?}
	La comunità internazionale è formata da Stati sovrani posti in posizione di reciproca parità\\
	L'ordinamento internazionale è costituito da consuetudini, convenzioni e organizzazioni degli Stati\\
	Una particolare comunità internazionale a cui appartiene l'Italia è il Consiglio d'Europa, che ha come obiettivo principale la tutela dei diritti umani\\
	L'organo giudiziario del Consiglio d'Europa è la Corte europea dei diritti dell’uomo (CEDU), che si occupa di garantire il rispetto della Convenzione europea dei diritti dell'uomo (CEDU) da parte degli Stati membri\\
	Le fonti normative dell'Unione Europea derivano principalmente da:  
	\begin{enumerate}
		\item \textbf{Trattati}:  
		\begin{itemize}
			\item \textbf{Trattati istitutivi}, che creano e regolano il funzionamento dell’Unione Europea (ad esempio, il Trattato di Roma, il Trattato di Maastricht, il Trattato di Lisbona)
			\item \textbf{Trattati di modifica}, che aggiornano o integrano i trattati istitutivi
		\end{itemize}
		\item \textbf{Principi generali del diritto dell'UE}, ricavati dalle tradizioni giuridiche comuni degli Stati membri e dal diritto dell’Unione
		\item \textbf{Carta dei diritti fondamentali dell'UE}, che raccoglie i diritti e le libertà fondamentali riconosciuti nell’ordinamento UE
	\end{enumerate}
	Gli atti adottati dalle istituzioni dell'Unione Europea si distinguono in:  
	\begin{enumerate}
		\item \textbf{Atti tipici}:
		\begin{enumerate}
			\item \textbf{Vincolanti}:
			\begin{itemize}
				\item \textbf{Regolamento}: ha effetto immediato negli Stati membri senza necessità di recepimento nazionale.
				\item \textbf{Direttiva}: vincola gli Stati membri al raggiungimento di un obiettivo, ma lascia loro la libertà di scegliere i mezzi più idonei per attuarla attraverso la legislazione nazionale.
				\item \textbf{Decisione}: obbligatoria per i destinatari (che possono essere Stati o soggetti specifici).
			\end{itemize}
			\item \textbf{Non vincolanti}:
			\begin{itemize}
				\item \textbf{Raccomandazione}: suggerisce un comportamento agli Stati membri senza obbligo giuridico.
				\item \textbf{Parere}: esprime una valutazione su una questione, senza effetti vincolanti.
			\end{itemize}
		\end{enumerate}
		\item \textbf{Atti atipici}:  
		\begin{itemize}
			\item Accordi interistituzionali
			\item Dichiarazioni comuni
			\item Comunicazioni
			\item Codici di condotta
			\item Libri verdi e libri bianchi
		\end{itemize}
	\end{enumerate}
	\section{Diritti dei beni e dell'ingegno}
	I beni possono essere:
	\begin{enumerate}
		\item mobili: tutto ciò che non rientra nelle successive definizioni
		\item immobili: beni con vincolo naturale o artificiale che li collega al suolo (fiumi, torrenti, edifici)
		\item beni mobili registrati: beni mobili soggetti a registrazione in particolare registri (motoveicoli)
	\end{enumerate}
	Costituiscono beni anche quelli privi di materialità come software, energia elettrica, brevetto, registrazione marchio\\
	Si può distinguere ulteriormente i beni fra:
	\begin{enumerate}
		\item beni fungibili: sostituibili (denaro per ecccellenza)
		\item beni infungibili: non sostituibili
	\end{enumerate}
	I beni \textit{fruttiferi} sono beni che producono altri beni, i quali vengono chiamati \textit{frutti}, che possono essere naturali o civili\\
	I diritti dell'ingegno sono quelli descritti dall'art 2575 cc:\\
	\begin{quote}
	Formano oggetto del diritto di autore le opere dell'ingegno di carattere creativo che appartengono alle scienze, alla letteratura, alla musica, alle 
	arti figurative, all'architettura, al teatro e alla cinematografia, qualunque ne sia il modo o la forma di espressione\\
	\end{quote}
	Ad essere protetti non sono le idee base delle opere bensì la forma di espressione di tale idee, questo perché il modo in cui l'artista esprime è ciò che viene proiettato\\
	Si può suddividire in:
	\begin{enumerate}
		\item opere d'ingegno: fa parte il software che è quindi soggetto a diritto d'autore e copyright, possono anche essere brevettati come invenzioni industriali 
		\item invenzioni industriali
		\item modelli industriali
	\end{enumerate}
	\section{Copyright/Brevetto/Diritto d'autore}
	Il diritto d'autore deriva dall'art. 2575 del Codice Civile:
	\begin{quote}
		Formano oggetto del diritto di autore le opere dell'ingegno di carattere creativo che appartengono alle scienze, alla letteratura, alla musica, alle arti figurative, all'architettura, al teatro e alla cinematografia, qualunque ne sia il modo o la forma di espressione.
	\end{quote}
	Il diritto d'autore si compone di due elementi principali:
	\begin{enumerate}
		\item \textbf{Diritto morale}:  
		\begin{itemize}
			\item È il diritto della personalità dell'autore, riconosciuto come irrinunciabile
			\item Permette di difendere l'integrità e la paternità dell'opera, anche dopo la morte dell'autore
		\end{itemize}
		\item \textbf{Diritto patrimoniale}:  
		\begin{itemize}
			\item Consente lo sfruttamento economico dell'opera, tutelando lo sforzo creativo dell'autore
			\item Comprende la facoltà di autorizzare o vietare la riproduzione, distribuzione, noleggio e comunicazione al pubblico dell'opera
			\item Ha una durata di protezione che si estende fino a 70 anni dopo la morte dell'autore
			\item Può essere ceduto per contratto o per successione (mortis causa)
			\item Il principio di \textit{esaurimento} limita il diritto patrimoniale: una volta venduto un prodotto, il titolare non può controllarne il successivo commercio, fatta eccezione per la regolamentazione dei download digitali
		\end{itemize}
	\end{enumerate}
	Il brevetto, disciplinato dall'art. 2585 del Codice Civile, protegge le nuove invenzioni destinate ad avere un'applicazione industriale. In particolare:
	\begin{quote}
		Possono costituire oggetto di brevetto le nuove invenzioni atte ad avere un'applicazione industriale, quali un metodo o un processo di lavorazione industriale, una macchina, uno strumento, un utensile o un dispositivo meccanico, un prodotto o un risultato industriale e l'applicazione tecnica di un principio scientifico, purché essa dia immediati risultati industriali.
	\end{quote}
	Le caratteristiche principali del brevetto sono:
	\begin{itemize}
		\item \textbf{Sfruttamento economico esclusivo}: Il titolare del brevetto può escludere terzi dallo sfruttamento dell'invenzione
		\item \textbf{Durata}:
		\begin{itemize}
			\item Invenzioni industriali: durata massima di 20 anni, non rinnovabile
			\item Modelli d'utilità: durata di 10 anni
			\item Disegni (2D) e modelli (3D): durata iniziale di 5 anni, rinnovabile fino a un massimo di 5 volte
		\end{itemize}
		\item \textbf{Requisiti di brevettabilità}:  
		Per essere brevettabile, un'invenzione deve essere:
		\begin{enumerate}
			\item Innovativa
			\item Dotata di applicazione industriale
			\item Non divulgata al pubblico prima della presentazione della domanda di brevetto (novità)
		\end{enumerate}
	\end{itemize}
	Il termine \textit{copyright} è spesso utilizzato in senso intercambiabile con il diritto d'autore, sebbene in alcuni contesti vi siano delle distinzioni:
	\begin{itemize}
		\item \textbf{Ambito di applicazione}:  
		Il copyright, in particolare, si riferisce all'insieme dei diritti patrimoniali che tutelano l'opera e ne regolano l'utilizzo economico, come la riproduzione, la distribuzione e la comunicazione al pubblico
		\item \textbf{Tutela internazionale}:  
		Il concetto di copyright è ampiamente riconosciuto a livello internazionale e assume sfumature specifiche a seconda delle normative dei vari paesi
		\item \textbf{Differenza concettuale}:  
		Mentre il diritto d'autore comprende sia la componente morale (irrinunciabile e legata alla personalità dell'autore) sia quella patrimoniale, il copyright è spesso inteso come l'insieme delle tutele economiche derivanti dal diritto patrimoniale
	\end{itemize}
	\section{Diritto di proprietà}
	La proprietà è disciplinata dall'articolo 832 del Codice Civile:
	\begin{quote}
		Il proprietario ha diritto di godere e disporre delle cose in modo pieno ed esclusivo, entro i limiti e con l'osservanza degli obblighi stabiliti dall'ordinamento giuridico.
	\end{quote}
	Nel diritto privato sussiste il divieto di compiere atti emulativi, ovvero il proprietario non può compiere azioni con il solo scopo di arrecare molestie ad altri\\
	Inoltre, il diritto di proprietà può essere limitato per motivi di pubblica utilità, ad esempio mediante l'espropriazione
	Esistono due modi di acquisizione del diritto di proprietà:
	\begin{enumerate}
		\item \textbf{Titolo derivativo}: il contratto trasla il diritto di proprietà dal vecchio al nuovo proprietario (ad esempio, in caso di successione ereditaria)
		\item \textbf{Titolo originario}: il diritto di proprietà sorge direttamente, ad esempio, attraverso l'invenzione, la creazione o altri istituti giuridici
	\end{enumerate}
	Sebbene la proprietà non si acquisisca per prescrizione ordinaria, essa può essere oggetto di \textit{usucapione}: se un soggetto esercita il possesso continuativo e in buona fede per il periodo previsto dalla legge, e se il proprietario non rivendica il proprio diritto, quest'ultimo può venir meno\\
	Perché l'usucapione sia possibile è necessario il concetto di \textit{possesso}, così definito dall'articolo 1140 del Codice Civile:
	\begin{quote}
		Il possesso è il potere sulla cosa che si manifesta in un'attività corrispondente all'esercizio della proprietà o di altro diritto reale. Si può possedere direttamente o per mezzo di altra persona, che ha la detenzione della cosa.
	\end{quote}
	\section{Cosa dice la regola del possesso vale titolo?}
	La regola del possesso vale titolo è descritta dall'art 1153 cc:
	\begin{quote}
		Colui al quale sono alienati beni mobili da parte di chi non è proprietario, ne acquista la proprietà mediante il possesso, purché sia in buona fede al momento della consegna e sussista un titolo idoneo al trasferimento della proprietà
	\end{quote}
	Praticamente, se al momento dell'acquisto l'articolo era stato rubato/venduto in maniera illecita ma l'acquirente non ne aveva idea l'acquisto rimane valido perché in buona fede e non voleva compiere l'illecito\\
	Infatti, la regola mira a tutelare il possessore in buona fede
	\section{Software e Protezione del Software}
	Un software è un'opera dell'ingegno e in quanto tale è soggetto a diritto d'autore e copyright\\
	Il diritto d'autore ha 2 componenti:
	\begin{enumerate}
		\item normale: diritto della personalità, è irrinunciabile e si può far valere anche dopo la morte dell'autore, diritto a potersi difendere da danni
		\item patrimoniale: sfruttamento economico dell'opera, cedibile, prestabile sino a 70 anni dopo la morte dell'autore. Comunque se si compra una distribuzione si può fare quello che si vuole (escluso il download)
	\end{enumerate}
	Tutti i software possono essere anche brevettati (diritto a sfruttare economicamente la propria invenzione), portando alle Invenzioni Industriali
	\section{Contratti digitali}
	I \textbf{contratti digitali} sono accordi stipulati tra due o più parti mediante strumenti informatici e reti telematiche\\
	Ai sensi dell'art. 1321 del codice civile, un contratto è l'accordo tra due o più parti per costituire, regolare o estinguere un rapporto giuridico patrimoniale, e tale definizione si applica anche ai contratti conclusi in formato digitale\\
	La normativa italiana riconosce la validità dei contratti digitali, purché siano rispettati i requisiti di forma, consenso e capacità giuridica delle parti\\
	La modalità di sottoscrizione elettronica dei documenti è distinta tra firma elettronica semplice, avanzata e qualificata\\
	\begin{enumerate}
		\item Semplice: messaggio email con foto di una firma autografa
		\item Elettronica avanzata: richiesta in banca per autorizzare operazioni telematiche, richiedono particolari tablet con pennino che registrano movimento ed intensità (firma grafometrica), immutabile ed è possibile ricondurre la firma alla persona. Il limite è che viene usata solo nel contesto
		\item Elettronica qualificata: al momento contiene solo la firma digitale, ovvero una firma generabile esclusivamente con prodotti tecnologici che lo permettono forniti da gestori detti classificatori, come Aruba, garantendo immutabilità e provenienza
	\end{enumerate}
	La firma elettronica qualificata ha valore legale equivalente alla firma autografa, rendendo pienamente valido e opponibile il contratto digitale\\
	Tuttavia, anche le altre forme di firma elettronica possono essere utilizzate, purché siano idonee a garantire l’identificabilità del firmatario e l’integrità del documento\\	
	\section{Contratti (definizione) e tipi di contratto}
	Un contratto è l'accordo tra due o più parti per costituire, regolaree o estinguere tra loro un rapporto giuridico patrimoniale\\
	I contratti si possono classificare in diverse categorie, tra cui:
	\begin{itemize}
		\item \textbf{Contratti unilaterali e bilaterali}: a seconda che l'obbligazione gravi su una sola parte o su entrambe
		\item \textbf{Contratti onerosi e gratuiti}: in base alla presenza o meno di un corrispettivo
		\item \textbf{Contratti consensuali e reali}: a seconda che si perfezionino con il solo consenso o con la consegna della cosa
		\item \textbf{Contratti tipici e atipici}: i primi sono previsti dalla legge, i secondi sono frutto dell'autonomia contrattuale (come il contratto di franchising)
	\end{itemize}
	\section{Contratti stipulati dai minorenni}
	I \textbf{minorenni}, ai sensi dell'art. 2 del codice civile, non hanno la piena capacità di agire e, pertanto, non possono validamente stipulare contratti, se non nei casi previsti dalla legge\\
	In linea generale, i contratti conclusi da un minore sono \textit{annullabili}, salvo che siano stipulati dal rappresentante legale (solitamente un genitore o un tutore)\\
	Tuttavia, esistono eccezioni in cui il minore può compiere atti giuridici validi:
	\begin{itemize}
		\item atti di ordinaria amministrazione, se ritenuti proporzionati alla sua età e maturità;
		\item contratti favorevoli (ad esempio, una donazione), che non comportano obblighi a suo carico;
		\item esercizio di attività lavorativa, se autorizzata e conforme alle leggi sul lavoro minorile.
	\end{itemize}
	Per gli atti eccedenti l’ordinaria amministrazione, è generalmente necessaria l’autorizzazione del giudice tutelare
	\section{Illecito civile in particolare illecito extracontrattuale, facendo riferimento agli elementi necessari}
	L'illecito civile è disciplinato dall'articolo 1218 c.c.:
	\begin{quote}
		Il debitore che non esegue esattamente la prestazione dovuta è tenuto al risarcimento del danno...
	\end{quote}
	Riguarda quindi l'effettivo inadempimento del contratto\\
	Invece, l'illecito extracontrattuale è disciplinato dall'art 2043 cc:
	\begin{quote}
		Qualunque fatto doloso o colposo, che cagiona ad altri un danno ingiusto, obbliga colui che ha commesso il fatto a risarcire il danno
	\end{quote}
	Disciplina l'illecito extracontrattuale, e per essere applicabile devono esserci 4 elementi suddivisi fra soggettivi e oggettivi:
	\begin{itemize}
		\item Elementi oggettivi:
		\begin{enumerate}
			\item danno ingiusto: va a ledere interesse giuridico tutelato, come concorrenza sleale, violazione diritto soggettivo, lesione diritto personalità, diritto di credito. Vengono esclusi legittima difesa e stato di casualità/necessità
			\item nesso causale: legame di causa-effetto tra il comportamento (o l’omissione) dell’agente e il danno subito dalla vittima
		\end{enumerate}
		\item Elementi soggettivi:
		\begin{enumerate}
			\item imputabilità: presente nell'art 2046, non risponde delle conseguenze del fatto dannoso che in quel momento non aveva capacità di intendere e volere non per sua colpa. Se il danno avviene dall'incapace naturale risponde il sorvegliante, se esso non è presente il giudice, prese in considerazione le condizioni economiche delle parti, può condannare l'autore del danno ad un'equa indennità
			\item colpavolezza: il danno è risarcibile solo se con dolo o con colpa
		\end{enumerate}
	\end{itemize}
	\section{Condizioni generali di contratto}
	Art 1341 c.c.:
	\begin{quote}
		Le condizioni generali di contratto predisposte da uno dei contraenti sono efficaci nei confronti dell'altro, se al momento della conclusione del contratto questi le ha conosciute o avrebbe dovuto conoscerle per ordinaria diligenza...
	\end{quote}
	Non valgono clausole, se non per iscritto, a favore di chi le ha predisposte, limitazioni di responsabilità, facoltà di recedere dal contratto o di sospenderne l'esecuzione, ovvero sanciscono a carico dell'altro contraente decadenze, limitazioni alla facoltà di opporre eccezioni, restrizioni alla libertà contrattuale nei rapporti coi terzi, tacita proroga o rinnovazione del contratto, clausole compromissorie o deroghe alla competenza dell'autorità giudiziaria
	\section{Contratto collettivo}
	Il \textbf{contratto collettivo} è un accordo giuridico stipulato tra una o più organizzazioni sindacali dei lavoratori e una o più associazioni datoriali, al fine di regolare le condizioni di lavoro di una categoria di lavoratori o di un determinato settore economico\\
	Il contratto collettivo stabilisce le norme relative a vari aspetti del rapporto di lavoro, tra cui la retribuzione, l’orario di lavoro, le ferie, la sicurezza, le modalità di assunzione e licenziamento, e altri diritti e doveri reciproci tra le parti\\
	Esistono due tipi principali di contratto collettivo:
	\begin{itemize}
		\item \textbf{Contratto collettivo nazionale di lavoro (CCNL)}: regolamenta i rapporti di lavoro a livello nazionale e si applica a tutti i lavoratori di un determinato settore o categoria, indipendentemente dal datore di lavoro
		\item \textbf{Contratto collettivo aziendale o territoriale}: si applica a una singola azienda o a un territorio specifico, e può integrare o modificare le disposizioni del CCNL in relazione alle esigenze particolari dell’impresa o della zona
	\end{itemize}
	Il contratto collettivo ha forza di legge per le parti che lo sottoscrivono e, quando esteso, anche per i lavoratori e i datori di lavoro non aderenti alle organizzazioni firmatarie. L'eventuale violazione del contratto collettivo può comportare sanzioni legali
	\section{Elementi del contratto}
	Gli elementi del contratto sono:
	\begin{enumerate}
		\item Accordo: almeno 2 parti si devono mettere d'accordo al fine di regolare, concordare e istituire il contratto, tramite due parti principali, ovvero la "proposta ed accetazione" e "autonomia contrattuale" (libero arbitrio nella forma creando potenzialmente ulteriori tipi di contratto purché siano meritevoli dell'interesse giuridico)
		\item Causa: funzione del contratto (NON MOTIVO), nel caso della compravendita è lo scambio. Deve essere lecito (se illecito porta all'annullamento del contratto). Un esempio è la funzione di riposo nel contratto per alloggiare un posto in un hotel allo scopo di fare una vacanza
		\item Oggetto: prestazioni in cui le parti si vincolano e beni oggetto di quelle prestazioni. Deve essere lecito, possibile (se impossibile porta a contratto nullo) e determinato o determinabile
		\item Forma: modo di dichiarare una volontà. Di regola è libera ma è richiesta una determinata forma (scritta/digitale) per la prova del contratto
	\end{enumerate}
	\section{Quando un contratto è invalido e quando è inefficace}
	L'invalidità indica l'esistenza di u difetto originario dell'atto, ovvero mancanza o vizio di un suo elemento essenziale che impedisce di produrre effetti giuridici\\
	La nullità riguarda l'oggetto impossibile/non determinabile, non produce effetti (invalido e inefficace)\\
	L'annullabilità è il contratto stipulato dall'incapace legale (no capacità di agire), incapace naturale, violenza morale o dolo. Produce effetti fintanto che non è annullato (invalido ma efficace)
	\section{Come si interpreta il contratto?}
	L'interpretazione del contratto è il processo attraverso il quale si stabilisce il senso e l'ambito delle dichiarazioni e delle obbligazioni contenute nel contratto stesso\\
	Secondo l'art. 1362 del \textbf{codice civile italiano}, il contratto deve essere interpretato secondo la comune intenzione delle parti, e non solo secondo il significato letterale delle parole\\
	In particolare, si seguono alcuni principi fondamentali per determinare l'effettivo significato dell'accordo:
	\begin{itemize}
		\item \textbf{Intenzione delle parti}: il contratto deve essere interpretato alla luce dell'intenzione comune delle parti, anche se tale intenzione non risulta espressamente indicata nel testo. L'intenzione si deduce dalle circostanze del contratto e dal comportamento delle parti
		\item \textbf{Significato letterale}: il significato delle parole deve essere valutato in base al loro uso corrente, salvo che nel contratto non emergano elementi che indichino un significato diverso
		\item \textbf{Interpretazione sistematica}: il contratto deve essere interpretato nel contesto dell’intero accordo, evitando che una singola clausola venga letta isolatamente
		\item \textbf{Principio di buona fede}: il contratto deve essere interpretato in buona fede, nel rispetto delle regole di correttezza e leale cooperazione tra le parti
	\end{itemize}
	Nel caso in cui, nonostante l’applicazione di questi principi, rimangano dubbi interpretativi, si ricorre al \textbf{giudizio del tribunale} per risolvere la questione, tenendo conto anche della prassi consolidata o delle consuetudini del settore
	\section{GDPR e Diritti dell'interessato (Cos'è anche il principio di minimizzazione)}
	Il GDPR (regolamento generale per la protezione dei dati) disciplina riservatezza (libertà sull'utilizzo dei dati del cittadino) e protezione (non utilizzare dati altrui per propri scopi)\\
	I principi dell'interessato sono:
	\begin{enumerate}
		\item liceità: impresa che tratta i dati deve avere la valida base giuridica (consenso soggetto, protezione interessi vitali, adempimento compito pubblico interesse, legittimo interesse aziendale) per il trattamento
		\item correttezza: i dati devono essere trattati in modo corretto ed equo in modo da regolarizzare gli algoritmi
		\item trasparenza: l'individuo deve rendersi conto del come e del perché del trattamento dei dati
		\item minimizzazione: devono essere trattati meno dati possibili
		\item memoria limitata: i dati storici possono essere conservati ma con "soggetto anonimizzato"
		\item integrità e confidenzialità: devono essere introdotte misure per impedire l'accesso ai dati ad estranei
		\item protezione dei dati \textit{By Design} (tutto a nroma sia per lavoratori che per clienti) e \textit{By Default} (il minimo ammontare di dati necessario è collezionato e usato di default)
	\end{enumerate}
	\section{Differenza tra diritto di protezione alla privacy e del trattamento del dato personale}
	La protezione alla privacy si riferisce alla protezione dei dati e si intende il non utilizzare dati altrui per propri scopi\\
	Il trattamento dei dati personali invece riguarda quasiasi cosa relativa ai dati personali
	\section{GDPR e soggetti coinvolti}
	I  principali soggetti della Privacy sono: l’interessato, il titolare del trattamento dei dati, il Responsabile del trattamento, il DPO, il terzo e l’Autorità di Controllo
	\begin{enumerate}
		\item Interessato: la persona fisica cui si riferiscono i dati personali
		\item Titolare del Trattamento: la persona fisica, la società, l’associazione o un’altra entità che controlla il trattamento dei dati personali ed è autorizzata a prendere decisioni essenziali sulle finalità e modalità di tale trattamento, comprese le misure di sicurezza applicabili
		\item Responsabile del Trattamento: la persona fisica, la società, l’associazione o l’organizzazione a cui il Titolare ha affidato l’attività specifica di gestione e controllo dei dati personali, in base all’esperienza e/o alle competenze pertinenti in materia
		\item DPO (Data Protection Officer): il professionista con conoscenze specialistiche sulla legislazione e sulle pratiche in materia di protezione dei dati.
		\item Terzo: la persona fisica o giuridica, l’autorità pubblica, il servizio o qualsiasi altro organismo che non sia l’interessato, il titolare del trattamento, il responsabile del trattamento. E’ una persona autorizzata al trattamento dei dati sotto l’autorità diretta del titolare o del responsabile
		\item Autorità di Controllo: l’autorità pubblica indipendente istituita da uno Stato membro, nel nostro caso l’Autorità e il Garante per la protezione dei dati personali
	\end{enumerate}
	\section{Base Giuridica per il Trattamento dei Dati}
	\subsection{Definizione}
	La \textbf{base giuridica} (o \textit{legal basis}) è il fondamento legittimo che autorizza il trattamento dei dati personali, ai sensi dell’\textbf{Art. 6 GDPR}\\
	Ogni trattamento deve avere una base giuridica valida, esplicitata \textit{prima} della raccolta dei dati (Art. 13-14 GDPR)
	\noindent \textbf{Requisiti chiave}:
	\begin{itemize}
		\item \textbf{Necessaria}: Senza base giuridica, il trattamento è illecito (Art. 5 GDPR).
		\item \textbf{Documentata}: Deve essere indicata nell’informativa privacy.
		\item \textbf{Non retroattiva}: Non può essere applicata a trattamenti già avvenuti.
	\end{itemize}
	\subsection{Le 6 Basi Giuridiche del GDPR}
	\begin{enumerate}
		\item \textbf{Consenso dell’interessato} (Art. 6(1)(a)) \\
		\textit{Esempio}: Iscrizione a una newsletter tramite spunta esplicita. \\
		\textbf{Limiti}: Deve essere \textit{libero, specifico, informato} e \textit{revocabile} (Art. 7). Non valido per rapporti di forza squilibrati (es. datore di lavoro-dipendente).\\
		\item \textbf{Esecuzione di un contratto} (Art. 6(1)(b)) \\
		\textit{Esempio}: Trattamento dati per consegnare un ordine su un e-commerce. \\
		\textbf{Nota}: Vale solo per dati \textit{strettamente necessari} all’adempimento contrattuale.\\
		\item \textbf{Obbligo legale} (Art. 6(1)(c)) \\
		\textit{Esempio}: Invio dati fiscali all’Agenzia delle Entrate. \\
		\textbf{Requisito}: Deve derivare da una norma dell’UE o degli Stati membri (es. antiriciclaggio).\\
		\item \textbf{Interesse vitale dell’interessato} (Art. 6(1)(d)) \\
		\textit{Esempio}: Condivisione di dati medici in un’emergenza sanitaria. \\
		\textbf{Applicabilità}: Solo in casi eccezionali (es. pericolo di vita).\\
		\item \textbf{Interesse legittimo del titolare} (Art. 6(1)(f)) \\
		\textit{Esempio}: Prevenzione delle frodi in un servizio online. \\
		\textbf{Bilanciamento}: Richiede una \textit{valutazione proporzionalità} tra interesse del titolare e diritti dell’interessato (considerando 49 GDPR).\\ \textit{Non applicabile} a dati particolari (Art. 9).\\
		\item \textbf{Missione di interesse pubblico} (Art. 6(1)(e)) \\
		\textit{Esempio}: Trattamento dati da parte di scuole o enti pubblici. \\
		\textbf{Fonte}: Deve essere previsto da una legge nazionale/UE (es. sanità pubblica).\\
	\end{enumerate}
	\subsection{Successione del patrimonio digitale e consenso nel GDPR}
	La successione del patrimonio digitale riguarda la trasmissione dei dati personali di un defunto agli eredi\\
	Ai sensi del GDPR:
	\begin{itemize}
		\item Il \textbf{consenso} (Art. 6(1)(a)) non è trasferibile, poiché strettamente personale.
		\item Gli eredi possono esercitare diritti solo se:
		\begin{itemize}
			\item Il trattamento si basa su altre basi giuridiche (es. contratto, Art. 6(1)(b)).
			\item La legge nazionale lo preveda (es. Italia con Art. 2-terdecies D.lgs. 101/2018).
		\end{itemize}
	\end{itemize}
	\section{Privacy sul posto di lavoro e controlli occulti}  
	\subsection{Definizione}  
	I \textbf{controlli occulti} (o \textit{monitoraggi nascosti}) sono attività di sorveglianza o analisi dei dati personali condotte:  
	\begin{itemize}  
		\item \textbf{Senza trasparenza}: L’interessato non è informato del trattamento (violando l’Art. 5(1)(a) GDPR, principio di liceità e correttezza).  
		\item \textbf{Senza base giuridica valida}: Spesso privi di consenso, necessità contrattuale o interesse legittimo proporzionato.  
	\end{itemize}  
	\subsection{Esempi}  
	\begin{itemize}  
		\item \textbf{Workplace monitoring}:  
		\begin{itemize}  
			\item Tracciamento tastiera/schermo dei dipendenti senza avviso.  
			\item Analisi delle email aziendali con AI per valutare performance.  
		\end{itemize}  
		\item \textbf{Profilazione online}:  
		\begin{itemize}  
			\item Raccolta dati di navigazione tramite cookie nascosti.  
			\item Uso di fingerprinting del browser senza consenso.  
		\end{itemize}  
	\end{itemize}  
	\subsection{Violazioni GDPR}  
	\begin{itemize}  
		\item \textbf{Art. 13-14}: Mancata informativa sugli scopi del trattamento.  
		\item \textbf{Art. 22}: Decisioni automatizzate illegittime (es. licenziamento basato su algoritmi segreti).  
		\item \textbf{Art. 35}: Omessa Valutazione d’Impatto (DPIA) per trattamenti ad alto rischio.  
	\end{itemize}  
	\subsection{Sanzioni}  
	\begin{itemize}  
		\item Multe fino al 4\% del fatturato globale (Art. 83 GDPR).  
		\item Risarcimento danni agli interessati (Art. 82).  
	\end{itemize}  
	\subsection{Casi Reali}  
	\begin{itemize}  
		\item \textbf{Caso Meta (2023)}: Multa di 1,2 miliardi di euro per trasferimento dati USA-UE basato su clausole non trasparenti.  
		\item \textbf{Caso Clearview AI (2021)}: Sanzionata in UE per raccolta occulta di immagini biometriche.  
	\end{itemize}  
	\section{Tipi di dato}
	Nel contesto del Regolamento Generale sulla Protezione dei Dati (GDPR), si distinguono tre principali categorie di dati:
	\begin{enumerate}
		\item \textbf{Dati personali}: qualsiasi informazione riguardante una persona fisica identificata o identificabile, come il nome, l'indirizzo email o il numero di telefono
		\item \textbf{Dati particolari (ex sensibili)}: dati che rivelano l'origine razziale o etnica, opinioni politiche, convinzioni religiose o filosofiche, appartenenza sindacale, dati genetici, biometrici, relativi alla salute o alla vita sessuale e all'orientamento sessuale di una persona
		\item \textbf{Dati giudiziari}: informazioni relative a condanne penali, reati o misure di sicurezza ai sensi dell'art. 10 del GDPR
	\end{enumerate}
	\section{Sistemi Decisionali Automatizzati}
	Ai sensi dell’\textbf{Art. 22 GDPR}, i sistemi decisionali automatizzati sono strumenti che:  
	\begin{itemize}  
		\item \textbf{Processano dati personali} senza intervento umano (es. algoritmi di profiling, intelligenza artificiale).  
		\item \textbf{Prendono decisioni giuridiche o significative} per gli interessati (es. concessione di un prestito, assunzioni, valutazioni di credito).  
	\end{itemize}  
	\noindent \textbf{Esempi}:  
	\begin{itemize}  
		\item Piattaforme di \textit{recruiting} che filtrano CV con AI.  
		\item Sistemi bancari che approvano/rifiutano mutui basandosi su punteggi algoritmici.  
	\end{itemize}  
	\noindent \textbf{Vincoli GDPR}:  
	\begin{itemize}  
		\item \textit{Divieto generale} di decisioni solo automatizzate (\textbf{Art. 22(1)}), salvo eccezioni (consenso esplicito, necessità contrattuale).  
		\item Diritto dell’interessato a ottenere \textit{spiegazioni} (\textbf{Art. 13-15}) e a contestare la decisione.  
	\end{itemize}  
	\section{Quali sono i compiti del DPO e la responsabilità civile in generale e del service provider}
	Il \textbf{Data Protection Officer} (DPO), o Responsabile della Protezione dei Dati, è una figura prevista dal Regolamento (UE) 2016/679 (GDPR) con il compito di supportare il titolare e il responsabile del trattamento nel garantire il rispetto della normativa in materia di protezione dei dati personali. I suoi principali compiti includono:
	\begin{itemize}
		\item informare e consigliare il titolare o il responsabile del trattamento nonché i dipendenti circa gli obblighi derivanti dal GDPR;
		\item sorvegliare l'osservanza del regolamento e delle politiche interne in materia di protezione dei dati;
		\item fornire pareri in merito alla valutazione d'impatto sulla protezione dei dati (DPIA);
		\item cooperare con l'autorità di controllo e fungere da punto di contatto per essa.
	\end{itemize}
	Per quanto riguarda la \textbf{responsabilità civile}, in generale essa comporta l'obbligo di risarcire un danno cagionato a terzi per violazione di un dovere giuridico\\
	In ambito GDPR, sia il titolare sia il responsabile del trattamento sono responsabili per i danni causati da trattamenti illeciti\\
	Il \textbf{service provider} (responsabile del trattamento) risponde civilmente se non ha rispettato le istruzioni del titolare o ha agito in modo autonomo e illecito\\
	Tuttavia, può essere esonerato da responsabilità se dimostra di non essere in alcun modo responsabile dell’evento dannoso, ai sensi dell’art. 82 del GDPR
	\section{Definizioni GDPR}
	\subsection{Titolare del Trattamento (\textit{Data Controller})}
	Ai sensi dell'\textbf{Art. 4(7) GDPR}, il \textit{titolare del trattamento} è:
	\begin{quote}
		\textit{la persona fisica o giuridica, l'autorità pubblica, il servizio o altro organismo che, singolarmente o insieme ad altri, determina le finalità e i mezzi del trattamento di dati personali}.
	\end{quote}
	\noindent \textbf{Caratteristiche principali}:
	\begin{itemize}
		\item Decide \textbf{perché} (finalità) e \textbf{come} (mezzi) i dati sono trattati.
		\item Ha la \textbf{massima responsabilità} legale per il rispetto del GDPR.
		\item Può delegare l'esecuzione tecnica a un \textit{responsabile del trattamento} (Art. 28 GDPR).
	\end{itemize}
	\subsection{Responsabile del Trattamento (\textit{Data Processor})}
	Definito all'\textbf{Art. 4(8) GDPR} come:
	\begin{quote}
		\textit{la persona fisica o giuridica, l'autorità pubblica, il servizio o altro organismo che tratta dati personali per conto del titolare del trattamento}.
	\end{quote}
	\noindent \textbf{Differenze col Titolare}:
	\begin{itemize}
		\item \textbf{Ruolo}: Opera sotto le \textit{istruzioni} del titolare (es. società di hosting, servizi in cloud).
		\item \textbf{Responsabilità}: Deve garantire sicurezza e conformità, ma non decide le finalità del trattamento.
		\item \textbf{Obblighi}: Firmare un \textit{contratto di trattamento} (Art. 28) che specifichi limiti e garanzie.
	\end{itemize}
	\subsection{Interessato (\textit{Data Subject})}
	Per \textbf{Art. 4(1) GDPR} è:
	\begin{quote}
		\textit{la persona fisica i cui dati personali sono trattati} (es. cliente, dipendente, utente di un servizio).
	\end{quote}
	\noindent \textbf{Diritti principali}:
	\begin{itemize}
		\item Accesso, rettifica, cancellazione (\textit{right to be forgotten}).
		\item Opposizione al trattamento (es. marketing diretto).
		\item Portabilità dei dati (Art. 20 GDPR).
	\end{itemize}
	\subsection{Responsabile della Protezione dei Dati (\textit{DPO})}
	Figura introdotta dall'\textbf{Art. 37-39 GDPR}, obbligatoria in alcuni casi (es. autorità pubbliche, trattamenti su larga scala).  
	\noindent \textbf{Funzioni}:
	\begin{itemize}
		\item \textbf{Sorveglianza}: Monitora la conformità al GDPR.
		\item \textbf{Consulenza}: Fornisce pareri su valutazioni d'impatto (\textit{DPIAs}).
		\item \textbf{Punto di contatto}: Collabora con l'autorità garante (es. \textit{Garante Privacy italiano}).
	\end{itemize}
	\section{Contratti algoritmici}
	I contratti algoritmici, oppure smart contracts, sono una tipologia di contratto nella quale la volontà di una parte è formata ed eseguita dall'elaboratore elettronico mediante un algoritmo\\
	\begin{quote}
		Lo smart contract è \textit{"Un programma per elaboratore che opera su tecnologie basate su registri distribuiti la cui esecuzione vincola automaticamente due o più parti sulla base di effetti predefiniti tra le stesse}
	\end{quote}
	Vuol dire che è il software stesso a decidere al posto delle parti\\
	Un esempio può essere l'ordine di certi prodotti alla terminazione delle scorte nel magazzino\\
	Comunque sussiste la volontà iniziale del programmatore che ha creato l'algoritmo ma, nel caso di IA, può avvenire che con le capacità di auto apprendimento vengano compiute scelte parzialmente autonome o originali\\
	La responsabilità comunque gravano su chi usa l'agoritmo
	\section{Poteri Direttivi e di Controllo nel GDPR}
	Ai sensi dell’\textbf{Art. 28 GDPR}, il Titolare del Trattamento deve imporre al Responsabile (\textit{Processor}) obblighi contrattuali precisi, tra cui la possibilità di esercitare \textbf{poteri direttivi e di controllo}.  
	\noindent \textbf{Cosa includono?}  
	\begin{itemize}
		\item \textbf{Poteri Direttivi}:  
		\begin{itemize}
			\item \textit{Istruzioni vincolanti} su come trattare i dati (es. specifiche tecniche, limiti di conservazione).  
			\item \textit{Approvazione} di eventuali sub-responsabili (\textit{sub-processors}).  
			\item \textit{Modifiche unilaterali} al contratto (se previste) per adeguarsi al GDPR.  
		\end{itemize}  
		\item \textbf{Poteri di Controllo}:  
		\begin{itemize}
			\item \textit{Verifiche ispettive} (es. audit, richiesta di report sulla sicurezza).  
			\item \textit{Accesso ai dati} in qualsiasi momento (es. per rispondere a richieste degli interessati).  
			\item \textit{Valutazione delle misure di sicurezza} adottate dal Responsabile.  
		\end{itemize}  
	\end{itemize}  
	\noindent \textbf{Esempi pratici}:  
	\begin{itemize}
		\item Un’azienda (\textit{Titolare}) che usa un servizio di cloud storage (\textit{Responsabile}) può:  
		\begin{itemize}
			\item Imporre la crittografia dei dati (\textit{direttiva}).  
			\item Richiedere un audit annuale per verificare la compliance (\textit{controllo}).  
		\end{itemize}  
		\item Un ospedale (\textit{Titolare}) può vietare a un’azienda di software sanitario (\textit{Responsabile}) di trasferire dati al di fuori dell’UE.  
	\end{itemize}  
	\noindent \textbf{Rilevanza legale}:  
	L’esercizio di questi poteri è cruciale per:  
	\begin{itemize}
		\item \textit{Evitare sanzioni} (se il Responsabile viola il GDPR, il Titolare può essere ritenuto corresponsabile se non ha esercitato adeguato controllo).  
		\item \textit{Garantire la trasparenza} verso gli interessati (\textit{Art. 5 GDPR}).  
	\end{itemize}  
	\section{Il lavoro su piattaforma è regolato dal nostro ordinamento. La normativa li considera lavori autonomi o subordinati?}
	Nel corso degli ultimi anni è esplosa la richiesta dei \textit{lavori su richiesta (on demand)}\\
	Queste prestazioni sono caratterizzate dalla presenza di almeno tre soggetti:
	\begin{enumerate}
		\item cliente: richiede il servizio
		\item prestatore: esegue l'attività
		\item piattaforma: mette in relazione lavoratore e clientela
	\end{enumerate}
	Nel 2019, in Italia, è stata introdotta una normativa \textit{ad hoc} di qualificazione e protezione (parziale) dei "\textbf{rider}"\\
	I rider sono coloro che svolgono attività di consegna beni per conto altrui con l'ausilio di velocipedi o veicoli a motore, attraverso piattaforme anche digitali\\
	Le piattaforme digitali sono i programmi e le procedure informatiche utilizzati dal committente che, indipendentemente dal luogo di stabilimento, sono strumentali alle attività di consegna di beni, fissandone il compenso e determinando le modalità di esecuzione della prestazione\\
	Se il ciclofattorino si limita a collaborare con la piattaforma solo occasionalmente e con autonomia nelle modalità di esecuzione della prestazione, si instaura un rapporto genuino di lavoro autonomo
	\section{Il potere disciplinare del datore}
	Il potere disciplinare del datore di lavoro trova la sua fonte nelle seguenti norme:
	\begin{itemize}
		\item "L'inosservanza delle disposizioni contenute negli articoli riguardanti la violazione degli bblighi di lavorare con la prescritta diligenza e obbedienza e dell'obbligo di fedeltà possono dar luogo all'applicazione di sanzioni disciplinari, secondo la gravità dell'infrazione. Per applicarle, è necessaria contestazione dell'addebito e sentire la sua difesa. La multa non deve essere superiore a 4h e sospensione retribuzione non maggiore di 10 giorni e anche vietato mutamenti definitivi del rapporto di lavoro. In caso di licenziamento ingiusto la quantificazione richiede sempre l'analisi del caso concreto da parte di un professionista
		\item Il riferimento agli obbighi di diligenza e fedeltà richiama ogni sorta di inadempimento del lavoratore alle obbligazioni contrattuali, in quanto idoneo ad incidere, in modo disfunzionale, sull'organizzazione aziendale
	\end{itemize}
	Comunque il carattere extralavorativo non preclude la sanzionabilità in sede disciplinare
	\section{Cos'è la subordinazione? Contratto di lavoro subordinato}
	La subordinazione indica \textit{"Chi si obbliga mediante retribuzione a collaborare nell'impresa, prestando il proprio lavoro intellettuale o manuale alle dipendenze e sotto la direzione dell'imprenditore"}\\
	Il lavoratore subordinato deve osservare le disposizioni per l'esecuzione e la disciplina del lavoro impartite dall'imprenditore e dai collaboratori dai quali gerarchicamente dipende, definendo così la caratteristica essenziale, ovvero "eterodirezione dell'attività", il che vuol dire che la prestazione deve essere svolta nel modo imposto dal datore di lavoro, mediante ordini che il lavoratore è obbligato a rispettare\\
	L'imprenditore è definito come il capo dell'impresa e da lui dipendono gerarchicamente i suoi collaboratori\\
	Gli indici di subordinazione servono ad agevolare l'operazione di qualificazione del rapporto come lavoro subordinato, però hanno solo valore indicativo e non determinante:
	\begin{enumerate}
		\item inserimento nell'organizzazione
		\item vincolo di orario (la subordinazione attenuta è tipica delle prestazione intellettuali, come medici per case di cura private, dove è decisiva la timbratura del cartllino e l'obbligo di giustificare le assenze)
		\item esclusività del rapporto
		\item inerenza della prestazione al ciclo produttivo
		\item alienità dei mezzi di produzione
		\item retribuzione fissa a tempo senza rischio del risultato
	\end{enumerate}
	Alla fine, ogni qualificazione di contratto va preso come caso singolo e solo un giudice può stabilire di che tipo di contratto si tratta
	\section{Lavoro autonomo}
	\begin{quote}
		Chi si obbliga a compiere verso un corrispettivo un'opera o un servizio, con lavoro prevalentemente proprio e senza vincolo di subordinazione nei confronti del committente
	\end{quote}
	La caratteristica essenziale è l'assenza di etero-direzione: questo significa, concretamente, che l'attività lavorativa deve essere liberamente organizzata dal lavoratore autonomo\\
	Il lavoratore autonomo si distingue poi dall'imprenditore per la prevalenza del lavoro proprio rispetto all'organizzazione del lavoro altrui
	\section{Collaborazioni continuative e coordinate}
	A metà tra lavoro subordinato e autonomo\\
	La definizione riguarda \textit{"rapporti di collaborazione che si concretino in una prestazione di opera continuativa e coordinata, prevalentemente personale, anche se non a carattere subordinato}\\
	Si riguarda cnche il regime tributario dei redditi derivanti da \textit{"rapporti di collaborazione aventi per oggetto la prestazione di attività svolte senza vincolo di subordinazione a favore di un determinato soggetto nel quadro di un rapporto unitario e continuativo senza impiego di mezzi organizzati e con retribuzione periodica prestabilita"}\\
	\section{Contratto d'appalto e contratto d'opera}
	Il contratto d'appalto e quello d'opera rappresentano due figure contrattuali distinte con finalità affini ma discipline differenti\\
	L'appalto si caratterizza per l'obbligazione a raggiungere un risultato mediante l'organizzazione di mezzi e lavoro autonomi\\
	L'appaltatore assume il rischio dell'operazione e agisce con piena indipendenza nella gestione dell'attività, tipicamente fornendo sia i materiali che la manodopera\\
	Esempi classici sono le costruzioni edilizie o i lavori di ristrutturazione\\
	Il contratto d'opera, invece, si concentra esclusivamente sul risultato finale, con minore autonomia del prestatore\\
	Il rischio rimane in capo al committente, che spesso fornisce i materiali e dirige l'attività\\
	Tipico il caso del professionista che realizza un'opera su specifiche del cliente (es. progetto architettonico)\\
	La differenza fondamentale risiede nel grado di autonomia: mentre l'appaltatore opera come vero e proprio imprenditore, il prestatore d'opera agisce in posizione più subordinata\\
	La distinzione ha rilevanza cruciale per la qualificazione del rapporto, con conseguenze su responsabilità e tutela contrattuale\\
	La giurisprudenza considera proprio l'autonomia organizzativa come elemento dirimente per la classificazione
	\section{Controlli difensivi}
	Per controlli difensivi si intende un tipo di controllo attuato da superiori gerarchici per tutelare il patrimonio aziendale
	\section{Gamification}
	g
	\section{Domande su RRI (concezione procedurale/normativa)}
	g
	\section{Esempio di concorrenza di pasticcerie}
	g
\end{document}