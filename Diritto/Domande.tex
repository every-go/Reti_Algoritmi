\documentclass[10pt,oneside,a4paper]{article}
\usepackage{ragged2e,graphicx,tocloft,hyperref}
\title{Diritto informatica e società}
\author{Matteo Mazzaretto}
\date{2024/2025}
\hypersetup{
	colorlinks=true,   % Abilita il colore dei link (senza box intorno)
	linkcolor=blue,    % Colore per i link interni (come quelli nell'indice)
	urlcolor=red,      % Colore per i link URL esterni
	filecolor=magenta, % Colore per i link ai file locali
	citecolor=green,   % Colore per i riferimenti bibliografici
	pdfborder={0 0 0}  % Disabilita i bordi intorno ai link
}
\begin{document}
	% tolgo la numerazione in modo che la lunghezza dell'indice non incida sulla lunghezza del documento
	\pagenumbering{gobble}
	\maketitle
	\begin{center}
		%do un nuovo nome alla tabella degli indici e la inizializzo
		\renewcommand{\contentsname}{Indice}
		\tableofcontents
	\end{center}
	\newpage
	%inizio l'effettivo conteggio delle pagine
	\pagenumbering{arabic}
	\setcounter{page}{1}
	\section{GDPR e Diritti dell'interessato (Cos'è anche il principio di minimizzazione}
	g
	\section{Diritto di informazione del GDPR}
	g
	\section{Software e Protezione del Software}
	g
	\section{Open Software}
	g
	\section{Privacy by design/by default}
	g
	\section{Diritti dei beni e dell'ingegno}
	g
	\section{Tipi di dato}
	g
	\section{Contratti digitali}
	g
	\section{Contratti (definizione) e tipi di contratto}
	g
	\section{Differenza tra diritto di protezione alla privacy e del trattamento del dato personale}
	g
	\section{Contratti stipulati dai minorenni}
	g
	\section{Gamification}
	g
	\section{GDPR e soggetti coinvolti}
	g
	\section{Vari tipi di contratti}
	g
	\section{Poteri direttivi e di controllo}
	g
	\section{Copyright/Brevetto/Diritto d'autore}
	g
	\section{Successione del patrimonio digitale e il consenso come base giuridica (gdpr)}
	g
	\section{Quali sono i compiti del DPO e la responsabilità civile in generale e del service provider}
	g
	\section{Privacy sul posto di lavoro}
	g
	\section{Provider}
	g
	\section{Domande su RRI (concezione procedurale/normativa)}
	g
	\section{Illecito civile in particolare illecito extracontrattuale}
	g
	\section{Quali elementi ci devono essere per poter parlare di un illecito extracontrattuale}
	g
	\section{Esempio di concorrenza di pasticcerie}
	g
	\section{Condizioni generali di contratto}
	g
	\section{Panoramica delle fonti del diritto, definizione, produzione e cognizione, gerarchia}
	g
	\section{Le fonti europee quali sono?}
	g
	\section{Le leggi chi le fa?}
	g
	\section{Contratto collettivo}
	g
	\section{La definizione del titolare (data controller)}
	g
	\section{Elementi del contratto}
	g
	\section{Quando un contratto è invalido e quando è inefficace}
	g
	\section{Diritto della personalità}
	g
	\section{Il lavoro su piattaforma è regolato dal nostro ordinamento. La normativa li considera lavori autonomi o subordinati?}
	g
	\section{Diritto di proprietà}
	g
	\section{Cosa dice la regola del possesso vale titolo?}
	g
	\section{Come si interpreta il contratto?}
	g
	\section{Come si interpretano le leggi? Le analogie?}
	g
	\section{Cos'è la causa?}
	g
	\section{Cosa sono i sistemi decisionali di monitoraggio automatizzati?}
	g
	\section{Principio trasparenza GDPR}
	g
	\section{Cos'è la subordinazione? Contratto di lavoro subordinato}
	g
	\section{Capacità}
	g
	\section{Soggetti di diritto}
	g
	\section{L'autonomia patrimoniale imperfetta cosa implica?}
	g
	\section{Potere di controllo}
	g
	\section{Cos'è una base giuridia e quali sono le possibili basi giuridiche per trattamento dei dati}
	g
	\section{Illecito extracontrattuale}
	g
	\section{Proprietà}
	g
	\section{Controlli occulti}
	g
	\section{Il potere disciplinare del datore}
	g
	\section{Contratto d'appalto e contratto d'opera}
\end{document}