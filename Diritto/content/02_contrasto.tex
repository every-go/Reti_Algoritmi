\section{Cosa succede in caso di contrasti fra diverse leggi}

	Nel caso di contrasti fra leggi rimane:
	\begin{enumerate}
		\item quella più in alto nella gerarchia (quando due norme contrastano, si applica il criterio gerarchico: la norma superiore prevale su quella inferiore. Tuttavia, quando il contrasto avviene tra fonti di pari grado, si applicano altri criteri come la cronologia o la competenza)
		\item fra stesse fonti, la più recente (stesso legislatore)
		\item fra leggi statali/regionali, ci si basa sul criterio di competenza
	\end{enumerate}
	Il principio generale è che una nuova legge disciplina solo i fatti successivi alla sua entrata in vigore (principio di irretroattività)\\
	Tuttavia, in diritto penale si applica sempre la norma più favorevole al reo (principio del favor rei)\\
	Se una norma viene dichiarata invalida, si considera nulla sin dall'origine (efficacia retroattiva), ma questo avviene solo in determinati casi, ad esempio quando viene dichiarata incostituzionale dalla Corte Costituzionale