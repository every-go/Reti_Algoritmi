\section{Panoramica delle fonti del diritto, definizione, produzione e cognizione, gerarchia}

   Le fonti del diritto sono tutti gli atti e fatti che l'ordinamento giuridico riconosce come idonei a produrre norme giuridiche\\
   In particolare, si distinguono in fonti normative e fonti non normative\\
   \textbf{Fonti normative:} atti che producono direttamente norme giuridiche, tra cui la Costituzione, le leggi ordinarie, i regolamenti e le consuetudini\\
   \textbf{Fonti non normative:} atti che non producono direttamente norme giuridiche, ma sono utili per la conoscenza delle stesse, come la pubblicazione sulla Gazzetta Ufficiale, in questo caso si parla di fatti di cognizione\\
   Atti: Costituzione, legge, regolamento.\\
   Fatti: consuetudine (comportamento posto in essere dalla generalità di un'organizzazione per un periodo di tempo indefinito, in grado di acquisire valore normativo).\\
   La gerarchia delle fonti del diritto in Italia presenta cinque livelli principali:\\
   \begin{enumerate}
      \item Costituzione\\
      \item Trattati UE e legislazione comunitaria\\
      \item Leggi statali, decreti legislativi, decreti-legge, leggi regionali\\
      \item Regolamenti\\
      \item Usi e consuetudini
   \end{enumerate}