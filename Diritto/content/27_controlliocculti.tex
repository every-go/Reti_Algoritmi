\section{Privacy sul posto di lavoro e controlli occulti}

	\subsection{Definizione}  
	I \textbf{controlli occulti} (o \textit{monitoraggi nascosti}) sono attività di sorveglianza o analisi dei dati personali condotte:  
	\begin{itemize}  
		\item \textbf{Senza trasparenza}: L’interessato non è informato del trattamento (violando l’Art. 5(1)(a) GDPR, principio di liceità e correttezza).  
		\item \textbf{Senza base giuridica valida}: Spesso privi di consenso, necessità contrattuale o interesse legittimo proporzionato.  
	\end{itemize}  
	\subsection{Esempi}  
	\begin{itemize}  
		\item \textbf{Workplace monitoring}:  
		\begin{itemize}  
			\item Tracciamento tastiera/schermo dei dipendenti senza avviso.  
			\item Analisi delle email aziendali con AI per valutare performance.  
		\end{itemize}  
		\item \textbf{Profilazione online}:  
		\begin{itemize}  
			\item Raccolta dati di navigazione tramite cookie nascosti.  
			\item Uso di fingerprinting del browser senza consenso.  
		\end{itemize}  
	\end{itemize}  
	\subsection{Violazioni GDPR}  
	\begin{itemize}  
		\item \textbf{Art. 13-14}: Mancata informativa sugli scopi del trattamento.  
		\item \textbf{Art. 22}: Decisioni automatizzate illegittime (es. licenziamento basato su algoritmi segreti).  
		\item \textbf{Art. 35}: Omessa Valutazione d’Impatto (DPIA) per trattamenti ad alto rischio.  
	\end{itemize}  
	\subsection{Sanzioni}  
	\begin{itemize}  
		\item Multe fino al 4\% del fatturato globale (Art. 83 GDPR).  
		\item Risarcimento danni agli interessati (Art. 82).  
	\end{itemize}  
	\subsection{Casi Reali}  
	\begin{itemize}  
		\item \textbf{Caso Meta (2023)}: Multa di 1,2 miliardi di euro per trasferimento dati USA-UE basato su clausole non trasparenti.  
		\item \textbf{Caso Clearview AI (2021)}: Sanzionata in UE per raccolta occulta di immagini biometriche.  
	\end{itemize}  