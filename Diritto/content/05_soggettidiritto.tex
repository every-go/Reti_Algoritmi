\section{Soggetti di diritto}

	I soggetti di diritto sono coloro che possono essere titolari di diritti e doveri giuridici\\	
	Si distinguono in:
	\begin{itemize}
		\item \textbf{Persone fisiche}: ogni essere umano è un soggetto di diritto sin dalla nascita	
		\item \textbf{Persone giuridiche ed enti collettivi}:
		\begin{itemize}
			\item \textbf{Persone giuridiche} (con autonoma personalità giuridica):
			\begin{itemize}
				\item \textit{Private}:
				\begin{itemize}
					\item Associazioni riconosciute
					\item Fondazioni
					\item Società di capitali (S.p.A., S.r.l.)
				\end{itemize}
				\item \textit{Pubbliche}:
				\begin{itemize}
					\item Stato, Regioni, Comuni
					\item Enti pubblici (INPS, Università)
				\end{itemize}
			\end{itemize}	
			\item \textbf{Gruppi senza personalità giuridica}:
			\begin{itemize}
				\item Associazioni non riconosciute
				\item Comitati
				\item Società di persone (S.n.c., S.a.s.)
			\end{itemize}
		\end{itemize}
	\end{itemize}	
	Il concetto di diritto può essere distintio in due accezioni principali:
	\begin{itemize}
		\item \textbf{Diritto soggettivo}: posizione giuridica attiva di un soggetto (es. proprietà, credito)
		\item \textbf{Diritto oggettivo}: insieme delle norme giuridiche che regolano la società
	\end{itemize}