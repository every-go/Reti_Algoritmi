\section{Copyright/Brevetto/Diritto d'autore}

	Il diritto d'autore deriva dall'art. 2575 del Codice Civile:
	\begin{quote}
		Formano oggetto del diritto di autore le opere dell'ingegno di carattere creativo che appartengono alle scienze, alla letteratura, alla musica, alle arti figurative, all'architettura, al teatro e alla cinematografia, qualunque ne sia il modo o la forma di espressione.
	\end{quote}
	Il diritto d'autore si compone di due elementi principali:
	\begin{enumerate}
		\item \textbf{Diritto morale}:  
		\begin{itemize}
			\item È il diritto della personalità dell'autore, riconosciuto come irrinunciabile
			\item Permette di difendere l'integrità e la paternità dell'opera, anche dopo la morte dell'autore
		\end{itemize}
		\item \textbf{Diritto patrimoniale}:  
		\begin{itemize}
			\item Consente lo sfruttamento economico dell'opera, tutelando lo sforzo creativo dell'autore
			\item Comprende la facoltà di autorizzare o vietare la riproduzione, distribuzione, noleggio e comunicazione al pubblico dell'opera
			\item Ha una durata di protezione che si estende fino a 70 anni dopo la morte dell'autore
			\item Può essere ceduto per contratto o per successione (mortis causa)
			\item Il principio di \textit{esaurimento} limita il diritto patrimoniale: una volta venduto un prodotto, il titolare non può controllarne il successivo commercio, fatta eccezione per la regolamentazione dei download digitali
		\end{itemize}
	\end{enumerate}
	Il brevetto, disciplinato dall'art. 2585 del Codice Civile, protegge le nuove invenzioni destinate ad avere un'applicazione industriale. In particolare:
	\begin{quote}
		Possono costituire oggetto di brevetto le nuove invenzioni atte ad avere un'applicazione industriale, quali un metodo o un processo di lavorazione industriale, una macchina, uno strumento, un utensile o un dispositivo meccanico, un prodotto o un risultato industriale e l'applicazione tecnica di un principio scientifico, purché essa dia immediati risultati industriali.
	\end{quote}
	Le caratteristiche principali del brevetto sono:
	\begin{itemize}
		\item \textbf{Sfruttamento economico esclusivo}: il titolare del brevetto può escludere terzi dallo sfruttamento dell'invenzione
		\item \textbf{Durata}:
		\begin{itemize}
			\item Invenzioni industriali: durata massima di 20 anni, non rinnovabile
			\item Modelli d'utilità: durata di 10 anni
			\item Disegni (2D) e modelli (3D): durata iniziale di 5 anni, rinnovabile fino a un massimo di 5 volte
		\end{itemize}
		\item \textbf{Requisiti di brevettabilità}:  
		Per essere brevettabile, un'invenzione deve essere:
		\begin{enumerate}
			\item Innovativa
			\item Dotata di applicazione industriale
			\item Non divulgata al pubblico prima della presentazione della domanda di brevetto (novità)
		\end{enumerate}
	\end{itemize}
	Il termine \textit{copyright} è spesso utilizzato in senso intercambiabile con il diritto d'autore, sebbene in alcuni contesti vi siano delle distinzioni:
	\begin{itemize}
		\item \textbf{Ambito di applicazione}:  
		Il copyright, in particolare, si riferisce all'insieme dei diritti patrimoniali che tutelano l'opera e ne regolano l'utilizzo economico, come la riproduzione, la distribuzione e la comunicazione al pubblico
		\item \textbf{Tutela internazionale}:  
		Il concetto di copyright è ampiamente riconosciuto a livello internazionale e assume sfumature specifiche a seconda delle normative dei vari paesi
		\item \textbf{Differenza concettuale}:  
		Mentre il diritto d'autore comprende sia la componente morale (irrinunciabile e legata alla personalità dell'autore) sia quella patrimoniale, il copyright è spesso inteso come l'insieme delle tutele economiche derivanti dal diritto patrimoniale
	\end{itemize}