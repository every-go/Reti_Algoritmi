\section{Cos'è la subordinazione? Contratto di lavoro subordinato}

	La subordinazione indica \textit{"Chi si obbliga mediante retribuzione a collaborare nell'impresa, prestando il proprio lavoro intellettuale o manuale alle dipendenze e sotto la direzione dell'imprenditore"}\\
	Il lavoratore subordinato deve osservare le disposizioni per l'esecuzione e la disciplina del lavoro impartite dall'imprenditore e dai collaboratori dai quali gerarchicamente dipende, definendo così la caratteristica essenziale, ovvero "eterodirezione dell'attività", il che vuol dire che la prestazione deve essere svolta nel modo imposto dal datore di lavoro, mediante ordini che il lavoratore è obbligato a rispettare\\
	L'imprenditore è definito come il capo dell'impresa e da lui dipendono gerarchicamente i suoi collaboratori\\
	Gli indici di subordinazione servono ad agevolare l'operazione di qualificazione del rapporto come lavoro subordinato, però hanno solo valore indicativo e non determinante:
	\begin{enumerate}
		\item inserimento nell'organizzazione
		\item vincolo di orario (la subordinazione attenuta è tipica delle prestazione intellettuali, come medici per case di cura private, dove è decisiva la timbratura del cartllino e l'obbligo di giustificare le assenze)
		\item esclusività del rapporto
		\item inerenza della prestazione al ciclo produttivo
		\item alienità dei mezzi di produzione
		\item retribuzione fissa a tempo senza rischio del risultato
	\end{enumerate}
	Alla fine, ogni qualificazione di contratto va preso come caso singolo e solo un giudice può stabilire di che tipo di contratto si tratta