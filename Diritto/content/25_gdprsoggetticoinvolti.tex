\section{GDPR e soggetti coinvolti}

	I  principali soggetti della Privacy sono: l’interessato, il titolare del trattamento dei dati, il Responsabile del trattamento, il DPO, il terzo e l’Autorità di Controllo
	\begin{enumerate}
		\item Interessato: la persona fisica cui si riferiscono i dati personali
		\item Titolare del Trattamento: la persona fisica, la società, l’associazione o un’altra entità che controlla il trattamento dei dati personali ed è autorizzata a prendere decisioni essenziali sulle finalità e modalità di tale trattamento, comprese le misure di sicurezza applicabili
		\item Responsabile del Trattamento: la persona fisica, la società, l’associazione o l’organizzazione a cui il Titolare ha affidato l’attività specifica di gestione e controllo dei dati personali, in base all’esperienza e/o alle competenze pertinenti in materia
		\item DPO (Data Protection Officer): il professionista con conoscenze specialistiche sulla legislazione e sulle pratiche in materia di protezione dei dati.
		\item Terzo: la persona fisica o giuridica, l’autorità pubblica, il servizio o qualsiasi altro organismo che non sia l’interessato, il titolare del trattamento, il responsabile del trattamento. E’ una persona autorizzata al trattamento dei dati sotto l’autorità diretta del titolare o del responsabile
		\item Autorità di Controllo: l’autorità pubblica indipendente istituita da uno Stato membro, nel nostro caso l’Autorità è il Garante per la protezione dei dati personali
	\end{enumerate}