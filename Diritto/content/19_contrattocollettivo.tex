\section{Contratto collettivo}

	Il \textbf{contratto collettivo} è un accordo giuridico stipulato tra una o più organizzazioni sindacali dei lavoratori e una o più associazioni datoriali, al fine di regolare le condizioni di lavoro di una categoria di lavoratori o di un determinato settore economico\\
	Il contratto collettivo stabilisce le norme relative a vari aspetti del rapporto di lavoro, tra cui la retribuzione, l’orario di lavoro, le ferie, la sicurezza, le modalità di assunzione e licenziamento, e altri diritti e doveri reciproci tra le parti\\
	Esistono due tipi principali di contratto collettivo:
	\begin{itemize}
		\item \textbf{Contratto collettivo nazionale di lavoro (CCNL)}: regolamenta i rapporti di lavoro a livello nazionale e si applica a tutti i lavoratori di un determinato settore o categoria, indipendentemente dal datore di lavoro
		\item \textbf{Contratto collettivo aziendale o territoriale}: si applica a una singola azienda o a un territorio specifico, e può integrare o modificare le disposizioni del CCNL in relazione alle esigenze particolari dell’impresa o della zona
	\end{itemize}
	Il contratto collettivo ha forza di legge per le parti che lo sottoscrivono e, quando esteso, anche per i lavoratori e i datori di lavoro non aderenti alle organizzazioni firmatarie. L'eventuale violazione del contratto collettivo può comportare sanzioni legali