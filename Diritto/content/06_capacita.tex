\section{Capacità}

	Esistono due principali tipi di capacità: \textbf{capacità giuridica} e \textbf{capacità di agire}\\
	La capacità giuridica è l'attitudine a essere titolari di diritti e doveri\\
	Si acquisisce al momento della nascita e spetta a chiunque senza distinzioni\\
	In via eccezionale, l'ordinamento attribuisce alcuni diritti patrimoniali anche a soggetti non ancora nati (\textit{nascituri}) nei casi di:  
	\begin{itemize}
		\item Testamento  
		\item Donazione  
	\end{itemize}
	In questi casi, i diritti sono condizionati alla nascita del soggetto\\
	La capacità di agire è la capacità di esercitare personalmente i diritti di cui si è titolari, disponendone mediante atti giuridici come vendita, donazione o contratti\\
	Poiché gli atti possono avere conseguenze giuridiche rilevanti, la capacità di agire si acquista al compimento del \textbf{18° anno di età}\\
	Dai 18 anni si possono compiere validamente atti giuridici, ma in alcuni casi si possono compiere anche da minorenni (es. apertura di un conto corrente bancario può richiedere autorizzazione)\\
	Gli atti compiuti da un minore possono essere annullabili su richiesta del rappresentante legale\\
	La capacità di agire si distingue in:  
	\begin{enumerate}
		\item \textbf{Capacità naturale}: è la capacità di intendere e di volere, ossia di comprendere le proprie azioni e assumerne la responsabilità. È rilevante per determinare la validità di alcuni atti e la responsabilità civile o penale. 
		\item \textbf{Capacità legale}: è il riconoscimento formale della capacità di agire. I maggiorenni sono titolari di capacità legale, ma in alcuni casi possono essere dichiarati incapaci o sottoposti a limitazioni
	\end{enumerate}
	L'ordinamento prevede misure di protezione per i soggetti incapaci:  
	\begin{enumerate}
		\item \textbf{Minore di età}: non può compiere atti giuridici autonomamente, salvo eccezioni (es. acquisti di modesta entità). Gli atti sono compiuti dai genitori o da chi esercita la responsabilità genitoriale  
		\item \textbf{Inabilitazione}: misura che limita parzialmente la capacità di agire. Il soggetto inabilitato può compiere alcuni atti da solo, ma per quelli più rilevanti è affiancato da un \textbf{curatore}
		\item \textbf{Interdizione}: comporta la totale incapacità di agire. Gli atti giuridici vengono compiuti per suo conto da un \textbf{tutore} 
		\item \textbf{Amministrazione di sostegno}: misura più flessibile, introdotta per assistere persone con ridotta autonomia. L’amministratore di sostegno opera secondo le disposizioni stabilite dal giudice nel provvedimento di nomina
	\end{enumerate}
	La legge disciplina i poteri e i doveri di tutori, curatori e amministratori di sostegno, imponendo controlli e limiti per tutelare gli interessi dell’incapace.  