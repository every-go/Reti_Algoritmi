\section{Contratto d'appalto e contratto d'opera}

   Il contratto d'appalto e quello d'opera rappresentano due figure contrattuali distinte con finalità affini ma discipline differenti\\
   L'appalto si caratterizza per l'obbligazione a raggiungere un risultato mediante l'organizzazione di mezzi e lavoro autonomi\\
   L'appaltatore assume il rischio dell'operazione e agisce con piena indipendenza nella gestione dell'attività, tipicamente fornendo sia i materiali che la manodopera\\
   Esempi classici sono le costruzioni edilizie o i lavori di ristrutturazione\\
   Il contratto d'opera, invece, si concentra esclusivamente sul risultato finale, con minore autonomia del prestatore\\
   Il rischio rimane in capo al committente, che spesso fornisce i materiali e dirige l'attività\\
   Tipico il caso del professionista che realizza un'opera su specifiche del cliente (es. progetto architettonico)\\
   La differenza fondamentale risiede nel grado di autonomia: mentre l'appaltatore opera come vero e proprio imprenditore, il prestatore d'opera agisce in posizione più subordinata\\
   La distinzione ha rilevanza cruciale per la qualificazione del rapporto, con conseguenze su responsabilità e tutela contrattuale\\
   La giurisprudenza considera proprio l'autonomia organizzativa come elemento dirimente per la classificazione