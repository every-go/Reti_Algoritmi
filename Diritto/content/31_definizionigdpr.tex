\section{Definizioni GDPR}

	\subsection{Titolare del Trattamento (\textit{Data Controller})}
	Ai sensi dell'\textbf{Art. 4(7) GDPR}, il \textit{titolare del trattamento} è:
	\begin{quote}
		\textit{la persona fisica o giuridica, l'autorità pubblica, il servizio o altro organismo che, singolarmente o insieme ad altri, determina le finalità e i mezzi del trattamento di dati personali}.
	\end{quote}
	\noindent \textbf{Caratteristiche principali}:
	\begin{itemize}
		\item Decide \textbf{perché} (finalità) e \textbf{come} (mezzi) i dati sono trattati.
		\item Ha la \textbf{massima responsabilità} legale per il rispetto del GDPR.
		\item Può delegare l'esecuzione tecnica a un \textit{responsabile del trattamento} (Art. 28 GDPR).
	\end{itemize}
	\subsection{Responsabile del Trattamento (\textit{Data Processor})}
	Definito all'\textbf{Art. 4(8) GDPR} come:
	\begin{quote}
		\textit{la persona fisica o giuridica, l'autorità pubblica, il servizio o altro organismo che tratta dati personali per conto del titolare del trattamento}.
	\end{quote}
	\noindent \textbf{Differenze col Titolare}:
	\begin{itemize}
		\item \textbf{Ruolo}: Opera sotto le \textit{istruzioni} del titolare (es. società di hosting, servizi in cloud).
		\item \textbf{Responsabilità}: Deve garantire sicurezza e conformità, ma non decide le finalità del trattamento.
		\item \textbf{Obblighi}: Firmare un \textit{contratto di trattamento} (Art. 28) che specifichi limiti e garanzie.
	\end{itemize}
	\subsection{Interessato (\textit{Data Subject})}
	Per \textbf{Art. 4(1) GDPR} è:
	\begin{quote}
		\textit{la persona fisica i cui dati personali sono trattati} (es. cliente, dipendente, utente di un servizio).
	\end{quote}
	\noindent \textbf{Diritti principali}:
	\begin{itemize}
		\item Accesso, rettifica, cancellazione (\textit{right to be forgotten}).
		\item Opposizione al trattamento (es. marketing diretto).
		\item Portabilità dei dati (Art. 20 GDPR).
	\end{itemize}
	\subsection{Responsabile della Protezione dei Dati (\textit{DPO})}
	Figura introdotta dall'\textbf{Art. 37-39 GDPR}, obbligatoria in alcuni casi (es. autorità pubbliche, trattamenti su larga scala)\\
	\noindent \textbf{Funzioni}:
	\begin{itemize}
		\item \textbf{Sorveglianza}: Monitora la conformità al GDPR.
		\item \textbf{Consulenza}: Fornisce pareri su valutazioni d'impatto (\textit{DPIAs}).
		\item \textbf{Punto di contatto}: Collabora con l'autorità garante (es. \textit{Garante Privacy italiano}).
	\end{itemize}