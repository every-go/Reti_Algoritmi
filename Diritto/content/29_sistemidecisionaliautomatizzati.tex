\section{Sistemi Decisionali Automatizzati}

	Ai sensi dell’\textbf{Art. 22 GDPR}, i sistemi decisionali automatizzati sono strumenti che:  
	\begin{itemize}  
		\item \textbf{Processano dati personali} senza intervento umano (es. algoritmi di profiling, intelligenza artificiale).  
		\item \textbf{Prendono decisioni giuridiche o significative} per gli interessati (es. concessione di un prestito, assunzioni, valutazioni di credito).  
	\end{itemize}  
	\noindent \textbf{Esempi}:  
	\begin{itemize}  
		\item Piattaforme di \textit{recruiting} che filtrano CV con AI.  
		\item Sistemi bancari che approvano/rifiutano mutui basandosi su punteggi algoritmici.  
	\end{itemize}  
	\noindent \textbf{Vincoli GDPR}:  
	\begin{itemize}  
		\item \textit{Divieto generale} di decisioni solo automatizzate (\textbf{Art. 22(1)}), salvo eccezioni (consenso esplicito, necessità contrattuale).  
		\item Diritto dell’interessato a ottenere \textit{spiegazioni} (\textbf{Art. 13-15}) e a contestare la decisione.  
	\end{itemize}  