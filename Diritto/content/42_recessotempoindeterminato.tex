\section{Recesso contratto a tempo indeterminato}

	Bisogna dare il preavviso nel termine e nei modi stabiliti dalle norme corporative, dagli usi o secondo equità\\
	In mancanza di preavviso il recedente è tenuto verso l'altra parte a un'indennità equivalente all'importo della retribuzione che sarebbe spettata per il periodo di preavviso\\
	La stessa indennità è dovuta dal datore di lavoro nel caso di cessazione del rapporto per morte del prestatore di lavoro\\
	Si può rescindere unilateralmente senza preavviso qualora si verifichi una causa che non consenta la prosecuzione (sia oggettivo che soggettivo), anche provvisoria, del rapporto\\
	Modo di licenziare:
	\begin{enumerate}
		\item giusta causa: inadempimento notevole, giustificato motivo soggettivo, bisogna pagare inadempimento/preavviso
		\item massimo inadempimento: giusta causa, licenziamento immediato, no pagare inadempimento
		\item giustificato motivo oggettivo: non dipende dal comportamento (esempio: cambio sede)
	\end{enumerate}