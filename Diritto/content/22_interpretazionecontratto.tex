\section{Come si interpreta il contratto?}

	L'interpretazione del contratto è il processo attraverso il quale si stabilisce il senso e l'ambito delle dichiarazioni e delle obbligazioni contenute nel contratto stesso\\
	Secondo l'art. 1362 del \textbf{codice civile italiano}, il contratto deve essere interpretato secondo la comune intenzione delle parti, e non solo secondo il significato letterale delle parole\\
	In particolare, si seguono alcuni principi fondamentali per determinare l'effettivo significato dell'accordo:
	\begin{itemize}
		\item \textbf{Intenzione delle parti}: il contratto deve essere interpretato alla luce dell'intenzione comune delle parti, anche se tale intenzione non risulta espressamente indicata nel testo. L'intenzione si deduce dalle circostanze del contratto e dal comportamento delle parti
		\item \textbf{Significato letterale}: il significato delle parole deve essere valutato in base al loro uso corrente, salvo che nel contratto non emergano elementi che indichino un significato diverso
		\item \textbf{Interpretazione sistematica}: il contratto deve essere interpretato nel contesto dell’intero accordo, evitando che una singola clausola venga letta isolatamente
		\item \textbf{Principio di buona fede}: il contratto deve essere interpretato in buona fede, nel rispetto delle regole di correttezza e leale cooperazione tra le parti
	\end{itemize}
	Nel caso in cui, nonostante l’applicazione di questi principi, rimangano dubbi interpretativi, si ricorre al \textbf{giudizio del tribunale} per risolvere la questione, tenendo conto anche della prassi consolidata o delle consuetudini del settore