\section{Le fonti europee quali sono?}

	La comunità internazionale è formata da Stati sovrani posti in posizione di reciproca parità\\
	L'ordinamento internazionale è costituito da consuetudini, convenzioni e organizzazioni degli Stati\\
	Una particolare comunità internazionale a cui appartiene l'Italia è il Consiglio d'Europa, che ha come obiettivo principale la tutela dei diritti umani\\
	L'organo giudiziario del Consiglio d'Europa è la Corte europea dei diritti dell’uomo (CEDU), che si occupa di garantire il rispetto della Convenzione europea dei diritti dell'uomo (CEDU) da parte degli Stati membri\\
	Le fonti normative dell'Unione Europea derivano principalmente da:  
	\begin{enumerate}
		\item \textbf{Trattati}:  
		\begin{itemize}
			\item \textbf{Trattati istitutivi}, che creano e regolano il funzionamento dell’Unione Europea (ad esempio, il Trattato di Roma, il Trattato di Maastricht, il Trattato di Lisbona)
			\item \textbf{Trattati di modifica}, che aggiornano o integrano i trattati istitutivi
		\end{itemize}
		\item \textbf{Principi generali del diritto dell'UE}, ricavati dalle tradizioni giuridiche comuni degli Stati membri e dal diritto dell’Unione
		\item \textbf{Carta dei diritti fondamentali dell'UE}, che raccoglie i diritti e le libertà fondamentali riconosciuti nell’ordinamento UE
	\end{enumerate}
	Gli atti adottati dalle istituzioni dell'Unione Europea si distinguono in:  
	\begin{enumerate}
		\item \textbf{Atti tipici}:
		\begin{enumerate}
			\item \textbf{Vincolanti}:
			\begin{itemize}
				\item \textbf{Regolamento}: ha effetto immediato negli Stati membri senza necessità di recepimento nazionale.
				\item \textbf{Direttiva}: vincola gli Stati membri al raggiungimento di un obiettivo, ma lascia loro la libertà di scegliere i mezzi più idonei per attuarla attraverso la legislazione nazionale.
				\item \textbf{Decisione}: obbligatoria per i destinatari (che possono essere Stati o soggetti specifici).
			\end{itemize}
			\item \textbf{Non vincolanti}:
			\begin{itemize}
				\item \textbf{Raccomandazione}: suggerisce un comportamento agli Stati membri senza obbligo giuridico.
				\item \textbf{Parere}: esprime una valutazione su una questione, senza effetti vincolanti.
			\end{itemize}
		\end{enumerate}
		\item \textbf{Atti atipici}:  
		\begin{itemize}
			\item Accordi interistituzionali
			\item Dichiarazioni comuni
			\item Comunicazioni
			\item Codici di condotta
			\item Libri verdi e libri bianchi
		\end{itemize}
	\end{enumerate}