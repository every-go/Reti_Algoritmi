\section{Diritti dei beni e dell'ingegno}

	I beni possono essere:
	\begin{enumerate}
		\item mobili: tutto ciò che non rientra nelle successive definizioni
		\item immobili: beni con vincolo naturale o artificiale che li collega al suolo (fiumi, torrenti, edifici)
		\item beni mobili registrati: beni mobili soggetti a registrazione in particolare registri (motoveicoli)
	\end{enumerate}
	Costituiscono beni anche quelli privi di materialità come software, energia elettrica, brevetto, registrazione marchio\\
	Si può distinguere ulteriormente i beni fra:
	\begin{enumerate}
		\item beni fungibili: sostituibili (denaro per ecccellenza)
		\item beni infungibili: non sostituibili
	\end{enumerate}
	I beni \textit{fruttiferi} sono beni che producono altri beni, i quali vengono chiamati \textit{frutti}, che possono essere naturali o civili\\
	I diritti dell'ingegno sono quelli descritti dall'art 2575 cc:\\
	\begin{quote}
		Formano oggetto del diritto di autore le opere dell'ingegno di carattere creativo che appartengono alle scienze, alla letteratura, alla musica, alle 
		arti figurative, all'architettura, al teatro e alla cinematografia, qualunque ne sia il modo o la forma di espressione\\
	\end{quote}
	Ad essere protetti non sono le idee base delle opere bensì la forma di espressione di tale idee, questo perché il modo in cui l'artista esprime è ciò che viene proiettato\\
	Si può suddividire in:
	\begin{enumerate}
		\item opere d'ingegno: fa parte il software che è quindi soggetto a diritto d'autore e copyright, possono anche essere brevettati come invenzioni industriali 
		\item invenzioni industriali
		\item modelli industriali
	\end{enumerate}