\section{Diritto della personalità}

	Il diritto della personalità è disciplinato dall'articolo 2 della Costituzione italiana, il quale afferma:
	\begin{quote}
		"La Repubblica riconosce e garantisce i diritti inviolabili dell'uomo, sia come singolo, sia nelle formazioni sociali ove si svolge la sua personalità, e richiede l'adempimento dei doveri inderogabili di solidarietà politica, economica e sociale"
	\end{quote}
	I principali diritti della personalità sono:  
	\begin{enumerate}
		\item \textbf{Diritto al nome}  
		\item \textbf{Diritto all'immagine}  
		\item \textbf{Diritto morale d'autore}  
		\item \textbf{Diritto all'onore, reputazione e decoro}  
		\item \textbf{Diritto all'identità personale}  
		\item \textbf{Diritto all'integrità fisica}  
		\item \textbf{Diritto alla salute (e autodeterminazione terapeutica)}  
		\item \textbf{Diritto alla privacy}  
		\item \textbf{Diritto alla dignità}  
	\end{enumerate}
	Presentano le seguenti caratteristiche:  
	\begin{enumerate}
		\item \textbf{Innati e connaturati alla persona}: spettano all'individuo sin dalla nascita
		\item \textbf{Non patrimoniali}: non hanno contenuto economico diretto
		\item \textbf{Indisponibili}: non possono essere trasmessi, rinunciati o alienati
		\item \textbf{Imprescrittibili}: non si estinguono con il passare del tempo
	\end{enumerate}
	Le situazioni giuridiche soggettive possono acquisirsi in due modi:  
	\begin{itemize}
		\item \textbf{Acquisto originario}: il diritto nasce in capo al soggetto senza che vi sia un precedente titolare (es. diritto alla vita, alla salute)
		\item \textbf{Acquisto derivativo}: il diritto si trasferisce da un individuo all'altro attraverso atti giuridici come compravendita, successione ereditaria, donazione
	\end{itemize}