\section{Base Giuridica per il Trattamento dei Dati}

	\subsection{Definizione}
	La \textbf{base giuridica} (o \textit{legal basis}) è il fondamento legittimo che autorizza il trattamento dei dati personali, ai sensi dell’\textbf{Art. 6 GDPR}\\
	Ogni trattamento deve avere una base giuridica valida, esplicitata \textit{prima} della raccolta dei dati (Art. 13-14 GDPR)\\
	\noindent \textbf{Requisiti chiave}:
	\begin{itemize}
		\item \textbf{Necessaria}: Senza base giuridica, il trattamento è illecito (Art. 5 GDPR).
		\item \textbf{Documentata}: Deve essere indicata nell’informativa privacy.
		\item \textbf{Non retroattiva}: Non può essere applicata a trattamenti già avvenuti.
	\end{itemize}
	\subsection{Le 6 Basi Giuridiche del GDPR}
	\begin{enumerate}
		\item \textbf{Consenso dell’interessato} (Art. 6(1)(a)) \\
		\textit{Esempio}: Iscrizione a una newsletter tramite spunta esplicita. \\
		\textbf{Limiti}: Deve essere \textit{libero, specifico, informato} e \textit{revocabile} (Art. 7). Non valido per rapporti di forza squilibrati (es. datore di lavoro-dipendente).\\
		\item \textbf{Esecuzione di un contratto} (Art. 6(1)(b)) \\
		\textit{Esempio}: Trattamento dati per consegnare un ordine su un e-commerce. \\
		\textbf{Nota}: Vale solo per dati \textit{strettamente necessari} all’adempimento contrattuale.\\
		\item \textbf{Obbligo legale} (Art. 6(1)(c)) \\
		\textit{Esempio}: Invio dati fiscali all’Agenzia delle Entrate. \\
		\textbf{Requisito}: Deve derivare da una norma dell’UE o degli Stati membri (es. antiriciclaggio).\\
		\item \textbf{Interesse vitale dell’interessato} (Art. 6(1)(d)) \\
		\textit{Esempio}: Condivisione di dati medici in un’emergenza sanitaria. \\
		\textbf{Applicabilità}: Solo in casi eccezionali (es. pericolo di vita).\\
		\item \textbf{Interesse legittimo del titolare} (Art. 6(1)(f)) \\
		\textit{Esempio}: Prevenzione delle frodi in un servizio online. \\
		\textbf{Bilanciamento}: Richiede una \textit{valutazione proporzionalità} tra interesse del titolare e diritti dell’interessato (considerando 49 GDPR).\\ \textit{Non applicabile} a dati particolari (Art. 9).\\
		\item \textbf{Missione di interesse pubblico} (Art. 6(1)(e)) \\
		\textit{Esempio}: Trattamento dati da parte di scuole o enti pubblici. \\
		\textbf{Fonte}: Deve essere previsto da una legge nazionale/UE (es. sanità pubblica).\\
	\end{enumerate}
	\subsection{Successione del patrimonio digitale e consenso nel GDPR}
	La successione del patrimonio digitale riguarda la trasmissione dei dati personali di un defunto agli eredi\\
	Ai sensi del GDPR:
	\begin{itemize}
		\item Il \textbf{consenso} (Art. 6(1)(a)) non è trasferibile, poiché strettamente personale.
		\item Gli eredi possono esercitare diritti solo se:
		\begin{itemize}
			\item Il trattamento si basa su altre basi giuridiche (es. contratto, Art. 6(1)(b)).
			\item La legge nazionale lo preveda (es. Italia con Art. 2-terdecies D.lgs. 101/2018).
		\end{itemize}
	\end{itemize}