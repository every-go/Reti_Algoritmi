\section{L'autonomia patrimoniale}

	L'autonomia patrimoniale è il principio secondo cui un ente collettivo dispone di un proprio patrimonio separato da quello delle persone fisiche che lo compongono\\
	Essa determina in che misura i beni dell'ente e quelli dei soci o membri restano distinti in caso di obbligazioni e responsabilità\\
	L'autonomia patrimoniale può essere:  
	\begin{enumerate}
		\item \textbf{Perfetta}: il patrimonio dell'ente è completamente separato da quello dei soci o membri. Ciò significa che i creditori dell'ente possono soddisfarsi solo sui beni dell'ente stesso, senza intaccare il patrimonio personale dei soci. Questo vale, ad esempio, per le società di capitali (S.p.A., S.r.l.)
		\item \textbf{Imperfetta}: i soci rispondono anche con il proprio patrimonio personale per le obbligazioni dell'ente. Questo accade nelle società di persone (S.n.c., S.a.s. per i soci accomandatari), dove almeno alcuni soci hanno responsabilità illimitata per i debiti sociali
	\end{enumerate}
	La distinzione tra autonomia patrimoniale perfetta e imperfetta è fondamentale per comprendere il regime di responsabilità in caso di insolvenza:  
	\begin{itemize}
		\item Se una persona fisica ha debiti personali, i creditori non possono aggredire il patrimonio dell'ente di cui è socio o membro.  
		\item Se invece è l'ente collettivo ad avere debiti, i creditori possono soddisfarsi sul suo patrimonio. Tuttavia, in presenza di autonomia patrimoniale imperfetta, possono rivalersi anche sui beni personali dei soci responsabili illimitatamente.  
	\end{itemize}