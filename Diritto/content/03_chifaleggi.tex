\section{Le leggi: chi le fa?}

	In Italia, il potere legislativo è esercitato dal Parlamento, composto dalla Camera dei deputati e dal Senato della Repubblica.  \\
	Le leggi possono essere proposte da:  
	\begin{itemize}
		\item Governo  
		\item Parlamentari  
		\item Consigli regionali
	\end{itemize}
	Il procedimento legislativo segue il principio del bicameralismo perfetto: una legge deve essere approvata nello stesso testo da entrambe le Camere\\  
	Ogni Camera approva il testo con maggioranza semplice (50\%+1 dei presenti, con almeno il 50\% dei componenti presenti)\\
	Se il Senato modifica il testo, la legge torna alla Camera per una nuova approvazione\\
	Le leggi di revisione costituzionale e altre leggi particolari richiedono una doppia approvazione da parte delle Camere, con almeno tre mesi di intervallo tra le due votazioni. 
	\begin{itemize}
		\item Se approvate con una maggioranza di almeno i \(\frac{2}{3}\) dei membri di ciascuna Camera, entrano in vigore direttamente.  
		\item Se approvate con una maggioranza compresa tra \(\frac{1}{2}\) e \(\frac{2}{3}\), possono essere sottoposte a referendum confermativo se richiesto da almeno 500.000 elettori, 5 Consigli regionali o 1/5 dei membri di una Camera.  
	\end{itemize}
	Oltre alle leggi ordinarie, il Governo può emanare atti con forza di legge in due casi particolari:  
	\begin{itemize}
		\item \textbf{Decreto legislativo (d.lgs.)}: il Parlamento delega il Governo a legiferare su una materia specifica tramite una \textit{legge delega}, che stabilisce principi e criteri direttivi. Il Governo, seguendo tali direttive, emana il decreto legislativo.  
		\item \textbf{Decreto-legge (d.l.)}: il Governo, in casi straordinari di necessità e urgenza, può emanare un decreto con forza di legge, che entra in vigore immediatamente. Tuttavia, deve essere convertito in legge dal Parlamento entro 60 giorni, altrimenti perde efficacia.  
	\end{itemize}