\section{Illecito civile in particolare illecito extracontrattuale, facendo riferimento agli elementi necessari}

	L'illecito civile è disciplinato dall'articolo 1218 c.c.:
	\begin{quote}
		Il debitore che non esegue esattamente la prestazione dovuta è tenuto al risarcimento del danno...
	\end{quote}
	Riguarda quindi l'effettivo inadempimento del contratto\\
	Invece, l'illecito extracontrattuale è disciplinato dall'art 2043 cc:
	\begin{quote}
		Qualunque fatto doloso o colposo, che cagiona ad altri un danno ingiusto, obbliga colui che ha commesso il fatto a risarcire il danno
	\end{quote}
	Disciplina l'illecito extracontrattuale, e per essere applicabile devono esserci 4 elementi suddivisi fra soggettivi e oggettivi:
	\begin{itemize}
		\item Elementi oggettivi:
		\begin{enumerate}
			\item danno ingiusto: va a ledere interesse giuridico tutelato, come concorrenza sleale, violazione diritto soggettivo, lesione diritto personalità, diritto di credito. Vengono esclusi legittima difesa e stato di casualità/necessità
			\item nesso causale: legame di causa-effetto tra il comportamento (o l’omissione) dell’agente e il danno subito dalla vittima
		\end{enumerate}
		\item Elementi soggettivi:
		\begin{enumerate}
			\item imputabilità: presente nell'art 2046, non risponde delle conseguenze del fatto dannoso che in quel momento non aveva capacità di intendere e volere non per sua colpa. Se il danno avviene dall'incapace naturale risponde il sorvegliante, se esso non è presente il giudice, prese in considerazione le condizioni economiche delle parti, può condannare l'autore del danno ad un'equa indennità
			\item colpevolezza: il danno è risarcibile solo se con dolo o con colpa
		\end{enumerate}
	\end{itemize}