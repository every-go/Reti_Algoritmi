\section{Quali sono i compiti del DPO e la responsabilità civile in generale e del service provider}

	Il \textbf{Data Protection Officer} (DPO), o Responsabile della Protezione dei Dati, è una figura prevista dal Regolamento (UE) 2016/679 (GDPR) con il compito di supportare il titolare e il responsabile del trattamento nel garantire il rispetto della normativa in materia di protezione dei dati personali. I suoi principali compiti includono:
	\begin{itemize}
		\item informare e consigliare il titolare o il responsabile del trattamento nonché i dipendenti circa gli obblighi derivanti dal GDPR
		\item sorvegliare l'osservanza del regolamento e delle politiche interne in materia di protezione dei dati
		\item fornire pareri in merito alla valutazione d'impatto sulla protezione dei dati (DPIA)
		\item cooperare con l'autorità di controllo e fungere da punto di contatto per essa
	\end{itemize}
	In ambito GDPR, sia il titolare sia il responsabile del trattamento sono responsabili per i danni causati da trattamenti illeciti\\
	Il \textbf{service provider} (responsabile del trattamento) risponde civilmente se non ha rispettato le istruzioni del titolare o ha agito in modo autonomo e illecito\\
	Tuttavia, può essere esonerato da responsabilità se dimostra di non essere in alcun modo responsabile dell’evento dannoso, ai sensi dell’art. 82 del GDPR