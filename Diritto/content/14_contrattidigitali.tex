\section{Contratti digitali}

	I \textbf{contratti digitali} sono accordi stipulati tra due o più parti mediante strumenti informatici e reti telematiche\\
	Ai sensi dell'art. 1321 del codice civile, un contratto è l'accordo tra due o più parti per costituire, regolare o estinguere un rapporto giuridico patrimoniale, e tale definizione si applica anche ai contratti conclusi in formato digitale\\
	La normativa italiana riconosce la validità dei contratti digitali, purché siano rispettati i requisiti di forma, consenso e capacità giuridica delle parti\\
	La modalità di sottoscrizione elettronica dei documenti è distinta tra firma elettronica semplice, avanzata e qualificata\\
	\begin{enumerate}
		\item Semplice: messaggio email con foto di una firma autografa
		\item Elettronica avanzata: richiesta in banca per autorizzare operazioni telematiche, richiedono particolari tablet con pennino che registrano movimento ed intensità (firma grafometrica), immutabile ed è possibile ricondurre la firma alla persona. Il limite è che viene usata solo nel contesto
		\item Elettronica qualificata: al momento contiene solo la firma digitale, ovvero una firma generabile esclusivamente con prodotti tecnologici che lo permettono forniti da gestori detti classificatori, come Aruba, garantendo immutabilità e provenienza
	\end{enumerate}
	La firma elettronica qualificata ha valore legale equivalente alla firma autografa, rendendo pienamente valido e opponibile il contratto digitale\\
	Tuttavia, anche le altre forme di firma elettronica possono essere utilizzate, purché siano idonee a garantire l’identificabilità del firmatario e l’integrità del documento\\	