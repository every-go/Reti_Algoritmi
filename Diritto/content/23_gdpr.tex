\section{GDPR e Diritti dell'interessato}

	Il GDPR (regolamento generale per la protezione dei dati) disciplina riservatezza (libertà sull'utilizzo dei dati del cittadino) e protezione (non utilizzare dati altrui per propri scopi)\\
	I principi dell'interessato sono:
	\begin{enumerate}
		\item liceità: impresa che tratta i dati deve avere la valida base giuridica (consenso soggetto, protezione interessi vitali, adempimento compito pubblico interesse, legittimo interesse aziendale) per il trattamento
		\item correttezza: i dati devono essere trattati in modo corretto ed equo in modo da regolarizzare gli algoritmi
		\item trasparenza: l'individuo deve rendersi conto del come e del perché del trattamento dei dati
		\item minimizzazione: devono essere trattati meno dati possibili
		\item memoria limitata: i dati storici possono essere conservati ma con "soggetto anonimizzato"
		\item integrità e confidenzialità: devono essere introdotte misure per impedire l'accesso ai dati ad estranei
		\item protezione dei dati \textit{By Design} (tutto a norma sia per lavoratori che per clienti) e \textit{By Default} (il minimo ammontare di dati necessario è collezionato e usato di default)
	\end{enumerate}