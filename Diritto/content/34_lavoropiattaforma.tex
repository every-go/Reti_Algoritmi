\section{Il lavoro su piattaforma è regolato dal nostro ordinamento. La normativa li considera lavori autonomi o subordinati?}

	Nel corso degli ultimi anni è esplosa la richiesta dei \textit{lavori su richiesta (on demand)}\\
	Queste prestazioni sono caratterizzate dalla presenza di almeno tre soggetti:
	\begin{enumerate}
		\item cliente: richiede il servizio
		\item prestatore: esegue l'attività
		\item piattaforma: mette in relazione lavoratore e clientela
	\end{enumerate}
	Nel 2019, in Italia, è stata introdotta una normativa \textit{ad hoc} di qualificazione e protezione (parziale) dei "\textbf{rider}"\\
	I rider sono coloro che svolgono attività di consegna beni per conto altrui con l'ausilio di velocipedi o veicoli a motore, attraverso piattaforme anche digitali\\
	Le piattaforme digitali sono i programmi e le procedure informatiche utilizzati dal committente che, indipendentemente dal luogo di stabilimento, sono strumentali alle attività di consegna di beni, fissandone il compenso e determinando le modalità di esecuzione della prestazione\\
	Se il ciclofattorino si limita a collaborare con la piattaforma solo occasionalmente e con autonomia nelle modalità di esecuzione della prestazione, si instaura un rapporto genuino di lavoro autonomo