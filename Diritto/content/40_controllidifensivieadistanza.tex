\section{Controlli difensivi e a distanza}

	Per controlli difensivi si intende un tipo di controllo attuato da superiori gerarchici per tutelare il patrimonio aziendale, difendendosi da fatti gravi\\
	È ammesso il controllo datoriale a distanza sull'operato dei dipendenti esclusivamente per esigenze imprenditoriali\\
	L'utilizzazione di impianti audiovisivi è consentita per almeno una delle tre seguenti finalità:
	\begin{enumerate}
		\item esigenze organizzative e produttive
		\item sicurezza del lavoro
		\item tutela del patrimonio aziendale
	\end{enumerate}
	Per l'installazione di telecamere di sicurezza è necessario l'accordo sindacale o l'autorizzazione amministrativa o previa autorizzazione dell'Ispettorato del lavoro\\
	Per gli impianti di geolocalizzazione/registrazione delle presenze non è invece necessario\\
	Ad ogni modo, i dati raccolti in maniera illeggittima non sono utilizzabili dal datore di lavoro come fondamento di sanzione disciplinare\\
	I controlli effettuati aldifuori dei requisiti descritti sono solo per fatti gravi:
	\begin{enumerate}
		\item "senso lato": controllo a protezione del patrimonio aziendale che riguarda tutti i dipendenti
		\item "senso stretto": controlli che riguardano il lavoratore singolo, mirato a verificare condotte illecite dei singoli lavoratori, da svolgere dopo il sospetto fondato (devono esistere fatti concreti) e deve bilanciare tutela del patrimonio e dignità lavoratore, rispettando principi generali e disposizioni precetture della normativa privacy, fra le quali rispetto della propria vita privata e familiare, domicilio e corrispondenza
	\end{enumerate}
	Le 3 categorie di soggetti che possono esercitare controllo sul lavoro:
	\begin{enumerate}
		\item guardie giurate: solo per tutela patrimonio aziendale
		\item personale sorveglianza: nominativi e mansioni specifiche comunicate ai lavoratori
		\item agenzie investigative: solo per frodi, assenteismo prolungato, finta malattia, attività concorrenziale, controlli anche fuori da luogo di lavoro
	\end{enumerate}