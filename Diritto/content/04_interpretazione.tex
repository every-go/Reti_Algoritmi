\section{Come si interpretano le leggi? E le analogie?}

	Chiunque lavori con le norme giuridiche svolge necessariamente un ruolo interpretativo, ossia deve attribuire un significato preciso alle disposizioni legislative\\
	Poiché l'interpretazione delle leggi potrebbe essere influenzata dalla soggettività, l'ordinamento giuridico stabilisce criteri per garantire coerenza nell'interpretazione\\
	Questi criteri sono stabiliti nell'articolo 12 delle \textit{Disposizioni sulla legge in generale} (Preleggi)\\
	L'articolo 12 delle preleggi stabilisce che:  
	\begin{enumerate}
		\item \textbf{Interpretazione letterale e logica}: nell'applicare la legge, non si può attribuirle un significato diverso da quello reso evidente dal senso proprio delle parole, tenendo conto della loro connessione logica e dell'intenzione del legislatore
		\item \textbf{Interpretazione sistematica e teleologica}: se il significato letterale non è sufficiente, si considera l'intenzione del legislatore al momento della promulgazione della norma e la coerenza della norma con il sistema giuridico complessivo
	\end{enumerate}
	A volte, la legge può risultare imprecisa:  
	\begin{itemize}
		\item può esprimere più di quanto il legislatore intendeva dire
		\item può dire meno di quanto necessario per regolare un caso concreto  
	\end{itemize}
	I principali criteri interpretativi sono:  
	\begin{enumerate}
		\item \textbf{Interpretazione letterale}: analisi del significato proprio delle parole
		\item \textbf{Interpretazione logica}: considerazione del contesto logico della norma
		\item \textbf{Interpretazione sistematica}: valutazione della connessione della norma con altre disposizioni
		\item \textbf{Interpretazione teleologica}: ricerca dell'intenzione del legislatore al momento della promulgazione della legge
	\end{enumerate}Se manca una norma specifica per un caso concreto, si ricorre all'analogia, che può essere di due tipi:  
	\begin{enumerate}
		\item \textbf{Analogia legis}: si applica al caso non regolato una norma prevista per un caso simile
		\item \textbf{Analogia iuris}: se non esiste una norma specifica, si ricavano principi generali dall'intero ordinamento giuridico
	\end{enumerate}
	L'analogia non è ammessa in diritto penale e in altre materie di stretta interpretazione, poiché potrebbe violare il principio di legalità. 