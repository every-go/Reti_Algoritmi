\section{Il potere disciplinare del datore}

	Il potere disciplinare del datore di lavoro trova la sua fonte nelle seguenti norme:
	\begin{itemize}
		\item "L'inosservanza delle disposizioni contenute negli articoli riguardanti la violazione degli obblighi di lavorare con la prescritta diligenza e obbedienza e dell'obbligo di fedeltà possono dar luogo all'applicazione di sanzioni disciplinari, secondo la gravità dell'infrazione. Per applicarle, è necessaria contestazione dell'addebito e sentire la sua difesa. La multa non deve essere superiore a 4h e sospensione retribuzione non maggiore di 10 giorni e anche vietato mutamenti definitivi del rapporto di lavoro. In caso di licenziamento ingiusto la quantificazione richiede sempre l'analisi del caso concreto da parte di un professionista
		\item Il riferimento agli obbighi di diligenza e fedeltà richiama ogni sorta di inadempimento del lavoratore alle obbligazioni contrattuali, in quanto idoneo ad incidere, in modo disfunzionale, sull'organizzazione aziendale
	\end{itemize}
	Comunque il carattere extralavorativo non preclude la sanzionabilità in sede disciplinare