\section{Poteri Direttivi e di Controllo nel GDPR}

	Ai sensi dell’\textbf{Art. 28 GDPR}, il Titolare del Trattamento deve imporre al Responsabile (\textit{Processor}) obblighi contrattuali precisi, tra cui la possibilità di esercitare \textbf{poteri direttivi e di controllo}\\  
	\noindent \textbf{Cosa includono?}  
	\begin{itemize}
		\item \textbf{Poteri Direttivi}:  
		\begin{itemize}
			\item \textit{Istruzioni vincolanti} su come trattare i dati (es. specifiche tecniche, limiti di conservazione).  
			\item \textit{Approvazione} di eventuali sub-responsabili (\textit{sub-processors}).  
			\item \textit{Modifiche unilaterali} al contratto (se previste) per adeguarsi al GDPR.  
		\end{itemize}  
		\item \textbf{Poteri di Controllo}:  
		\begin{itemize}
			\item \textit{Verifiche ispettive} (es. audit, richiesta di report sulla sicurezza).  
			\item \textit{Accesso ai dati} in qualsiasi momento (es. per rispondere a richieste degli interessati).  
			\item \textit{Valutazione delle misure di sicurezza} adottate dal Responsabile.  
		\end{itemize}  
	\end{itemize}  
	\noindent \textbf{Esempi pratici}:  
	\begin{itemize}
		\item Un’azienda (\textit{Titolare}) che usa un servizio di cloud storage (\textit{Responsabile}) può:  
		\begin{itemize}
			\item Imporre la crittografia dei dati (\textit{direttiva}).  
			\item Richiedere un audit annuale per verificare la compliance (\textit{controllo}).  
		\end{itemize}  
		\item Un ospedale (\textit{Titolare}) può vietare a un’azienda di software sanitario (\textit{Responsabile}) di trasferire dati al di fuori dell’UE.  
	\end{itemize}  
	\noindent \textbf{Rilevanza legale}:  
	L’esercizio di questi poteri è cruciale per:  
	\begin{itemize}
		\item \textit{Evitare sanzioni} (se il Responsabile viola il GDPR, il Titolare può essere ritenuto corresponsabile se non ha esercitato adeguato controllo).  
		\item \textit{Garantire la trasparenza} verso gli interessati (\textit{Art. 5 GDPR}).  
	\end{itemize}