\section{Diritto di proprietà}

	La proprietà è disciplinata dall'articolo 832 del Codice Civile:
	\begin{quote}
		Il proprietario ha diritto di godere e disporre delle cose in modo pieno ed esclusivo, entro i limiti e con l'osservanza degli obblighi stabiliti dall'ordinamento giuridico.
	\end{quote}
	Nel diritto privato sussiste il divieto di compiere atti emulativi, ovvero il proprietario non può compiere azioni con il solo scopo di arrecare molestie ad altri\\
	Inoltre, il diritto di proprietà può essere limitato per motivi di pubblica utilità, ad esempio mediante l'espropriazione\\
	Esistono due modi di acquisizione del diritto di proprietà:
	\begin{enumerate}
		\item \textbf{Titolo derivativo}: il contratto trasla il diritto di proprietà dal vecchio al nuovo proprietario (ad esempio, in caso di successione ereditaria)
		\item \textbf{Titolo originario}: il diritto di proprietà sorge direttamente, ad esempio, attraverso l'invenzione, la creazione o altri istituti giuridici
	\end{enumerate}
	Sebbene la proprietà non si acquisisca per prescrizione ordinaria, essa può essere oggetto di \textit{usucapione}: se un soggetto esercita il possesso continuativo e in buona fede per il periodo previsto dalla legge, e se il proprietario non rivendica il proprio diritto, quest'ultimo può venir meno\\
	Perché l'usucapione sia possibile è necessario il concetto di \textit{possesso}, così definito dall'articolo 1140 del Codice Civile:
	\begin{quote}
		Il possesso è il potere sulla cosa che si manifesta in un'attività corrispondente all'esercizio della proprietà o di altro diritto reale. Si può possedere direttamente o per mezzo di altra persona, che ha la detenzione della cosa.
	\end{quote}