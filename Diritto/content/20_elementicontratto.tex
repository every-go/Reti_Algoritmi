\section{Elementi del contratto}

	Gli elementi del contratto sono:
	\begin{enumerate}
		\item Accordo: almeno 2 parti si devono mettere d'accordo al fine di regolare, concordare e istituire il contratto, tramite due parti principali, ovvero la "proposta ed accetazione" e "autonomia contrattuale" (libero arbitrio nella forma creando potenzialmente ulteriori tipi di contratto purché siano meritevoli dell'interesse giuridico)
		\item Causa: funzione del contratto (NON MOTIVO), nel caso della compravendita è lo scambio. Deve essere lecito (se illecito porta all'annullamento del contratto). Un esempio è la funzione di riposo nel contratto per alloggiare un posto in un hotel allo scopo di fare una vacanza
		\item Oggetto: prestazioni in cui le parti si vincolano e beni oggetto di quelle prestazioni. Deve essere lecito, possibile (se impossibile porta a contratto nullo) e determinato o determinabile
		\item Forma: modo di dichiarare una volontà. Di regola è libera ma è richiesta una determinata forma (scritta/digitale) per la prova del contratto
	\end{enumerate}