\section{Contratti stipulati dai minorenni}

	I \textbf{minorenni}, ai sensi dell'art. 2 del codice civile, non hanno la piena capacità di agire e, pertanto, non possono validamente stipulare contratti, se non nei casi previsti dalla legge\\
	In linea generale, i contratti conclusi da un minore sono \textit{annullabili}, salvo che siano stipulati dal rappresentante legale (solitamente un genitore o un tutore)\\
	Tuttavia, esistono eccezioni in cui il minore può compiere atti giuridici validi:
	\begin{itemize}
		\item atti di ordinaria amministrazione, se ritenuti proporzionati alla sua età e maturità;
		\item contratti favorevoli (ad esempio, una donazione), che non comportano obblighi a suo carico;
		\item esercizio di attività lavorativa, se autorizzata e conforme alle leggi sul lavoro minorile.
	\end{itemize}
	Per gli atti eccedenti l’ordinaria amministrazione, è generalmente necessaria l’autorizzazione del giudice tutelare