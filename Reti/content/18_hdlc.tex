\section{HDLC}
\subsection{Descrizione}
Protocollo concreto ideato inizialmente dall'IBM\\
L'HDLC (High-Level Data Link Control) è un protocollo standard è un protocollo del livello data link\\
Si basa su una struttura a frame (trama) e supporta comunicazioni punto-punto e multipunto\\
Il protocollo si occupa principalmente di:\\
Organizzazione dei dati\\
Affidabilità: fornisce meccanismi di rilevamento e correzione degli errori\\
Controllo del flusso: evita la congestione gestendo il ritmo della trasmissione
\subsection{Ambiti d'uso}
Attualmente viene usato per modem/fax, reti di vario tipo (come reti LAN  e WAN) e molti circuiti bancari
\subsection{Pregi}
Affidabilità nella trasmissione:\\
HDLC utilizza meccanismi di rilevamento e correzione degli errori tramite checksum (Cyclic Redundancy Check, CRC), riducendo significativamente il rischio di errori nella trasmissione\\
Versatilità:\\
supporta configurazioni punto-punto e multipunto, rendendolo adatto a diverse architetture di rete\\
Controllo di flusso efficiente:\\
implementa tecniche di acknowledgment e gestione della finestra scorrevole per garantire una trasmissione fluida ed evitare congestioni\\
Efficienza nella trasmissione continua:\\
HDLC utilizza frame strutturati con overhead minimo, ottimizzando la trasmissione di dati su collegamenti con capacità elevate
\subsection{Difetti}
Complessità di implementazione:\\
nonostante la standardizzazione, HDLC richiede un'implementazione relativamente complessa per gestire tutte le sue funzioni (controllo di flusso, rilevamento errori, configurazioni multipunto)\\
Overhead di controllo:\\
anche se il protocollo è efficiente, l'uso di campi di controllo, intestazioni e sequenze di frame introduce un overhead che può influire negativamente su collegamenti a bassa velocità\\
Non ottimale per reti moderne:\\
è stato progettato per linee seriali e reti tradizionali; potrebbe non essere ideale per reti moderne basate su Ethernet o reti wireless, dove altri protocolli sono più efficienti\\
Limiti nella gestione degli errori:\\
sebbene HDLC rilevi errori, non sempre riesce a correggerli\\
In caso di errore, il frame deve essere ritrasmesso, aumentando il ritardo
