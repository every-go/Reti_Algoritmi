\section{QPSK (2015)}
\subsection{Descrizione}
QPSK vuol dire "Quadrature phase shift keying) e indica lo spostamento di fase delle onde con la chiave con 4 intervalli simmetrici: 45°, 135°, 225°, 315°\\
Ogni stato rappresenta un simbolo, consentendo la trasmissione di 2 bit per simbolo\\
Ciò rende il QPSK il doppio più efficiente rispetto alla modulazione BPSK (Binary Phase Shift Keying) in termini di bit trasmessi per baud
\subsection{Ambiti d'uso}
Si usa per modulare il segnale digitale in modo da far funzionare correttamente l'infrastruttura telefonica tra cui reti mobili (3G, 4G), sistemi satellitari per comunicazioni bidirezionali, nella televisione digitale e nel Wi-Fi
\subsection{Pregi}
Efficienza spettrale: Raddoppia il numero di bit trasmessi rispetto a tecniche più semplici come BPSK\\
Robustezza: È meno suscettibile al rumore rispetto a modulazioni con più simboli, come QAM-16\\
Facilità di implementazione grazie alla semplicità dei circuiti richiesti\\
\subsection{Difetti}
Più aumentano i simboli più sono simili causando problemi nelle telecomunicazioni\\
Ci sono limitazioni in ambienti rumorosi senza tecniche avanzate di correzione degli errori\\
Efficienza limitata, sicuramente inferiore alla QAM
