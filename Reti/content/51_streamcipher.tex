\section{Stream cipher (2015, 16, 18, 19, 20, 23, 24)}
\subsection{Descrizione}
I cifrari a flusso cifrano i dati un bit o un byte alla volta\\
A differenza dei cifrari a blocchi, che operano su intere unità di dati (blocchi), i cifrari a flusso utilizzano una chiave segreta per generare una sequenza di bit che viene combinata con il testo in chiaro tramite XOR\\
Sono più adatti per applicazioni che richiedono una cifratura continua, come nei canali di comunicazione
\subsection{Ambiti d'uso}
\begin{enumerate}
\item Comunicazioni in tempo reale: utilizzato in ambienti come il VoIP (Voice over IP) e altre forme di comunicazione in tempo reale
\item Protezione dei dati mobili: implementato in dispositivi mobili per garantire la cifratura continua dei dati
\item Trasmissioni in tempo reale: impiegato per la cifratura di flussi di dati in tempo reale in applicazioni come video streaming e giochi online
\end{enumerate}
\subsection{Pregi}
\begin{enumerate}
\item Alta velocità: i cifrari a flusso sono veloci, poiché operano su singoli bit o byte, rendendoli ideali per applicazioni che richiedono elevata velocità di cifratura
\item Efficienza in spazi limitati: essendo basati su una chiave continua, i cifrari a flusso sono ideali per dispositivi con risorse di calcolo limitate
\item Adatti per flussi di dati infiniti: sono perfetti per applicazioni dove il flusso di dati non è predefinito o è continuo nel tempo
\end{enumerate}
\subsection{Difetti}
\begin{enumerate}
\item Vulnerabilità al riutilizzo della chiave: se la stessa chiave viene riutilizzata, i dati cifrati diventano vulnerabili agli attacchi
\item Difficoltà di implementazione sicura: un'implementazione scorretta può esporre la chiave segreta a possibili attacchi
\end{enumerate}
