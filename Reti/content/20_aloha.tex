\section{Aloha (2014, 17, 18, 19, 21, 23)}
\subsection{Descrizione}
Aloha è un tipo di protocollo multiaccesso, ovvero quei sistemi di comunicazione multipla in cui c'è un unico canale condiviso da molti (contention), che lo occuperanno in momenti diversi\\
Ogni protocollo di questo tipo ha la station model (entità che trasmettono) e le collision (quando 2 frame si sovrappongono c'è una collisione e sono inutilizzabili)\\
Questo protocollo sfrutta le probabilità, infatti, nell'eventualità di una collisione, il tempo di ritrasmissione è deciso dalle probabilità\\
La probabilità che k frames siano generati in un certo intervallo di tempo è di tipo Poisson e viene studiata con la sua distribuzione\\
Grazie a questa si ottiene che con Aloha "classico" si ottiene un 18,4$\%$ di banda che ha il pregio di non dipendere dal numero di trasmissioni contemporanee\\
Invece, con lo slotted Aloha (in cui la trasmissione è permessa solo all'inizio di uno slot) si arriva a un 36,8$\%$ di banda\\
Questo tipo di protocollo non ha il carrier sense (la stazione non può analizzare il canale finché non lo usa)
\subsection{Ambiti d'uso}
Reti wireless a bassa complessità:\\
è stato utilizzato nelle reti satellitari e nelle prime reti mobili, dove la semplicità era una priorità\\
Reti di sensori:\\
può essere applicato in reti di sensori distribuiti, dove i nodi trasmettono dati solo sporadicamente\\
Sistemi di comunicazione a bassa velocità:\\
ideale per applicazioni a basso traffico e ridotta necessità di coordinamento
\subsection{Pregi}
Semplicità di implementazione:\\
il protocollo non richiede una gestione complessa o una sincronizzazione rigorosa tra i nodi\\
Distribuzione decentralizzata:\\
ogni dispositivo agisce in modo autonomo, rendendo il sistema flessibile e adatto a reti distribuite\\
Adattabilità:\\
può essere facilmente implementato in sistemi con traffico intermittente o poco intenso\\
Robustezza:\\
in reti con pochi nodi o basso traffico, ALOHA funziona bene, garantendo una trasmissione rapida dei dati
\subsection{Difetti}
Efficienza bassa:\\
Nel Pure ALOHA, l'efficienza massima teorica è del 18$\%$ (1/2e), a causa delle collisioni frequenti\\
Nello Slotted ALOHA, l'efficienza migliora ma si ferma a circa il 37$\%$\\
Gestione delle collisioni:\\
non esiste un meccanismo preventivo per evitare collisioni; ciò comporta ritrasmissioni e perdita di tempo e risorse\\
Non adatto a traffico elevato:\\
quando il numero di utenti cresce, il tasso di collisioni aumenta esponenzialmente, rendendo il protocollo inefficace\\
Latenza alta nei casi peggiori:\\
con molte collisioni, i tempi di ritrasmissione possono aumentare significativamente, generando latenza
