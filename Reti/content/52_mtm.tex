\section{MTM}
\subsection{Descrizione}
MTM (Man In The Middle) si riferisce a un tipo di attacco in cui un attaccante modifica i dati scambiati tra due parti senza il loro consenso\\
Questi attacchi possono compromettere l'integrità e l'autenticità delle informazioni, inoltre i dati possono essere rubati dagli attaccanti\\
Per essere contrastato, sono stati ideati i CA (Certification Authority) che certificano che l'unità con cui si sta parlando è effettivamente quella vera e non qualcuno che finge di essere ad esempio una banca
\subsection{Ambiti d'uso}
\begin{enumerate}
\item Attacchi contro i sistemi di comunicazione sicura: utilizzati per compromettere l'integrità dei dati trasmessi su reti sicure
\item Furto di dati sensibili: Gli attacchi MTM possono essere utilizzati per alterare o rubare dati sensibili
\end{enumerate}
\subsection{Pregi}
\begin{enumerate}
\item Facilità di esecuzione: gli attacchi MTM non richiedono necessariamente una conoscenza avanzata delle tecniche di hacking
\item Alta efficacia: gli attacchi MTM possono compromettere rapidamente la sicurezza dei sistemi target
\end{enumerate}
\subsection{Difetti}
\begin{enumerate}
\item Contrasto tramite crittografia: l'uso di crittografia e firme digitali può prevenire gli attacchi MTM
\item Alta visibilità in ambienti protetti: gli attacchi MTM sono più facili da rilevare in ambienti sicuri e ben protetti
\end{enumerate}
