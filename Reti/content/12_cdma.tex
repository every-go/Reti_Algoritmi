\section{CDMA (Code Division Multiple Access) (2016, 18)}
\subsection{Descrizione}
Il CDMA (Code Division Multiple Access) è una tecnologia di comunicazione wireless che consente a più utenti di condividere la stessa banda di frequenza simultaneamente\\
Ogni utente è identificato da un codice univoco (codice di spreading), che permette di distinguere i segnali sovrapposti sfruttando la teoria della codifica\\
CDMA lavora sullo spazio multidimensionale in cui i codici generano degli "assi" per garantire la separazione tra utenti
\subsection{Ambiti d'uso}
È stata utilizzata nelle reti cellulari di seconda generazione (2G) e terza generazione (3G) dove è alla base dello standard W-CDMA, usato per comunicazioni mobili con velocità superiori
\subsection{Pregi}
Efficienza nell'uso della banda (più utenti condividono lo stesso spettro), gestione intelligente del traffico, robustezza contro le interferenze e utilizzo flessibile dello spettro
\subsection{Difetti}
Presenta difetti legati alla gestione della potenza (le variazioni della distanza tra l'utente e la stazione base comportano che gli utenti più lontani debbano aumentare la loro potenza di trasmissione, causando così una maggiore interferenza e aumentando il consumo energetico), all'interferenza tra utenti (se i codici non sono abbastanza distinti o se vi sono errori nei codici, può verificarsi un'interferenza tra i segnali), alla complessità dell'hardware (i dispositivi e le stazioni base sono complesse e costose da progettare)
