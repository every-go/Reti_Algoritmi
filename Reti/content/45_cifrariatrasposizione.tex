\section{Cifrari a trasposizione (2014, 15, 20)}
\subsection{Descrizione}
I cifrari a trasposizione sono tecniche di cifratura che consistono nel riordinare le lettere del testo in chiaro seguendo uno schema predeterminato, senza modificarne il valore\\
A differenza dei cifrari a sostituzione, che sostituiscono i caratteri con altri, la trasposizione si limita a permutare l'ordine delle lettere, generando un testo cifrato che appare casuale\\
Questi metodi possono essere applicati in vari modi, come attraverso griglie, schemi a zig-zag o scacchiere
\subsection{Ambiti d'uso}
\begin{enumerate}
\item Storia: Utilizzati in contesti militari antichi, come la Scitala spartana, per la trasmissione sicura di messaggi
\item Didattica: Usati per insegnare i principi fondamentali della crittografia e introdurre le differenze tra trasposizione e sostituzione
\item Giochi e enigmistica: Applicati in puzzle crittografici o escape room per simulare la decifrazione di messaggi cifrati
\end{enumerate}
\subsection{Pregi}
\begin{enumerate}
\item Semplicità di implementazione: i cifrari a trasposizione sono facili da applicare sia manualmente che con strumenti minimi, come carta e penna
\item Maggiore sicurezza rispetto alla sostituzione semplice: non conservano direttamente la frequenza delle lettere del testo in chiaro, rendendo più complessa l'analisi di frequenza
\item Combinabilità: possono essere combinati con altri cifrari per aumentare la complessità della cifratura
\item Nessuna propagazione degli errori: un errore nel testo cifrato influisce solo sulla posizione specifica e non compromette il resto del messaggio
\end{enumerate}
\subsection{Difetti}
\begin{enumerate}
\item Pattern evidenti: la trasposizione, se applicata a testi lunghi o con schemi ripetuti, può lasciare indizi utili per la decifrazione
\item Vulnerabilità ad attacchi combinatori: con strumenti moderni, uno schema di trasposizione può essere decifrato analizzando tutte le possibili permutazioni
\item Necessità di una chiave condivisa: la decifratura richiede che il destinatario conosca lo schema esatto usato per la trasposizione
\item Limitata sicurezza moderna: contro i metodi crittografici avanzati, i cifrari a trasposizione non offrono un livello di protezione adeguato
\end{enumerate}
