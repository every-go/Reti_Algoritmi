\section{UDP (2014, 16, 17, 18, 20, 22, 23, 24, 25)}
\subsection{Descrizione}
Il User Datagram Protocol (UDP) è un protocollo di trasporto\\
UDP è progettato per la trasmissione rapida di dati con un overhead minimo, adottando un approccio best-effort\\
A differenza del TCP, non fornisce meccanismi di controllo della connessione, ritrasmissione o verifica dell'ordine dei pacchetti\\
Le principali caratteristiche di UDP includono:
\begin{enumerate}
\item Nessuna connessione:\\
UDP è un protocollo connectionless, il che significa che non stabilisce una connessione prima dell'invio dei dati
\item Struttura semplice:\\
il segmento UDP contiene solo informazioni essenziali: porta di origine, porta di destinazione, lunghezza del segmento e checksum
\item Velocità e leggerezza:\\
poiché non implementa controlli di errore o ritrasmissione, UDP è parecchio veloce
\item Multiplexing:\\
UDP consente la comunicazione tra più applicazioni attraverso l'uso delle porte
\item Affidabilità delegata:\\
qualsiasi controllo di errore o meccanismo di affidabilità deve essere implementato dall'applicazione che utilizza UDP
\end{enumerate}
\subsection{Ambiti d'uso}
UDP viene utilizzato in diversi scenari, principalmente in applicazioni che richiedono bassa latenza, velocità o trasmissione continua di dati:
\begin{enumerate}
\item Streaming audio e video:\\
per garantire una riproduzione fluida, è preferibile perdere qualche pacchetto piuttosto che ritardare la trasmissione
\item Trasmissioni multicast e broadcast:\\
UDP è utilizzato per inviare dati a più destinatari contemporaneamente
\end{enumerate}
\subsection{Pregi}
UDP presenta numerosi vantaggi per applicazioni specifiche:
\begin{enumerate}
\item Bassa latenza:\\
l'assenza di meccanismi di controllo della connessione rende UDP ideale per applicazioni in tempo reale
\item Efficienza:\\
il ridotto overhead del protocollo garantisce una trasmissione dei dati più veloce rispetto a TCP
\item Semplicità:\\
grazie alla sua struttura minimale, UDP è facile da implementare e consuma meno risorse di sistema
\item Supporto multicast e broadcast:\\
UDP consente trasmissioni simultanee a più destinatari, a differenza di TCP
\item Adatto per applicazioni specifiche:\\
è perfetto per scenari in cui la perdita di pacchetti non compromette l'esperienza complessiva, come lo streaming
\end{enumerate}
\subsection{Difetti}
Nonostante i suoi vantaggi, UDP ha delle limitazioni:
\begin{enumerate}
\item Mancanza di affidabilità:\\
UDP non garantisce la consegna dei pacchetti, che possono essere persi o arrivare fuori ordine
\item Assenza di controllo di errore:\\
il checksum di UDP rileva solo errori nei dati, ma non offre alcuna correzione o ritrasmissione automatica
\item Non adatto per dati critici:\\
applicazioni che richiedono integrità e affidabilità dei dati (come il trasferimento di file) devono utilizzare protocolli come TCP
\end{enumerate}
