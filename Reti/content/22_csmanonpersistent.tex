\section{CSMA non persistent (2022)}
\subsection{Descrizione}
Il CSMA non persistente è un tipo di CSMA (protocollo multiaccesso) in cui, quando un dispositivo rileva che il canale è occupato, non continua a monitorarlo costantemente\\
Invece, attende un intervallo di tempo casuale (random backoff) prima di verificare nuovamente se il canale è libero e riprovare la trasmissione\\
Questo approccio aiuta a ridurre la probabilità di collisioni rispetto al CSMA 1-persistente (dove i dispositivi trasmettono immediatamente quando rilevano il canale libero)\\
Esso può raggiungere performance fino al 90$\%$
\subsection{Ambiti d'uso}
Il CSMA non persistente è adatto a scenari in cui la riduzione delle collisioni è cruciale e il traffico della rete non è eccessivamente intenso\\
Gli ambiti principali includono:\\
Reti locali cablate (LAN):\\
può essere utilizzato in reti Ethernet tradizionali con basso o moderato traffico\\
Reti di sensori e IoT (Internet of Things):\\
sistemi con dispositivi che trasmettono dati sporadicamente, dove si vuole ridurre il rischio di collisioni\\
Comunicazioni satellitari o radio:\\
in reti dove i tempi di propagazione sono elevati e si richiede un approccio più conservativo per ridurre le collisioni\\
Sistemi di controllo industriale:\\
utilizzato in ambienti con un numero limitato di dispositivi che devono trasmettere senza sovraccaricare il canale
\subsection{Pregi}
Il CSMA non persistente offre diversi vantaggi:\\
Riduzione delle collisioni:\\
l’attesa di un tempo casuale prima di riprovare la trasmissione riduce la probabilità che più dispositivi trasmettano contemporaneamente quando il canale diventa libero\\
Maggiore efficienza in condizioni di traffico moderato:\\
la probabilità di saturare il canale diminuisce rispetto ad altri approcci, come il CSMA 1-persistente\\
Semplicità di implementazione:\\
non richiede un coordinamento centrale o sincronizzazione complessa, rendendolo adatto a reti distribuite\\
Adattabilità dinamica:\\
funziona bene in reti con traffico non prevedibile o carico variabile
\subsection{Difetti}
Il CSMA non persistente presenta anche alcune limitazioni:\\
Maggiore latenza:\\
l’attesa casuale introduce ritardi anche quando il canale è libero, causando una latenza maggiore rispetto ad altri protocolli (ad esempio, CSMA 1-persistente)\\
Prestazioni degradate in reti congestionate:\\
in condizioni di alto traffico, il numero di tentativi falliti e le attese casuali possono aumentare il ritardo medio e diminuire l'efficienza\\
Minore utilizzo del canale in reti leggere:\\
Rispetto al CSMA 1-persistente, in situazioni con pochi dispositivi attivi, il protocollo non persistente può essere meno efficiente perché introduce attese inutili\\
Difficoltà nel garantire priorità:\\
Non offre meccanismi per assegnare priorità alla trasmissione di dati urgenti, rendendolo meno adatto a reti con traffico prioritario
