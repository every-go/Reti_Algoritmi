\section{ICMP (2014, 15, 22)}
\subsection{Descrizione}
Internet Control Message Protocol (ICMP) è un protocollo di rete del livello 3 (network layer) progettato per trasmettere messaggi di controllo e segnalazione tra dispositivi in una rete IP\\
ICMP non è utilizzato per il trasferimento di dati applicativi, ma per comunicare errori, condizioni di rete o informazioni diagnostiche\\
Alcune delle sue funzioni principali includono
\begin{itemize}
\item Segnalare errori, come l'inaccessibilità di una destinazione o il superamento del TTL (Time to Live)
\item Gestione del controllo del traffico, ad esempio avvisando un mittente di ridurre la velocità di trasmissione
\end{itemize}
I messaggi ICMP vengono trasportati all'interno di datagrammi IP, ma non garantiscono la consegna: ICMP si basa sul protocollo IP, che è di tipo best-effort (il sistema fa del suo meglio per consegnare i dati, ma non garantisce alcuna affidabilità, ordine o tempi di consegna precisi)
\subsection{Ambiti d'uso}
ICMP è utilizzato in vari contesti per il controllo e la gestione delle reti:
\begin{enumerate}
\item Diagnostica della rete
\item Gestione degli errori:
\begin{enumerate}
\item Notifica di destinazione irraggiungibile (Destination Unreachable) quando un pacchetto non può essere consegnato
\item Avviso di superamento del TTL (Time Exceeded) quando un pacchetto supera il numero massimo di salti consentiti
\end{enumerate}
\item Ottimizzazione della rete:\\
ICMP può suggerire al mittente di ridurre la velocità di trasmissione o instradare i pacchetti su percorsi alternativi
\item Supporto al routing dinamico:\\
ICMP viene utilizzato dai router per segnalare problemi o aggiornamenti di stato ai sistemi di gestione della rete
\end{enumerate}
\subsection{Pregi}
ICMP offre numerosi vantaggi per il monitoraggio e la gestione delle reti:
\begin{enumerate}
\item Leggerezza:\\
ICMP è un protocollo leggero con overhead minimo, rendendolo ideale per notifiche e diagnosi rapide
\item Strumento diagnostico essenziale:\\
fornisce informazioni cruciali sulla connettività e le prestazioni di rete, indispensabili per i tecnici e gli amministratori di rete
\item Universalità:\\
essendo parte integrante del protocollo IP, ICMP è supportato da quasi tutti i dispositivi di rete
\end{enumerate}
\subsection{Difetti}
Nonostante la sua utilità, ICMP presenta alcune limitazioni e vulnerabilità:
\begin{enumerate}
\item Nessuna garanzia di consegna: ICMP si basa sul protocollo IP, che è un protocollo best-effort, quindi i messaggi ICMP possono andare persi
\item Vulnerabilità alla sicurezza
\item Blocco in alcune reti:\\
per motivi di sicurezza, molti firewall e dispositivi di rete bloccano i messaggi ICMP, riducendo la loro utilità per diagnosi o gestione della rete
\end{enumerate}
