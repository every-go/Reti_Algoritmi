\section{Sostituzione monoalfabetica (2019, 24)}
\subsection{Descrizione}
Tipo di crittografia molto debole nell'epoca moderna\\
Consiste nel sostituire ogni lettera con un'altra cosicché il testo sia pressoché indecifrabile (se letto senza opportuni algoritmi di cifratura)\\
Viene considerato un algoritmo che è parte del principio di Kerchoff, ovvero che "Il design di un sistema non dovrebbe richiedere segretezza e compromettere il sistema non dovrebbe dare problemi ai corrispondenti"\\
Infatti il suo principio ha dato vita all'idea che l'algoritmo di encrypting deve essere pubblico (tutti sanno che ogni lettera è sostituita con un'altra) ma le chiavi sono segrete (nessuno sa quale lettere sono state sostituite)
\subsection{Ambiti d'uso}
Il suo uso è nato nel passato, grazie all'uso di questa tecnica in documenti militari e diplomatici\\
L'esempio più famoso è il cifrario di Cesare, il quale consiste nel scambiare ogni lettera con quella 3 lettere davanti nell'alfabeto (esempio la A con la D, la B con la E e così via)\\
Attualmente viene usato anche come principio d'insegnamento della crittografia e per sfide educative ma non trova quasi mai contesti moderni nell'ambito di internet in cui esso sia realmente utilizzato e sicuro, infatti ormai viene usato più per comunicazioni informali o non critiche
\subsection{Pregi}
Il suo grosso pregio è che ci possono essere tantissime chiavi, ovvero 26!-1 chiavi diverse e, con la brute-force, per tentare tutte le combinazioni ci vorrebbe troppo tempo\\
Inoltre ha dalla sua la facilità d'implementazione e decifratura, comodo se si vuole dare un minimo di sicurezza, ha una bassa dipendenza da risorse tecnologiche ed è ottimo contro i "non esperti"\\
Inoltre un errore nella cifratura o decifratura di un singolo carattere non compromette gli altri caratteri del messaggio, infatti ogni lettera è cifrata in modo indipendente, a differenza di cifrari più avanzati che collegano i blocchi.
\subsection{Difetti}
Ampiamente superata dagli algoritmi moderni, come il DES o anche il triple DES, è decifrabile da qualunque attaccante che abbia una quasi minima conoscenza di crittografia\\
Infatti, quando si hanno ulteriori informazioni sul sistema, come l'alfabeto utilizzato, la frequenza delle lettere più presenti in quell'alfabeto, i bigrammi e i trigrammi più frequenti, è facile risalire con un'analisi semi-statistica ogni lettere quale lettera rappresenta realmente, rendendo poi più facile la decifratura nei successivi passaggi\\
Tutto questo infatti si chiama frequency analysis
