\section{ECB (2014)}
\subsection{Descrizione}
ECB (Electronic Code Book) è un modo di trasmettere testi cifrati\\
Infatti, il testo viene suddiviso in blocchi, e ogni blocco viene considerato come se fosse la pagina di un libro\\
Per risolvere eventuali mescolamenti delle pagine da parte di attaccanti, si creano dipendenze tra le pagine del libro in modo che non sia possibile scambiarle di posto
\subsection{Ambiti d'uso}
\begin{enumerate}
\item Cifratura di dati in batch: adatto per situazioni in cui i dati devono essere cifrati separatamente e senza dipendenza tra i blocchi
\item Applicazioni con requisiti di alta velocità: l'assenza di dipendenza tra i blocchi permette un'elaborazione rapida, ideale per applicazioni che richiedono performance elevate
\item Sistemi legacy: utilizzato in sistemi più vecchi o in ambienti con limitate risorse di elaborazione
\end{enumerate}
\subsection{Pregi}
I principali vantaggi dell'ECB includono:
\begin{enumerate}
\item Semplicità: il funzionamento di ECB è facile da comprendere e implementare
\item Parallellizzazione: poiché ogni blocco è cifrato separatamente, l'algoritmo può essere facilmente parallellizzato per migliorare le performance
\item Velocità: essendo una modalità senza dipendenza tra i blocchi, è generalmente veloce e consuma meno risorse computazionali rispetto ad altri metodi
\end{enumerate}
\subsection{Difetti}
Nonostante i vantaggi, ECB presenta anche alcuni svantaggi significativi:
\begin{enumerate}
\item Vulnerabilità agli attacchi di pattern: poiché lo stesso blocco di testo in chiaro viene sempre cifrato nello stesso modo, i pattern nei dati possono essere facilmente identificati, rendendo ECB meno sicuro per dati sensibili, come quelli contenenti informazioni ripetitive
\item Mancanza di diffusione: non essendo basato su meccanismi di mescolamento tra blocchi, eventuali debolezze in un singolo blocco possono propagarsi attraverso i dati cifrati
\end{enumerate}
