\section{TCP, Trasmission Control Protocol (2016)}
\subsection{Descrizione}
TCP è uno dei principali protocolli di rete su Internet\\
È un protocollo di comunicazione orientato alla connessione, che si basa sul 3-way handshake\\
Full-duplex e point-to-point, è un protocollo reliable (affidabile) e il checksum è solo frutto di semplici somme e non codici di error detection complessi\\
Oltretutto ha integrati il controllo del flusso, il controllo della congestione e la ritrasmissione automatica
\subsection{Ambiti d'uso}
TCP viene utilizzato in tutti quei contesti dove è essenziale avere una trasmissione dati affidabile, anche a costo di una maggiore latenza\\
Alcuni esempi includono:
\begin{enumerate}
\item Navigazione web: protocollo HTTP e HTTPS
\item Email: protocolli come SMTP, IMAP e POP3
\item Trasferimenti di file: FTP (File Transfer Protocol)
\item Applicazioni di gestione remota: SSH e Telnet
\item Sistemi di database: comunicazioni tra client e server di database
\end{enumerate}
\subsection{Pregi}
I suoi pregi sono anche le sue principali caratteristiche:
\begin{enumerate}
\item Affidabile
\item Permette di controllare la congestione
\item Supporto universale: ampiamente supportato su tutti i sistemi operativi e dispositivi connessi a Internet
\item Versatilità: adatto a una vasta gamma di applicazioni, dalle comunicazioni interattive (come SSH) ai trasferimenti di file
\end{enumerate}
\subsection{Difetti}
\begin{enumerate}
\item Maggiore latenza: l'affidabilità e il controllo di flusso aumentano la latenza rispetto ai protocolli più semplici, come UDP
\item Overhead elevato: le informazioni aggiuntive necessarie per la gestione di sequenze, ritrasmissioni e controllo della congestione aumentano l'overhead
\item Non adatto a trasmissioni in tempo reale: per applicazioni come lo streaming video o i giochi online, l'elevata latenza e la gestione dell'ordine dei pacchetti possono essere controproducenti
\item Gestione della connessione: essendo orientato alla connessione, richiede una fase di handshake iniziale, che può essere onerosa in applicazioni con molte connessioni brevi
\item Difficoltà in reti instabili: in reti molto instabili, il numero di ritrasmissioni può crescere significativamente, rallentando ulteriormente la comunicazione
\end{enumerate}
