\section{CSMA (2017, 18, 20)}
\subsection{Descrizione}
CSMA (Carrier Sense Multiple Access) è un protocollo multiaccesso più complesso ispirato da Aloha\\
Prima di trasmettere si controlla se non ci sia già una trasmissione e, se essa è presente, si controlla che il canale sia libero\\
Se così fosse, viene effettuato un controllo l'istante successivo finché non si ha la possibilità di trasmettere\\
Ci sono diverse varianti di CSMA, che migliorano il protocollo in base al modo in cui le collisioni vengono gestite:\\
CSMA/CD (Collision Detection):\\
utilizzato nelle reti Ethernet cablate, rileva le collisioni durante la trasmissione e interrompe immediatamente la trasmissione\\
CSMA/CA (Collision Avoidance):\\
utilizzato nelle reti wireless (come Wi-Fi), cerca di evitare le collisioni implementando un sistema di attesa (backoff) e segnalazioni di conferma (acknowledgments)\\
CSMA senza controllo delle collisioni\\
il dispositivo trasmette solo se il canale è libero, ma non reagisce alle collisioni
\subsection{Ambiti d'uso}
Ethernet (CSMA/CD):\\
reti Ethernet cablate delle prime generazioni (ad esempio reti 10BASE5 e 10BASE2), dove più dispositivi condividevano lo stesso mezzo trasmissivo\\
Wi-Fi (CSMA/CA):\\
reti wireless, come IEEE 802.11, dove le collisioni sono difficili da rilevare ma possono essere mitigate con strategie di evitamento\\
Sistemi wireless a bassa potenza:\\
ad esempio in reti di sensori o reti IoT (Internet of Things), dove i dispositivi comunicano su un canale condiviso\\
Reti satellitari o radio:\\
utilizzato nei sistemi dove il mezzo trasmissivo è condiviso da molteplici dispositivi e le risorse radio sono limitate
\subsection{Pregi}
Semplicità:\\
CSMA è relativamente semplice da implementare e si adatta bene a reti a basso traffico o con poche stazioni attive\\
Efficienza in reti leggere:\\
con pochi dispositivi attivi, la probabilità di collisione è bassa, e il protocollo garantisce una trasmissione efficace\\
Flessibilità:\\
è adatto sia a reti cablate (CSMA/CD) sia a reti wireless (CSMA/CA), con modifiche per adattarsi alle caratteristiche specifiche del mezzo trasmissivo\\
Uso decentralizzato:\\
Non richiede un coordinatore centrale, rendendolo adatto a reti distribuite\\
Adattabilità dinamica:\\
il meccanismo di rilevazione o evitamento di collisioni consente al protocollo di reagire in tempo reale ai cambiamenti nel traffico della rete
\subsection{Difetti}
Problemi con traffico elevato:\\
all'aumentare del numero di dispositivi, aumenta la probabilità di collisioni, riducendo drasticamente l'efficienza\\
Collisioni inevitabili:\\
anche con il carrier sensing, le collisioni possono avvenire, specialmente nei sistemi con alti tempi di propagazione del segnale (ad esempio reti wireless)\\
Degrado delle prestazioni:\\
Quando la rete è congestionata, CSMA può portare a ritrasmissioni frequenti, riducendo l'efficienza complessiva e aumentando i ritardi
