\section{IPv4 (2024)}
\subsection{Descrizione}
IPv4 (Internet Protocol versione 4) è il protocollo che definisce il formato del datagramma IP e il meccanismo per il trasferimento di dati tra dispositivi su reti basate su IP\\
L'intestazione (header) di un datagramma IPv4 è suddivisa in due parti principali:
\begin{itemize}
\item Parte fissa: È lunga 20 byte e include campi essenziali
\item Parte variabile: può contenere aggiunte opzionali, come timestamp, sicurezza, o registrazione del percorso, pensate per eventuali funzionalità future
\end{itemize}
IPv4 è stato progettato con un'architettura robusta per gestire le comunicazioni su reti eterogenee, garantendo adattabilità
\subsection{Ambiti d'uso}
IPv4 è il protocollo fondamentale per l'invio di pacchetti nello strato di rete (network layer) e rappresenta la base delle comunicazioni su Internet\\
Alcuni ambiti di utilizzo:
\begin{itemize}
\item Trasporto dati su Internet:\\
è il protocollo principale per il trasferimento di pacchetti tra dispositivi collegati alla rete
\item Reti locali (LAN) e geografiche (WAN):\\
IPv4 è utilizzato in contesti di rete di diversa scala, dal piccolo ufficio a reti globali
\item Interconnessione di reti eterogenee:\\
permette la comunicazione tra dispositivi con architetture diverse, garantendo universalità e flessibilità
\item Gestione del traffico dinamico:\\
IPv4 utilizza tecniche come il routing dinamico e il flooding per garantire la consegna anche in condizioni di rete non ottimali (es. congestione o attacchi)
\end{itemize}
IPv4 non implementa meccanismi di controllo degli errori per garantire alte prestazioni; questo compito è demandato ai livelli superiori
\subsection{Pregi}
\begin{itemize}
\item Struttura gerarchica degli indirizzi:\\
gli indirizzi IPv4 vengono assegnati attraverso un sistema gerarchico gestito da autorità centralizzate garantendo efficienza nella distribuzione e nel routing
\item Compatibilità universale:\\
IPv4 è supportato da praticamente tutti i dispositivi connessi alla rete, rendendolo il protocollo più utilizzato per le comunicazioni IP
\item Semplicità e robustezza:\\
è progettato per essere operativo in contesti di rete dinamici, garantendo resilienza contro interruzioni o perdite di pacchetti
\item Cachability e instradamento ottimizzato:\\
grazie alla struttura gerarchica, il routing può essere ottimizzato attraverso subnetting e aggregazione degli indirizzi
\end{itemize}
\subsection{Difetti}
Nonostante i suoi pregi, IPv4 presenta alcuni limiti intrinseci:
\begin{itemize}
\item Limitazione degli indirizzi disponibili:\\
gli indirizzi IPv4 sono a 32 bit, il che consente circa 4,3 miliardi di indirizzi univoci\\
Tuttavia, lo spazio degli indirizzi è ormai quasi esaurito a causa della crescita esponenziale dei dispositivi connessi
\item Frammentazione dei pacchetti:\\
in reti con diverse dimensioni di MTU (Maximum Transmission Unit), i pacchetti devono essere frammentati, aumentando l'overhead e la complessità
\item Vulnerabilità alla sicurezza:\\
IPv4 non include nativamente meccanismi di sicurezza, come l'autenticazione o la crittografia, che devono essere implementati tramite protocolli aggiuntivi
\item Limite del TTL:\\
il campo TTL consente al datagramma un massimo di 255 salti\\
Anche se i router possono rigenerare i datagrammi, questo valore rappresenta una limitazione nei casi estremi
\end{itemize}
