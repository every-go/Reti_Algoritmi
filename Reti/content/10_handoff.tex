\section{Handoff (2023)}
\subsection{Handoff 1G}
Nella prima generazione di trasmissione dati attraverso la rete di telefonia mobile analogica, uno degli standard principali era l'AMPS (Advanced Mobile Phone System) per gli USA\\
L'handoff era una tecnica che occorreva quando il segnale era debole per ricollegarsi ad un segnale migliore\\
In questa situazione lo switching office (la stazione base di ogni cella) verifica attraverso le celle lo stato del segnale ricevuto dal dispositivo ed esso viene assegnato alla cella con potenza più alta\\
Si presenta in due tipologie:
\begin{enumerate}
\item \textbf{hard handoff}: la vecchia stazione rilascia il cellulare prima che la nuova lo riagganci causando un ritardo di circa 0,3 secondi
\item \textbf{soft handoff}: la nuova cella acquisisce il cellulare prima che la vecchia cella lo lasci, eliminando le interruzioni ma il cellulare deve collegarsi a due frequenze contemporaneamente aumentando costi e consumo energetico
\end{enumerate}
\subsection{Handoff 2G}
Mentre nell'handoff 1G se ne occupa il control switch, nel nuovo standard D-AMPS (evoluzione dell'AMPS e retrocompatibile) inizia l'idea delle "tacchette" presente nei telefoni attuali\\
Con questo sistema si rappresenta la potenza del segnale permettendo di monitorare costantemente la qualità della connessione\\
Questo approccio è noto come MAHO (Mobile Assisted Hand Off) permettendo un carico aggiuntivo minimo poiché le misurazioni vengono effettuate durante i tempi morti del TDM\\
