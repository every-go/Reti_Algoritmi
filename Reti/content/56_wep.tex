\section{WEP}
\subsection{Descrizione}
WEP (Wired Equivalent Privacy) è un protocollo di sicurezza per le reti wireless, progettato per fornire un livello di protezione simile a quello delle reti cablate\\
Utilizza il sistema di cifratura RC4 e una chiave di 64 o 128 bit per criptare i dati trasmessi tra il dispositivo wireless e il punto di accesso
\subsection{Ambiti d'uso}
\begin{enumerate}
\item Reti wireless domestiche: WEP è stato uno dei primi protocolli di sicurezza per reti Wi-Fi e veniva utilizzato principalmente nelle reti domestiche
\item Reti aziendali: In passato, molte aziende implementavano WEP per proteggere le loro reti wireless interne, sebbene sia stato successivamente sostituito da protocolli più sicuri come WPA e WPA2
\end{enumerate}
\subsection{Pregi}
\begin{enumerate}
\item Implementazione semplice: WEP è facile da configurare e implementare, rendendolo una scelta popolare nelle prime fasi della sicurezza delle reti wireless
\item Supporto diffuso: essendo uno dei primi protocolli di sicurezza per le reti wireless, WEP è stato ampiamente supportato da vari dispositivi
\end{enumerate}
\subsection{Difetti}
\begin{enumerate}
\item Vulnerabilità nella cifratura: Il metodo di cifratura RC4 e la gestione delle chiavi sono vulnerabili ad attacchi di forza bruta e attacchi di tipo "cracking", rendendo WEP obsoleto
\item Debolezza nella gestione delle chiavi: WEP utilizza una chiave condivisa statica, il che rende facile per un attaccante decifrare i dati dopo l'acquisizione di una quantità sufficiente di pacchetti cifrati
\end{enumerate}
