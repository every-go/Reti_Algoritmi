\section{Modulazione delta (2016, 17, 18, 19, 20, 22, 23)}
\subsection{Descrizione}
La modulazione delta è una tecnica di codifica a basso consumo utilizzata per comprimere segnali analogici in modo semplice ed efficiente\\
Il principio base è quello di campionare il segnale a intervalli regolari e confrontare ogni campione con il valore precedente\\
Se il segnale cresce, si registra un 1, mentre se diminuisce, uno 0\\
Questa rappresentazione descrive l'andamento del segnale, ma non ne conserva la forma precisa
\subsection{Ambiti d'uso}
Usato per le compressioni delle trasmissioni in digitale (come ad esempio per il 2G), nelle trasmissioni audio in dispositivi che non richiedono una qualità elevata, sistemi di acquisizione dati, nei dispositivi a bassa potenza, comunicazioni wireless a bassa velocità
\subsection{Pregi}
Efficienza di compressione, basso consumo energetico e velocità di elaborazione
\subsection{Difetti}
Si ha una perdita di qualità in quanto non conserva informazioni dettagliate sulla forma dell'onda originale\\
Inoltre si ha dipendenza dalla velocità di campionamento, se troppo lento può introdurre errori significativi nella ricostruzione del segnale
