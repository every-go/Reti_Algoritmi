\section{802.3}
\subsection{Descrizione}
802.3 è uno dei protocolli più famosi della storia ed è uno dei protocolli di Ethernet, il quale si prevede come minimo sarà in vigore fino al 2080/2100\\
Questo protocollo in particolare ha un tipo di cablaggio a "serpente", a "lisca di pesce" oppure ad "albero"\\
Per indicare i vari tipi di cavi si utilizza una notazione particolare, del tipo XbaseY, dove X indica la banda in Mbps, "Base" indica che è una connessione baseband (a frequenza unica) e Y è il tipo di cavo che differisce a seconda della lunghezza massima di ogni tratto senza ripetitori
\subsection{Ambiti d'uso}
Utilizzato principalmente per le reti locali LAN in cui addirittura è l'ambito principale dello standard, per il data center e cloud computing (è necessario un Ethernet ad alte prestazioni), in ambienti industriali, per le reti di provider di servizi e per le reti domestiche
\subsection{Pregi}
L'Ethernet ha un'elevatissima diffusione e interoperabilità, un'altissima affidabiltà grazie all'uso di tecnologie come il controllo delle collisioni e miglioramenti successivi con reti full-duplex, la velocità scalabile (dai 10Mbps ai 400Gbps), ai suoi costi contenuti grazie all'ampia adozione e alla facilità di installazione e manutenzione
\subsection{Difetti}
Alcuni difetti possono essere le limitazioni geografiche, la dipendenza dal cablaggio fisico, la scalabilità rispetto al wireless, il costo delle soluzioni ad alte prestazioni, la mancanza di mobilità e i consumi energetici
