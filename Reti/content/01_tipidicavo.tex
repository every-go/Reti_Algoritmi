\section{Tipi di cavo (2014)}
Ci sono diversi tipi di cavo, fra i più utilizzati si trovano:\\
\begin{enumerate}
\item Unshielded Twisted Pair (UTP): coppia di fili annodati tra loro con il twist che serve a limitare l'interferenza reciproca che altrimenti sarebbe troppo elevata\\
La sua applicazione più comune è il sistema telefonico, possono estendersi per diversi chilometri ma per distanze più lunghe sono necessari dei ripetitori\\
Hanno un basso costo e un discreto livello di prestazioni, attualmente sono largamente utilizzati
\item Cavo coassiale: hanno una schermatura migliore dei cavi UTP e per questo sono molto usati per TV via cavo e le MAN (metropolitan area network)\\
La loro larghezza di banda è all'incirca 1GHz\\
Può estendersi per distanze più lunghe e consente velocità più elevate\\
Fornisce ampiezza di banda ed eccellente immunità al rumore
\item Fibra ottica: non si ha più elettricità ma si trasporta la luce, grazie a un pezzetto di vetro interno che non deve assolutamente rompersi, però non subisce interferenze elettriche\\
L'estensione di un cavo in fibra può avvenire attraverso: connettori che perdono 10-20$\%$ di luce, allineatori meccanici coi quali si perde il 10$\%$ di luce, oppure per fusione con la quale si perde il 2$\%$ di luce\\
L'ampiezza di banda raggiungibile va sicuramente oltre i 50Tbps ma si è chiusi dal limite pratico di 10Gbps\\
C'è anche un tipo di fibra detto monomodale in cui la luce si propaga solo in linea retta, è più costosa ed utilizzata soprattutto sulle lunghe distanze
\end{enumerate}
