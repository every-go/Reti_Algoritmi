\section{CIDR (2014, 15, 16, 17, 18, 19, 22)}
\subsection{Descrizione}
Il CIDR (Classless Inter-Domain Routing) è una tecnica di indirizzamento IP introdotta per superare le limitazioni delle vecchie classi di rete (Class A, B, C)\\
Permette di utilizzare blocchi di indirizzi di lunghezza variabile, invece dei blocchi fissi imposti dalle classi tradizionali (classless)\\
Con il CIDR, gli indirizzi IP sono rappresentati in forma di prefisso, ad esempio `192.168.0.0/24`, dove il numero dopo la barra indica la lunghezza del prefisso di rete.\\
Questo permette una suddivisione più granulare o un'aggregazione, ottimizzando l'uso degli indirizzi IP.
\subsection{Ambiti d'uso}
Principalmente utilizzato per l'ottimizzazione delle tabelle di routing (reti con prefissi comuni possono essere aggregate) e per la gestione degli indirizzi IP (evita sprechi dovuti ai blocchi fissi delle classi tradizionali)\\
Quando più classi di indirizzi devono essere indirizzate allo stesso router, possono essere combinate in un'unica voce di routing se condividono un prefisso comune\\
Tuttavia, in caso di sovrapposizione, l'entrata con il prefisso di rete più lungo (più specifica) ha la priorità, secondo la regola del Longest Prefix Match
\subsection{Pregi}
\begin{enumerate}
\item Migliore efficienza nell'uso degli indirizzi IP: Permette di ridurre il problema dello spreco di indirizzi grazie alla flessibilità della lunghezza del prefisso
\item Riduzione delle dimensioni delle tabelle di routing: \\
l'aggregazione di reti consente di semplificare e ottimizzare le tabelle di routing
\item Facilità di gestione del subnetting:\\
consente di creare sottoreti adattabili alle esigenze di organizzazioni di diverse dimensioni.
\end{enumerate}
\subsection{Difetti}
\begin{enumerate}
\item Complessità di configurazione:\\
la flessibilità del CIDR richiede una maggiore attenzione nella pianificazione e configurazione delle reti
\item Necessità di hardware aggiornato:\\
i router più vecchi potrebbero non supportare pienamente il CIDR, richiedendo aggiornamenti o sostituzioni
\item Maggiore difficoltà nella risoluzione dei problemi:\\
la granularità dei prefissi può complicare l'analisi e la diagnosi di problemi di rete
\item Dipendenza da un'adeguata progettazione del routing:\\
una configurazione errata può portare a inefficienze o a conflitti nel routing.
\end{enumerate}
