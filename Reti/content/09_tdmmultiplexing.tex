\section{TDM Multiplexing (2016)}
\subsection{Descrizione}
Il Time Division Multiplexing (TDM) è una tecnica di multiplazione che consente la condivisione di un unico canale di trasmissione tra più segnali\\
Invece di separare i segnali attraverso frequenze diverse (come avviene nel Frequency Division Multiplexing, FDM), il TDM utilizza intervalli temporali distinti per trasmettere i segnali
\subsection{Ambiti d'uso}
Ampiamente utilizzato nelle telecomunicazioni (trasmissione di segnali vocali e dati su reti digitali), reti di trasmissione dati (Ethernet e WAN), sistemi satellitari, sistemi di trasmissione televisiva (segnali multipli su un singolo canale)
\subsection{Pregi}
Permette di avere un'elevatissima flessibilità, un'ottima efficienza spettrale in quanto si usa un'unica banda e una buona compatibilità (dalle linee telefoniche ai sistemi satellitari)
\subsection{Difetti}
Più i canali aumentano più ognuno avrà a disposizione meno capacità e quindi sarà più lento\\
Sincronizzazione complessa precisa tra trasmettitore e ricevitore\\
Sensibilità al ritardo, il quale se accumulato influenza negativamente sulla qualità dei servizi in tempo reale come la voce o il video
