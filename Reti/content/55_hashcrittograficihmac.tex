\section{Hash crittografici, HMAC (2016, 18, 19)}
\subsection{Descrizione}
Un hash crittografico è una funzione che prende un input e restituisce una stringa di lunghezza fissa\\
Le funzioni hash sono progettate per essere unidirezionali, cioè facili da calcolare ma difficili da invertire\\
HMAC (Hashed Message Authentication Code) è una costruzione basata su una funzione hash che fornisce un meccanismo di autenticazione per verificare l'integrità e l'autenticità dei messaggi
\subsection{Ambiti d'uso}
\begin{enumerate}
\item Autenticazione nei protocolli di rete: utilizzato in protocollo come IPsec, TLS e HTTPS per garantire che i dati non siano stati alterati durante la trasmissione
\item Verifica dell'integrità dei dati: impiegato per assicurare che i dati non siano stati modificati durante la trasmissione o l'archiviazione
\item Generazione di chiavi segrete: utilizzato per la generazione di chiavi segrete condivise tra due parti
\end{enumerate}
\subsection{Pregi}
\begin{enumerate}
\item Sicurezza: HMAC fornisce una forte garanzia di integrità e autenticità, proteggendo i dati da modifiche non autorizzate
\item Flessibilità: È compatibile con qualsiasi funzione hash, come SHA-1, SHA-256, ecc
\item Efficienza: HMAC è relativamente efficiente dal punto di vista computazionale, rendendolo adatto anche a dispositivi con risorse limitate
\end{enumerate}
\subsection{Difetti}
Lentezza con funzioni hash deboli: se la funzione hash utilizzata non è ottimale le prestazioni e la sicurezza possono essere compromesse
