\section{IPSec (2019, 20, 22)}
\subsection{Descrizione}
IPsec è un protocollo di rete che fornisce crittografia e autenticazione per le comunicazioni IP\\
Viene utilizzato per garantire la sicurezza delle comunicazioni tramite reti non sicure, come Internet, ed è particolarmente utile per la creazione di VPN sicure
\subsection{Ambiti d'uso}
\begin{enumerate}
\item VPN (Virtual Private Network): utilizzato per creare reti private virtuali sicure su reti pubbliche
\item Comunicazioni sicure tra dispositivi: protegge le comunicazioni tra dispositivi in reti aziendali o su Internet
\end{enumerate}
\subsection{Pregi}
\begin{enumerate}
\item Sicurezza forte: fornisce protezione contro intercettazioni e alterazioni dei dati
\item Compatibilità con vari protocolli di rete: IPSec può essere utilizzato su molteplici protocolli di rete, inclusi IPv4 e IPv6
\end{enumerate}
\subsection{Difetti}
\begin{enumerate}
\item Complessità di implementazione: la configurazione e la gestione di IPSec possono essere complesse
\item Overhead di rete: l'uso di IPSec può introdurre un overhead di rete, rallentando le comunicazioni
\end{enumerate}
