\section{Satelliti (2015, 16, 18, 19, 20 21, 23, 24, 25)}
Esistono tre tipologie principali di satelliti: \\
GEO (geostazionari $>$35km), MEO (compreso fra 5km e 15km), LEO($<$5km)\\
Nelle parti non comprese ci sono le "fascie di Van Allen" le quali assorbono la luce del Sole e causano problemi alle telecomunicazioni\\
Esse devono essere evitate tramite dei "buchi" altrimenti i satelliti si scioglierebbero all'istante\\
Più basso è il satellite più ne servono perché coprono un'area bassa ma i tempi di comunicazione sono ridotti rispetto a  un satellite elevato
\subsection{Satelliti GEO}
\subsubsection{Descrizione}
Sono i satelliti più distanti dalla superficie terrestre, sopra i 35km\\
Essi devono superare le due fasce di Van Allen per restare stazionari nell'orbita circolare dell'Equatore\\
Garantisce una copertura continua di vaste aree terrestri e un posizionamento stabile
\subsubsection{Ambiti d'uso}
Sono utilizzati come satelliti spia, meteo, televisione e internet satellitari, osservazioni della Terra
\subsubsection{Pregi}
\begin{enumerate}
\item Posizionamento stabile e fisso\\
Un satellite GEO rimane costantemente sopra lo stesso punto della Terra, facilitando comunicazioni e trasmissioni continue
\item Copertura geografica ampia\\
Ogni satellite può coprire fino a un terzo della superficie terrestre\\
Con tre satelliti GEO opportunamente posizionati, è possibile garantire una copertura globale 
\item Ideale per comunicazioni e broadcasting\\
L'orbita GEO è ideale per applicazioni come la televisione satellitare, le trasmissioni radio e le telecomunicazioni a lunga distanza, grazie alla copertura costante e affidabile
\item Minor necessità di reti satellitari\\
A differenza delle costellazioni LEO o MEO, che richiedono molti satelliti per fornire copertura continua, pochi satelliti GEO possono coprire vaste aree, riducendo i costi di lancio e manutenzione
\item Riduzione delle complessità di tracciamento\\
Poiché i satelliti GEO appaiono "fissi" nel cielo, non è necessario un sistema complesso per tracciare il loro movimento, semplificando il design delle stazioni a terra
\end{enumerate}
\subsubsection{Difetti}
C'è un limite di 180 satelliti per evitare interferenze tra le frequenze radio e sovrapposizioni delle orbite, inoltre richiedono tantissima energia prodotta comunque dai pannelli solari\\
Sono satelliti fermi nella nostra testa che stanno nell'orbita circolare dell'equatore il che vuol dire che sono meno efficaci alle alte latitudini (vicino ai poli)
\subsection{Satelliti MEO}
\subsubsection{Descrizione}
Sono satelliti che stanno nell'orbita media, ed è stato il primo tipo di satellite lanciato dall'uomo, per la precisione lo Sputnik\\
Essi sono satelliti situati fra le due fasce di Van Allen\\
Si spostano lentamente lungo la longitudine impiegando circa 6 ore per compiere un giro intorno al pianeta causando la necessità del loro rintracciamento
\subsubsection{Ambiti d'uso}
Qui si trovano i satelliti utili per la geolocalizzazione, il cui servizio è attualmente del Dipartimento della Difesa USA
\subsubsection{Pregi}
Attualmente con la tecnologia A-GPS si migliora il GPS e si segnala dove sono presenti i satelliti (tramite le compagnie di rete telefonica) sfruttando l'effetto Doppler\\
Infatti a quest'altitudine si trovano i 24 satelliti GPS necessari
\subsubsection{Difetti}
Il cosmo è pieno di satelliti di tipo MEO che, una volta finito il carburante necessario alla loro ricollocazione, vengono allontanati. Quando un satellite, attivo o meno, si scontra con un altro corpo, esso "esplode" in una miriade di frammenti, che a loro volta costituiscono un rischio per gli altri satelliti attivi. Per questo, i frammenti devono essere tracciati e monitorati, in modo da evitare la possibilità di eventuali effetti domino
\subsection{Satelliti LEO}
\subsubsection{Descrizione}
Utilizzato per l'Internet satellitare e il telefono satellitare, ideato col progetto Iridium (sistema di 77 satelliti diventati 66) per coprire l'intera superficie terrestre\\
Ha la comodità di non dover superare alcuna fascia di Van Allen per essere lanciato nello spazio creando così meno difficoltà nella realizzazione e nella progettazione\\
Si ha un tempo di rivoluzione breve (dai 90 minuti alle 2 ore)
\subsubsection{Ambiti d'uso}
Attualmente fa parte dello Tsunami Warning System\\
Forniscono connessioni Internet a banda larga, permettono di osservare la Terra, sono utilizzati per scopi militari, di sicurezza e di ricerca scientifica (studi sull'atmosfera)
\subsubsection{Pregi}
Il tempo di latenza è molto breve, i costi di lancio e costruzione sono ridotti rispetto ai satelliti più elevati\\
Si ha un'alta risoluzione per ottenere immagini più dettagliate\\
Sono facili da aggiornare, nel senso che le costellazioni di satelliti possono essere integrate o sostituite rapidamente\\
Le stazioni terrestri non hanno bisogno di molta energia, il ritardo nelle comunicazioni è di pochi millisecondi
\subsubsection{Difetti}
Avendo una durata operativa molto bassa, ora sono presenti tantissimi detriti spaziali che rischiano di scombussolare l'intero sistema dei satelliti causando un effetto domino\\
Inoltre hanno una copertura limitata e dei costi di manutenzione elevati (nel senso che sono necessari lanci frequenti)
