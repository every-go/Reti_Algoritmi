\section{AES}
\subsection{Descrizione}
L'Advanced Encryption Standard (AES) è il successore del Data Encryption Standard (DES) ed è l'attuale standard mondiale per la crittografia simmetrica\\
AES è composto da diversi algoritmi di cifratura a blocchi, utilizzando chiavi di lunghezza variabile\\
Le chiavi e i blocchi possono essere scelti a 128 bit, 192 bit o 256 bit, sin dal momento della creazione dello standard, permettendo l'evoluzione della sicurezza e la scalabilità dello stesso quando necessario
\subsection{Ambiti d'uso}
AES è ampiamente utilizzato in molti ambiti, tra cui:
\begin{enumerate}
\item Crittografia dei dati: protezione dei dati sensibili in archiviazione e durante il trasferimento (ad esempio, in HTTPS, VPN, e-mail criptate).
\item Sistemi di autenticazione: protezione delle password e delle credenziali di accesso
\item Comunicazioni sicure: impiegato nei protocolli di sicurezza come SSL/TLS, WPA2 (Wi-Fi Protected Access), e IPsec
\item Sistemi di pagamento: crittografia delle transazioni bancarie e carte di credito
\end{enumerate}
\subsection{Pregi}
I principali vantaggi di AES includono:
\begin{enumerate}
\item Elevata sicurezza: AES è considerato sicuro contro attacchi come il brute force, e le chiavi di 256 bit offrono un livello di sicurezza molto elevato
\item Velocità: AES è molto veloce sia in hardware che in software, ed è progettato per essere efficiente anche in dispositivi con risorse limitate
\item Resilienza: AES è resistente a vari tipi di attacchi crittografici, come quelli basati su analisi delle frequenze
\item Standard ampiamente supportato: AES è universalmente adottato e supportato in numerosi dispositivi e software di sicurezza
\end{enumerate}
\subsection{Difetti}
Nonostante i numerosi vantaggi, AES presenta alcuni svantaggi:
\begin{enumerate}
\item Efficienza in dispositivi a bassa potenza: sebbene AES sia efficiente in hardware, può risultare meno ottimizzato in ambienti a bassa potenza o risorse limitate se implementato in software
\item Dipendenza dalla gestione delle chiavi: la sicurezza di AES dipende fortemente dalla corretta gestione delle chiavi. Se le chiavi vengono compromesse, la protezione dei dati è compromessa
\end{enumerate}
