\section{Modi di attaccare DNS (2018)}
\subsection{Descrizione}
Gli attacchi al Domain Name System (DNS) mirano a compromettere la risoluzione dei nomi di dominio, influenzando le comunicazioni su Internet. Due delle principali tecniche di attacco sono il DNS spoofing e il DDoS (Distributed Denial of Service)\\
Il DNS spoofing consiste nell'inviare risposte DNS false a un client o a un server DNS, ingannando il sistema di risoluzione dei domini e redirigendo il traffico verso destinazioni non desiderate\\
Il DDoS contro i server DNS può essere utilizzato per sovraccaricare un server DNS con un volume massiccio di traffico, impedendo che possa risolvere correttamente i domini per gli utenti legittimi
\subsection{Ambiti d'uso}
\begin{enumerate}
\item Attacchi contro la disponibilità di siti web: utilizzati per dirottare il traffico Internet verso server malevoli (DNS spoofing) o per sovraccaricare i server DNS, rendendo i siti web e i servizi online irraggiungibili (DDoS)
\item Attacchi di phishing: il DNS spoofing viene utilizzato per reindirizzare gli utenti a siti web fasulli, rubando informazioni sensibili come credenziali di login, dati bancari, e altro
\item Interruzione di servizio: il DDoS su server DNS è utilizzato per impedire la risoluzione dei domini, causando interruzioni nei servizi di rete
\end{enumerate}
\subsection{Pregi}
\begin{enumerate}
\item Semplicità: gli attacchi come il DNS spoofing sono relativamente facili da eseguire, richiedendo poche risorse e conoscenze tecniche
\item Efficacia: Il DNS spoofing può compromettere rapidamente la navigazione web, reindirizzando gli utenti a siti malevoli
\end{enumerate}
\subsection{Difetti}
\begin{enumerate}
\item Difesa con DNSSEC: meccanismi di protezione come DNSSEC (DNS Security Extensions) possono prevenire il DNS spoofing
\item Difficoltà di difesa contro DDoS: i DDoS contro i server DNS richiedono soluzioni scalabili come il bilanciamento del carico e l'uso di reti distribuite per mitigare l'impatto
\end{enumerate}
