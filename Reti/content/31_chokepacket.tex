\section{Choke packet (2016, 18, 19, 21, 22, 23, 24)}
\subsection{Descrizione}
Un choke packet funziona similmente al sistema a crediti di un bus: quando un router rileva una congestione nel flusso di trasmissione, si occupa di spedire ai vari mittenti un pacchetto di congestione, che obbliga un ritmo di spedizione moderato\\
Dopo un po’ il flusso tornerà al suo ritmo originale autonomamente (fading)
\subsection{Ambiti d'uso}
Il suo ambito d'uso principale è al fine di "decongestionare" la rete, infatti con questa tecnica si dimezza il flusso dati in uscita e permette di elaborare al meglio i pacchetti già arrivati
\subsection{Pregi}
\begin{enumerate}
\item Riduzione della Congestione\\
Aiuta a limitare la congestione della rete comunicando rapidamente al mittente di rallentare il flusso di dati, prevenendo ulteriori sovraccarichi
\item Semplicità del Meccanismo\\
La logica alla base del choke packet è relativamente semplice da implementare: il nodo rileva il sovraccarico e invia un messaggio di feedback al mittente
\item Adattamento Dinamico del Traffico\\
Permette al sistema di adattarsi in tempo reale alle condizioni della rete, mantenendo un flusso dati che evita il collasso
\item Minimizzazione della Perdita di Pacchetti\\
Riducendo la velocità di invio prima che la rete raggiunga uno stato critico, si evita la perdita massiccia di pacchetti dovuta alla saturazione dei buffer
\item Compatibilità con Altri Meccanismi di Controllo\\
Il choke packet può essere utilizzato insieme ad altri metodi di controllo della congestione, come il controllo basato sulla finestra (ad esempio TCP), per migliorare l'efficacia complessiva
\end{enumerate}
\subsection{Difetti}
Il suo principale problema è "il problema dell'entrata"\\
Se si ha una sequenza di router A, B, C, D, E ed avviene la congestione della rete fra A ed E i router B, C, D inviano ad A un choke portando il suo data rate a 12,5$\%$\\
Infatti, quando si riceve un choke, per un certo periodo di tempo detto fading alla rovescia (minore del fading) si ignorano degli eventuali altri choke in arrivo\\
Un altro problema è che una richiesta di choke può metterci troppo per decongestionare la rete, infatti si è progettata una variante choke hop-by-hop in cui ogni router incontrato subisce gli effetti del choke
