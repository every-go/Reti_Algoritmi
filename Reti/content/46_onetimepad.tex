\section{One time pad (blocco monouso) (2015, 16, 18, 20)}
\subsection{Descrizione}
Il One Time Pad (OTP), o cifrario a blocco monouso, è un metodo di cifratura simmetrica che garantisce sicurezza perfetta, a condizione che venga utilizzato correttamente\\
Il testo in chiaro viene combinato con una chiave completamente casuale, lunga almeno quanto il messaggio, utilizzando lo XOR (esclusivo logico)\\
Ogni chiave deve essere usata una sola volta, da cui il nome "blocco monouso"\\  
L'OTP è teoricamente inviolabile, ma è inutilizzabile in tutti i contesti moderni, è utilizzato infatti solamente in situazioni eccezionali come comunicazioni segretissime che non possono in nessuna situazione trapelare all'esterno
\subsection{Ambiti d'uso}
\begin{enumerate}
\item Comunicazioni militari e diplomatiche: utilizzato per la trasmissione di informazioni estremamente sensibili, come i messaggi tra capi di stato o militari durante la guerra
\item Sistemi ad alta sicurezza: Impiegato in sistemi dove la sicurezza assoluta è prioritaria, come nel caso di reti di intelligence
\item Applicazioni teoriche: usato come esempio ideale per comprendere i limiti e i principi fondamentali della crittografia
\end{enumerate}
\subsection{Pregi}
\begin{enumerate}
\item Sicurezza perfetta: Se usato correttamente, l'OTP è matematicamente inviolabile, perché il testo cifrato non contiene informazioni statistiche sul testo in chiaro
\item Semplicità matematica: Il metodo utilizza operazioni semplici, come lo XOR, che lo rendono facile da implementare
\item Resistenza a tutti gli attacchi crittografici: nessun attacco basato sulla crittoanalisi può violare l'OTP se le sue regole vengono seguite
\end{enumerate}
\subsection{Difetti}
\begin{enumerate}
\item Gestione della chiave: richiede una chiave lunga quanto il messaggio, che deve essere distribuita e conservata in modo assolutamente sicuro
\item Uso unico della chiave: ogni chiave può essere utilizzata una sola volta, aumentando la complessità logistica nelle comunicazioni frequenti
\item Impraticabilità per messaggi lunghi: per messaggi di grandi dimensioni, la gestione delle chiavi diventa onerosa
\item Difficoltà di distribuzione: la trasmissione sicura della chiave tra le parti è un problema significativo, soprattutto in scenari moderni.
\item Dipendenza dalla casualità: la sicurezza dipende dalla qualità della chiave, che deve essere completamente casuale
\end{enumerate}
