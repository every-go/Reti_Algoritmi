\section{ARP (Address Resolution Protocol) (2014, 17, 18, 19, 20, 22, 23, 24, 25)}
\subsection{Descrizione}
L'Address Resolution Protocol (ARP) è un protocollo appartenente al livello di rete (network layer) che consente di trovare la corrispondenza tra un indirizzo IP e un indirizzo MAC (Media Access Control) necessario per la comunicazione all'interno di una rete locale (LAN)\\
Il suo funzionamento può essere riassunto come segue:
\begin{itemize}
\item Ogni macchina all'interno della rete mantiene una tabella ARP, che contiene le corrispondenze tra indirizzi IP e indirizzi MAC conosciuti.
\item Quando un dispositivo deve inviare un pacchetto a un altro dispositivo della rete e conosce solo l'indirizzo IP di destinazione, genera un messaggio ARP in broadcast (richiesta ARP).
\item La macchina destinataria, riconoscendo il proprio indirizzo IP nella richiesta, risponde con un messaggio ARP di risposta (unicast) contenente il proprio indirizzo MAC.
\item La corrispondenza IP-MAC viene salvata nella cache ARP del dispositivo richiedente per ottimizzare le comunicazioni future.
\end{itemize}
Ogni pacchetto ARP include in piggyback (ossia come dato aggiuntivo) la corrispondenza tra indirizzo IP e indirizzo MAC, permettendo così il continuo aggiornamento della tabella ARP
\subsection{Ambiti d'uso}
Il protocollo ARP è fondamentale per la comunicazione in reti IPv4 locali e viene utilizzato in diverse situazioni, tra cui:
\begin{enumerate}
\item Trasmissione in reti Ethernet:\\
consente a un dispositivo di determinare l'indirizzo MAC del destinatario per poter incapsulare correttamente il pacchetto a livello data-link
\item Routing locale:\\
è indispensabile quando un router deve inviare pacchetti a un dispositivo nella rete locale
\item Comunicazioni client-server:\\
viene utilizzato ogni volta che un client locale deve comunicare con un server nella stessa rete, come in reti domestiche o aziendali
\end{enumerate}
\subsection{Pregi}
ARP offre diversi vantaggi nel contesto delle reti:
\begin{enumerate}
\item Semplicità:\\
il protocollo ARP è semplice e diretto, progettato per svolgere una funzione specifica: risolvere gli indirizzi IP in indirizzi MAC
\item Trasparenza:\\
il processo di risoluzione degli indirizzi avviene automaticamente e in modo trasparente per l'utente finale e le applicazioni
\item Efficienza:\\
una volta risolta la corrispondenza, i risultati vengono memorizzati nella cache ARP per ridurre la latenza e migliorare le prestazioni in comunicazioni successive
\item Compatibilità universale:\\
ARP è supportato dalla maggior parte delle reti Ethernet e IPv4, garantendo interoperabilità tra dispositivi di rete
\end{enumerate}
\subsection{Difetti}
Nonostante la sua utilità, il protocollo ARP presenta alcune limitazioni e vulnerabilità:
\begin{enumerate}
\item Dipendenza da IPv4:\\
ARP è progettato per IPv4, e quindi non è utilizzabile in reti IPv6
\item Possibili sovraccarichi di rete:\\
in reti molto grandi o con dispositivi instabili, un numero elevato di richieste ARP può causare congestione, degradando le prestazioni della rete
\end{enumerate}
