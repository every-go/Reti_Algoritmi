\section{Bit o baud rate}
\subsection{Descrizione}
Sono metriche di misurazione dell'informazione trasmessa ogni secondo e permettono di calcolare la larghezza di banda\\
Bit rate: è il numero di bit che si possono trasmettere contemporaneamente con ogni impulso, è uguale al baudrate*log2(numero simboli)\\
Baud rate: si trasmette un impulso usando 4 frequenze con l'alfabeto composto da 4 simboli con ognuno dal peso di 2 bits
\subsection{Ambiti d'uso}
Utilizzato per calcolare le larghezze di banda di diversi sistemi e confrontarli, trasferimento di file, comunicazioni digitali, media streaming, sistemi di trasmissione digitale con modulazione complessa
\subsection{Pregi}
\subsubsection{Bit}
Indicatore di qualità:\\
più alto è il bit rate, maggiore è la qualità dei dati trasmessi (ad esempio, immagini meno compresse nei video)\\
Scalabilità:\\
adattabile in base alle capacità del canale (esempio: streaming adattivo)\\
Misura diretta:\\
facile da interpretare come velocità di trasmissione
\subsubsection{Baud}
Misura dell'efficienza spettrale:\\
indica quanto efficacemente la banda di frequenza è utilizzata\\
Rilevanza per il canale fisico:\\
permette di ottimizzare la trasmissione per canali con larghezza di banda limitata
\subsection{Difetti}
\subsubsection{Bit}
Non tiene conto dell'efficienza del canale:\\
un bit rate elevato può consumare molta larghezza di banda anche se il canale non è utilizzato in modo efficiente\\
Dipende dalla codifica\\
un bit rate elevato non sempre significa qualità migliore; dipende dall'efficienza del codice usato\\
Suscettibilità agli errori:\\
a velocità più alte, i dati possono essere più vulnerabili al rumore e alle interferenze\\
\subsubsection{Baud}
Non direttamente legato alla velocità dei dati:\\
il baud rate da solo non indica quanti bit vengono effettivamente trasmessi (esempio: un simbolo può rappresentare più bit)\\
Dipendenza dalla modulazione:\\
richiede informazioni aggiuntive sulla tecnica di modulazione per essere interpretato correttamente\\
Più difficile da misurare:\\
in sistemi avanzati, calcolare il baud rate richiede una conoscenza precisa dei dettagli del sistema
