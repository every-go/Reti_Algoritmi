\section{Stop and wait}
\subsection{Descrizione}
Stop and wait è uno dei protocolli half-duplex (canale singolo) in cui tra ricevente e mittente si condivide un singolo canale e le comunicazioni avvengono in modo alternato\\
Il mittente trasmette un blocco dati (frame) e attende un segnale di conferma (acknowledgment) segnalandoci che si può inviare un altro messaggio\\
Se il ricevente rileva errori nel frame ricevuto, invia un messaggio di richiesta di ritrasmissione (NAK, negative acknowledgment)\\
Se il mittente non riceve nessun ACK entro un tempo prestabilito (timeout), ritrasmette il frame\\
Per evitare duplicazioni nel caso di ritrasmissioni, ogni frame include un identificatore (di solito un bit, 0 o 1) che permette di distinguere i dati già ricevuti da quelli nuovi
\subsection{Ambiti d'uso}
Viene utilizzato principalmente per gestire il flow control in situazioni dove è cruciale garantire l'affidabilità come nei collegamenti a bassa velocità o in ambienti con elevate probabilità di errore\\
Attualmente trova applicazione in contesti semplici
\subsection{Pregi}
Semplicità: l'implementazione è molto semplice, rendendolo ideale per applicazioni basilari o sistemi con risorse limitate\\
Robustezza: garantisce l'integrità dei dati grazie al meccanismo di ritrasmissione e alla conferma esplicita (ACK)\\
Compatibilità: funziona su canali half-duplex, dove la comunicazione simultanea non è possibile, riducendo i requisiti hardware
\subsection{Difetti}
Lentezza: poiché non è possibile inviare un nuovo frame prima di ricevere l'ACK del precedente, il protocollo introduce notevoli ritardi, soprattutto su canali con alta latenza\\
Bassa efficienza: utilizza male la larghezza di banda disponibile, dato che il canale rimane inattivo durante l'attesa degli ACK\\
Limiti in contesti moderni: non è adatto a sistemi ad alta velocità o reti con grandi volumi di dati, dove protocolli più avanzati (come sliding window) risultano preferibili
