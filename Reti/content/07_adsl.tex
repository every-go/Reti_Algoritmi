\section{ADSL (2024)}
\subsection{Descrizione}
L'Asymmetric DSL (ADSL) è un tipo di DSL (Digital Subscriber Line) le quali sono nate per via della crescente necessità di maggiore capacità di download come streaming video e contenuti multimediali\\
L'ADSL sfrutta la rete telefonica esistente, rimuovendo i tradizionali filtri limitati a 4 kHz e ampliando lo spettro di frequenza fino a 1,1 MHz\\
Per evitare interferenze tra il segnale telefonico e quello internet, viene introdotto lo splitter, un filtro economico che separa le due bande (voce e dati)
\subsection{Ambiti d'uso}
Usata come sistema di connessione per abitazioni private, piccole e medie imprese, appartamenti e condomini
\subsection{Pregi}
Si può spezzare la banda in 256 sottocanali da 4312.5Hz (1 voce, 5 vuoti, 32 upload, resto download) e indipendenti, ovvero ogni canale viene trattato come una connessione telefonica a sé stante e c'è controllo costante sulla qualità della trasmissione $\to$ ogni canale può essere rallentato/accelerato indipendentemente\\
Inoltre si ha un costo contenuto e un'accessibilità diffusa
\subsection{Difetti}
La velocità e la qualità della connessione ADSL diminuiscono significativamente con l'aumentare della distanza dell'abitazione o dell'ufficio dalla centrale telefonica\\
Ad esempio, a distanze superiori a 4-5 km dalla centrale, la velocità può ridursi drasticamente o la connessione potrebbe diventare instabile\\
Oltretutto un altro suo difetto è l'asimmetria, in quanto la banda disponibile per il download è molto maggiore rispetto a quella per l'upload limitando le applicazioni che richiedono connessione bilanciata\\
Infine oggi è ormai superata da tecnologie più moderne e veloci come fibra ottica, 5G, VDSL
