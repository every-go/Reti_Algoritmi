\section{Stazione nascosta (2019, 20, 24)}
\subsection{Descrizione}
Nei casi wireless la topologia della rete non è fissa ma cambia dinamicamente causando il fatto che non c'è un singolo canale per tutti ma varie zone spaziali dove alcune stazioni interagiscono ed altre no\\
Quindi il controllo diventa locale compromettendo l'invio singolo di dati ma rischiando l'invio contemporaneo di più dati\\
La stazione nascosta è il problema in cui una stazione non riesce a vedere che la stazione alla quale vuole inviare dei dati ne sta già ricevendo altri, quindi ne invia causando l'arrivo contemporaneo di dati alla stessa stazione e di conseguenza la collisione\\
Infatti, nei casi wireless, si trasmette "a bolla" propagando le informazioni e non in linea retta verso l'obiettivo
\subsection{Quando avviene}
Ipotizziamo che ci siano A, B, C in sequenza in cui il segnale a bolla di A, arrivi a B, il segnale a bolla di B arriva ad A e C e il segnale a bolla di C arriva ad A\\
A vuole inviare dei dati a B, però anche C vuole inviare dei dati a B, ma C non sente che A sta già inviando dei dati quindi li invia a B causano la collisione dei dati trasmessi in contemporanea da A e C, la quale è la stazione nascosta per A\\
\subsection{Come risolvere}
Si usa il MACA (Multiple Access with Collision Avoidance) in cui si sfrutta l'idea che chi trasmette renda il suo spazio locale conosciuto anche agli altri\\
Avviene tramite due comandi:
RTS (Request To Send) che contiene l'informazione del frame e a chi si vuole trasmettere\\
CTS (Confirm To Send) il quale è l'ACK\\
Chiunque sente l'RTS e il CTS ma non è né destinatario né mittente non trasmette a loro due fino a quando non è conclusa\\
Per controllare se la trasmissione è conclusa si sfrutta il protocollo Aloha in modalità non persistente che viene usato anche per trasmissioni multiple alla stessa stazione
