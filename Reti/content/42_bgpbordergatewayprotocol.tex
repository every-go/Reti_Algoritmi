\section{BGP: Border Gateway Protocol}
\subsection{Descrizione}
Il BGP è un protocollo che stabilisce le regole per l'instradamento dei dati tra reti autonome, definendo come devono essere condivise e applicate le informazioni sulle rotte quando il traffico passa da una rete all'altra
\subsection{Ambiti d'uso}
Internet: gestisce l'instradamento globale del traffico tra provider di servizi Internet (ISP) e grandi reti aziendali\\
Data center: ottimizza la comunicazione tra reti interne ed esterne nei grandi data center\\
Grandi aziende: permette di gestire infrastrutture di rete distribuite su scala globale con resilienza garantita
\subsection{Pregi}
Scalabilità: è in grado di gestire milioni di rotte, rendendolo adatto alle reti più grandi\\
Flessibilità: permette un controllo dettagliato del routing tramite policy personalizzate\\
Resilienza: garantisce continuità della connessione grazie al supporto per ridondanza e failover\\
Standard globale: è il protocollo di routing principale di Internet e universalmente supportato
\subsection{Difetti}
Complessità: la configurazione e gestione di BGP richiedono competenze tecniche avanzate\\
Sicurezza debole: non dispone di meccanismi di sicurezza intrinseci ed è vulnerabile ad attacchi \\
Errori umani: configurazioni errate possono propagarsi su larga scala, causando problemi globali\\
Dipendenza da policy: le decisioni di instradamento non sempre garantiscono il percorso più rapido o ottimale
