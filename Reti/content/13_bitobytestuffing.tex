\section{Bit o byte stuffing (2017, 18, 19, 20, 22, 23, 24)}
Nel campo delle reti l'escaping è una tecnica fondamentale per garantire che i dati trasmessi siano interpretati correttamente\\
Due tecniche principali sono il byte stuffing e il bit stuffing
\subsection{Byte stuffing}
\subsubsection{Descrizione}
Il byte stuffing consiste nell'aggiungere caratteri speciali di escape per distinguere i dati effettivi da caratteri riservati o delimitatori (FLAG) che indicano l'inizio o la fine del pacchetto\\
Nell'header dati si indica di cosa si tratta\\
Nel payload si caricano i dati necessari per il messaggio\\
Nel trailer, uguale all'header, ci sono i dati per identificare la chiusura del pacchetto\\
FLAG finale: segna la fine del frame\\
Se nel payload compare un carattere uguale al FLAG o un carattere di escape, viene preceduto da un ulteriore carattere di escape per evitarne l'interpretazione errata
\subsubsection{Ambiti d'uso}
Il byte stuffing è utilizzato per codificare pacchetti dati nei protocolli a frame, come nelle comunicazioni seriali o nei protocolli di livello data link (esempio: PPP)
\subsubsection{Pregi}
Metodo semplice e intuitivo per gestire il problema dei caratteri riservati\\
Il primo metodo adottato per realizzare l'escaping risultando storico e consolidato in molte applicazioni
\subsubsection{Difetti}
Il problema potenziale è che ci possono essere molte più flag/escape del necessario, in quanto ogni carattere originale dopo lo stuffing è preceduto da un escape, e se già ci sono degli escape nel messaggio originale in quello finale ce ne saranno molti di più rendendo lunga la decodifica del messaggio\\
Oltretutto usa grandezze fisse il che lo rende inefficiente in contesti che richiedono maggiore flessibilità
\subsection{Bit stuffing}
\subsubsection{Descrizione}
Il bit stuffing è una tecnica più sofisticata che lavora a livello di bit anziché di byte\\
In questa tecnica, viene aggiunto un bit 0 ogni volta che nel flusso di dati appaiono cinque bit consecutivi impostati a 1\\ Questo serve a evitare che il ricevitore interpreti erroneamente una sequenza di bit come un FLAG 
\subsubsection{Ambiti d'uso}
Il bit stuffing è ampiamente utilizzato in protocolli di comunicazione ad alta efficienza, come HDLC (High-Level Data Link Control), e altre tecnologie di trasmissione dati in cui è essenziale ottimizzare la trasmissione rispetto al byte stuffing
\subsubsection{Pregi}
Risolve il problema della grandezza fissa usando i bit e risolve il problema degli escaping multipli\\
Con la tecnica dello 0 dopo cinque 1 si ha un solo livello di escaping consentendo una più veloce decodifica
\subsubsection{Difetti}
\begin{enumerate}
\item Aumento della lunghezza del frame in quanto vengono aggiunti bit supplementari causando un aumento della lunghezza totale del frame, riducendo l'efficienza della trasmissione, specialmente in sistemi con messaggi lunghi
\item Maggiore complessità del decoder\\
Il dispositivo ricevente deve avere la capacità di rilevare e rimuovere i bit aggiunti
\item Se un errore di trasmissione altera i bit "stuffati" o la sequenza originale, il ricevitore potrebbe interpretare erroneamente i dati, perdendo la sincronizzazione con il flusso di bit
\item Maggiore complessità del debugging:\\
Durante il debug della comunicazione, i bit aggiunti rendono più complessa l'analisi del flusso dati, richiedendo strumenti in grado di gestire e interpretare correttamente il bit stuffing
\end{enumerate}
