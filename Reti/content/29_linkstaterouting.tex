\section{Link State Routing (2018, 21, 22, 23)}
\subsection{Descrizione}
Il Link State Routing è un algoritmo di routing che consente ai nodi di una rete di ottenere una visione completa della topologia di rete, permettendo il calcolo dei percorsi ottimali per l'instradamento dei pacchetti\\
Il processo si articola in diversi passaggi:\\
Rilevamento dei vicini: ogni nodo identifica i propri nodi vicini tramite l’invio di pacchetti HELLO\\
Misurazione delle distanze: i nodi misurano la distanza dai loro vicini utilizzando pacchetti ECHO\\
Costruzione delle informazioni locali: ogni nodo raccoglie informazioni sui propri vicini e sulla distanza che li separa, costruendo un pacchetto denominato Link State Packet (LSP)\\
Broadcast delle informazioni: il pacchetto LSP viene inviato a tutti gli altri nodi della rete tramite un’operazione di broadcast basata sul flooding\\
Ricostruzione della mappa globale: ogni nodo riceve le informazioni locali di tutti gli altri nodi, ricostruendo così una mappa completa della rete\\
Calcolo dei percorsi ottimali: Utilizzando la mappa completa, ogni nodo applica un algoritmo di instradamento (ad esempio, l’algoritmo di Dijkstra) per calcolare i percorsi migliori verso ciascuna destinazione\\
Poiché la topologia della rete può variare nel tempo, il processo di broadcast deve essere ripetuto periodicamente per effettuare il refresh delle informazioni, garantendo che la mappa sia aggiornata e accurata\\
Questo comporta un maggiore utilizzo di banda rispetto ad altri algoritmi, ma migliora notevolmente la robustezza globale della rete, superando i limiti di soluzioni basate su informazioni locali
\subsection{Ambiti d’uso}
Il Link State Routing è utilizzato in contesti in cui è necessario un routing altamente dinamico, accurato e robusto\\
Gli ambiti principali includono:\\
Reti complesse con topologie dinamiche: ad esempio, reti di grandi dimensioni come le reti geografiche (WAN) o le dorsali Internet, dove la topologia varia frequentemente\\
Reti ad alta priorità sulla robustezza: applicazioni critiche che richiedono affidabilità elevata, come reti di provider di servizi e sistemi di controllo industriale
\subsection{Pregi}
\begin{enumerate}
\item Conoscenza globale della rete: ogni nodo dispone di una mappa completa della rete, consentendo di calcolare i percorsi ottimali per qualsiasi destinazione
\item Adattabilità ai cambiamenti: grazie al processo periodico di broadcasting, l’algoritmo reagisce rapidamente ai cambiamenti nella topologia della rete, mantenendo la mappa aggiornata
\item Robustezza: essendo basato su informazioni globali, il Link State Routing evita i problemi legati alla conoscenza locale, riducendo il rischio di instradamenti inefficienti o errori di rete
\item Calcolo di percorsi ottimali: utilizzando algoritmi come Dijkstra, garantisce sempre il percorso più corto e veloce tra i nodi
\item Scalabilità: funziona bene anche in reti di grandi dimensioni, grazie alla precisione e alla granularità delle informazioni trasmesse
\end{enumerate}
\subsection{Difetti}
\begin{enumerate}
\item Elevato consumo di banda: il processo di broadcasting dei pacchetti LSP richiede una quantità significativa di larghezza di banda, specialmente in reti di grandi dimensioni
\item Maggiore complessità computazionale: il calcolo dei percorsi ottimali basato su una mappa globale richiede una maggiore capacità di elaborazione da parte dei nodi rispetto agli algoritmi di routing locali
\item Overhead per aggiornamenti periodici: anche quando la topologia della rete non cambia, il flooding deve essere ripetuto periodicamente per mantenere aggiornate le informazioni, causando un sovraccarico inutile in assenza di modifiche
\end{enumerate}
