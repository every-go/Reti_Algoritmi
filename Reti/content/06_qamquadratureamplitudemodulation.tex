\section{QAM, Quadrature Amplitude Modulation (2019, 20, 22, 23)}
\subsection{Descrizione}
Ci sono diversi tipi di QAM:\\
Col QAM-16 si combinano più tipi di modulazione (4 ampiezze e 4 fasi per un totale di 16 combinazioni) in modo che se qualcosa viene attenuato o disperso il sistema è più robusto permettendo la trasmissione di 4 bit per simbolo\\
Poi esiste anche il QAM-64 il quale permette di arrivare a un bitrate sestuplo rispetto ai baud (6 bit per simbolo) e 3 volte quello dei QPSK
\subsection{Ambiti d'uso}
Usato principalmete per telecomunicazioni e reti dati come DSL, ADSL, Reti wireless, tecnologie 4G e 5G, per la televisione digitale e il broadcasting, per le comunicazioni satellitari, per il modem e la trasmissione dati su fibra ottica
\subsection{Pregi}
Può trasmettere più bit per simbolo rispetto ad altre tecniche di modulazione aumentando capacità del canale senza aumentare la larghezza di banda\\
Maggiore velocità di trasmissione\\
Flessibilità (esistono molte varianti)\\
Compatibilità con diverse tecnologie\\
Supporto per applicazioni moderne (streaming video, gaming online e in generale per alta velocità)
\subsection{Difetti}
Richiede canali di alta qualità e un hardware complesso\\
Limitazioni dei QAM rettangolari: Sebbene più semplici da generare, non sono ottimali come i QAM circolari, che però sono più difficili da implementare in pratica e per questo si preferisce i QAM rettangolari
