\section{RSA}
\subsection{Descrizione}
RSA è un algoritmo di crittografia a chiave pubblica, ampiamente utilizzato per la protezione dei dati\\
La sicurezza di RSA si basa sulla difficoltà di fattorizzare numeri molto grandi, il che lo rende adatto per applicazioni come la firma digitale e lo scambio sicuro di chiavi
\subsection{Ambiti d'uso}
\begin{enumerate}
\item Firma digitale: utilizzato per verificare l'autenticità dei documenti e dei messaggi
\item Scambio sicuro di chiavi: usato per stabilire connessioni sicure tra due parti
\end{enumerate}
\subsection{Pregi}
\begin{enumerate}
\item Sicurezza basata su matematica solida: la sicurezza di RSA è ben compresa e si basa su principi matematici comprovati
\item Flessibilità: può essere utilizzato sia per la crittografia che per la firma digitale
\end{enumerate}
\subsection{Difetti}
\begin{enumerate}
\item Bassa velocità: RSA è più lento rispetto ad altri algoritmi simmetrici, il che lo rende meno ideale per cifrare grandi quantità di dati
\item Vulnerabilità alla fattorizzazione: la sicurezza di RSA può essere compromessa se i numeri utilizzati sono troppo piccoli o se vengono sviluppati metodi di fattorizzazione più efficienti
\end{enumerate}
