\documentclass[a4paper,12pt]{article}
\usepackage[utf8]{inputenc}
\usepackage{amssymb, booktabs, tabularx}  % Per simboli come \square
\usepackage{enumitem} % Per liste personalizzate
\usepackage{tikz}     % Per disegnare quadrati con lettere dentro
\usepackage{amsthm,amsmath,amssymb,ragged2e}

% Comando per il quadrato con lettera
\newcommand{\boxletter}[1]{%
	\tikz[baseline=(char.base)]{
		\node[draw, minimum size=1.2em, inner sep=0pt] (char) {#1};
	}%
}

\pagenumbering{arabic}
\title{Primo compitino algebra e matematica discreta}
\date{2024/2025}
\author{ATTENZIONE: questa è una rivisitazione del primo parziale, non garantisco siano presenti tutti gli esercizi svolti}
\begin{document}
	\maketitle
	\section{Quiz matematica discreta}
	$\sum$d(v) di un grafo semplice è sempre pari? \hfill \boxletter{V} \quad \boxletter{F} \\\\
		$K_3$ completo è sottografo di $K_{3,3}$? \hfill \boxletter{V} \quad \boxletter{F} \\\\
		Esiste grafo bipartito completo con 19 spigoli? \hfill \boxletter{V} \quad \boxletter{F} \\\\
		Il cammino di peso minimo tra 2 vertici è unico se è un albero? \hfill \boxletter{V} \quad \boxletter{F} \\\\
		Ogni albero ha esattamente 2 vertici di grado 1 \hfill \boxletter{V} \quad \boxletter{F} \\\\
		Un grafo che ammette $K_6$ come minore può essere planare \hfill \boxletter{V} \quad \boxletter{F} \\\\
		Secondo teorema di Dirac, se in un grafo esiste un vertice v \\ per ogni d(v) $< \frac{|V|}{2}$, il grafo non è Hamiltoniano \hfill \boxletter{V} \quad \boxletter{F} \\
	\section{Esercizi algebra}
	\subsection{Congruenze}
	Si risolva la seguente congruenza:\\
	10x $\equiv$ 16 mod 54\\
	\subsection{Esercizi matrice}
	Sia data A($\alpha$)=$\begin{pmatrix} 1 & \alpha + 1 & 1 \\
								1 & \alpha + 1 & -\alpha \\
								0 & \alpha + 1 & 0
				\end{pmatrix}$
	\newline
	\newline
	\newline
	Sia data b($\alpha$)=$\begin{pmatrix} 2  \\
								2  \\
								\alpha + 1
				\end{pmatrix}$
	\newline
	\newline
	\newline
	Si risolva il sistema lineare A($\alpha$) = b($\alpha$)
\end{document}