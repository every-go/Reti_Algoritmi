\documentclass[12pt,oneside,a4paper]{article}
%--------------------------------------------------------------------
%--------------------------------------------------------------------
\newcounter{choice}
\renewcommand\thechoice{\Alph{choice}}
\newcommand\choicelabel{\thechoice.}

\newenvironment{choices}%
  {\list{\choicelabel}%
     {\usecounter{choice}\def\makelabel##1{\hss\llap{##1}}%
       \settowidth{\leftmargin}{W.\hskip\labelsep\hskip 2.5em}%
       \def\choice{%
         \item
       } % choice
       \labelwidth\leftmargin\advance\labelwidth-\labelsep
       \topsep=0pt
       \partopsep=0pt
     }%
  }%
  {\endlist}

\newenvironment{oneparchoices}%
  {%
    \setcounter{choice}{0}%
    \def\choice{%
      \refstepcounter{choice}%
      \ifnum\value{choice}>1\relax
        \penalty -50\hskip 1em plus 1em\relax
      \fi
      \choicelabel
      \nobreak\enskip
    }% choice
    % If we're continuing the paragraph containing the question,
    % then leave a bit of space before the first choice:
    \ifvmode\else\enskip\fi
    \ignorespaces
  }%
  {}
%--------------------------------------------------------------------
%--------------------------------------------------------------------

%--------------------------------------------------------------------
\usepackage{amsthm,amsmath,amssymb,ragged2e}
\pagenumbering{arabic}
\title{Secondo compitino algebra e matematica discreta}
\date{2023/2024}
\begin{document}
\maketitle
\section{Quiz matematica discreta}
\begin{enumerate}
\item Quante sequenze di 6 cifre posso costruire con quattro 1 e due 2?
  \begin{choices}
    \choice 6!
    \choice 6!/4!
    \choice 15
    \choice 30
  \end{choices}
\item Qual è la soluzione di a$_{n}$=-8a$_{n-1}$?
  \begin{choices}
    \choice a$_{n}$=8$^{n}$
    \choice a$_{n}$=8$^{n}$+8
    \choice a$_{n}$=(-1)$^{n}$8$^{n+1}$
    \choice a$_{n}$=(-8)$^{n}$
  \end{choices}
\item In quanti modi si possono disporre 9 oggetti distinti in 6 scatole
  \begin{choices}
    \choice 9$^{6}$
    \choice 6$^{9}$
    \choice 15!/9!6!
    \choice 14!/8!6!
  \end{choices}
\end{enumerate}
\section{Esercizi algebra}
\textbf{Esercizio 1}\newline
Sia data A=$\begin{pmatrix}$i$ & n & -n\\0 & $i$ & 0 \\0 & $i$ & 0\end{pmatrix}$
con n lettere del proprio nome\newline
a) Si dica se A è unitariamente diagonalizzabile, motivando la risposta\newline
b) Sia U=C(A). Trovare base \textit{\textbf{B}}  di \textit{\textbf{U}}\newline
c) Sia U=C(A)$^{\perp}$. Trovare base \textit{\textbf{D}}  del complemento ortogonale \textit{\textbf{U}}\newline\newline
\textbf{Esercizio 2}\newline
Sia \textit{f}: $\mathbb{R}$$^{2}$$\to$$\mathbb{R}$$^{2}$ definita da:\newline
\[\textit{f} :\begin{pmatrix}a\\b\end{pmatrix}\to \begin{pmatrix}5a\\-5b\end{pmatrix}\]\newline
Trovare matrice A associata ad \textit{f} con basi ordinate\newline
\[\textit{\textbf{B}} = \{\begin{pmatrix}1\\ 1\end{pmatrix},\begin{pmatrix}1\\ -1\end{pmatrix}\}  e  \textit{\textbf{D}} =\{\begin{pmatrix}1\\ 5\end{pmatrix},\begin{pmatrix}1\\ -5\end{pmatrix}\}\]\newline
su dominio e codominio rispettivamente\newline\newline
\textbf{Esercizio 3}\newline\newline
Sia A=$\begin{pmatrix}5 & -2 \\ -2 & 2\end{pmatrix}$\newline
a) A è unitariamente diagonalizzabile, dimostrarlo\newline
b) scrivere A nella forma A=$\lambda$$_{1}$\textit{P}$_{1}$+$\lambda$$_{2}$\textit{P}$_{2}$ con  $\lambda$$_{1}$ e $\lambda$$_{2}$ autovalori di A e \textit{P}$_{1}$ e \textit{P}$_{2}$ matrici di proiezione su EA( $\lambda$$_{1}$) ed EA( $\lambda$$_{1}$) rispettivamente
\end{document}