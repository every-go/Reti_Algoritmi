\section{Definizione e vantaggi della memoizzazione}
La memoizzazione è una tecnica utilizzata nella programmazione dinamica che consente di evitare la risoluzione ripetuta di sottoproblemi, un fenomeno che può diventare particolarmente problematico in esempi come il calcolo della \textit{Longest Common Subsequence (LCS)}, della \textit{Longest Increasing Subsequence (LIS)}, o anche in problemi più semplici come la sequenza di \textit{Fibonacci}\\
In pratica, la memoizzazione consiste nel salvataggio dei risultati dei sottoproblemi risolti in una struttura dati, in modo che ogni sottoproblema venga calcolato solo una volta\\
Successivamente, se il sottoproblema viene richiesto di nuovo, il risultato viene recuperato dalla struttura dati, evitando di ricalcolarlo\\
Inoltre, la memoizzazione consente di ridurre drasticamente la complessità di un algoritmo\\
Ad esempio, nel caso del calcolo della LCS, si passa da una complessità di \( \Theta\left( \binom{m}{n} \right) \) a una complessità di \( \Theta(m \cdot n) \), grazie alla memorizzazione dei risultati intermedi
