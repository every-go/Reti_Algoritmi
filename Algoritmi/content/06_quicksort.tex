\section{Complessità quicksort e spiegazione breve algoritmo, caso medio, peggiore e perché tante ripartizioni sono caso medio}
Il quicksort è un algoritmo del tipo divide et impera, il quale, prendendo un pivot, porta alla sinistra dell'array gli elementi $<$= e alla destra gli elementi $>$= tramite la funzione Partition (divide), e successivamente esegue ricorsivamente il quicksort sulla parte sinistra e parte destra della partizione (impera)\\
\begin{lstlisting}[style=pseudocodice]
QuickSort(A,p,r)
if p<r
	q=Partition(A,p,r)
	QuickSort(A,p,q-1)
	QuickSort(A,q+1,r)
\end{lstlisting}
La sua complessità nel caso peggiore è del tipo $\Omicron$($n^2$) perché il caso peggiore avviene quando l'array è già ordinato e si avrebbe un tempo di esecuzione T(n)=T(n-1)+$\Omega$(n)+T(0) poiché T(n)=$\sum_{i=1}^n (a(n-i)+b)$ \\
La sua complessità nel caso medio è del tipo $\Omicron$(nlogn)\\
Il partizionamento proporzionale o sbilanciato, studiandolo tramite gli alberi di ricorsione, porta a dire che l'altezza dell'albero è di tipo logaritmica con una forma del tipo T(n)=$\Omega$(n)+T(k)+T(n-k-1) che porta ad affermare che T(n)$<$=cnlogn (base del partizionamento) $\to$ T(n)=$\Omicron$(nlogn)
