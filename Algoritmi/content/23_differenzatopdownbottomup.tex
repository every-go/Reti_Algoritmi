\section{Differenza tra approccio top-down e bottom-up per la programmazione dinamica. Vantaggi e svantaggi di entrambi}
L'approccio \textit{top-down} è una tecnica di risoluzione dei problemi in cui si parte dal problema globale, per poi suddividerlo ricorsivamente in sottoproblemi sempre più piccoli fino a risolverli\\
Dopo aver risolto i sottoproblemi, si risale per combinare i risultati e ottenere la soluzione del problema iniziale\\
Questo approccio è particolarmente utile quando i sottoproblemi sono ben definiti e si interconnettono in modo sincrono, come nel caso della sequenza di Fibonacci\\
L'approccio \textit{bottom-up}, invece, è il contrario: si parte dal problema più piccolo, risolvendo prima i sottoproblemi di base, per poi combinare i risultati per risolvere problemi più grandi\\
Questo approccio è tipico di problemi come la \textit{Longest Common Subsequence (LCS)}, in cui si risolvono prima le sequenze più corte per costruire progressivamente la soluzione completa, comunque in generale per problemi con ricorrenze\\
Entrambi gli approcci si prestano molto bene alla memoizzazione, che consente di evitare il ricalcolo di sottoproblemi già risolti, migliorando l'efficienza dell'algoritmo\\
\textbf{Vantaggi dell'approccio top-down:}
\begin{itemize}
\item È più naturale e intuitivo per problemi che si prestano a una soluzione ricorsiva
\item È più facile da implementare in alcuni casi, in quanto si può risolvere il problema direttamente in modo ricorsivo, con l'ausilio della memoizzazione per ottimizzare il calcolo
\item È ideale per problemi in cui non si conosce in anticipo la dimensione della soluzione o la struttura dei sottoproblemi
\end{itemize}
\textbf{Svantaggi dell'approccio top-down:}
\begin{itemize}
\item Può comportare un overhead di memoria e tempo a causa delle chiamate ricorsive, specialmente per problemi che richiedono una grande quantità di memoria stack
\item La ricorsione può essere meno efficiente rispetto a un approccio iterativo, a causa della gestione del controllo delle chiamate
\item Non è sempre facile evitare il ricalcolo dei sottoproblemi senza una memoizzazione adeguata
\end{itemize}
\textbf{Vantaggi dell'approccio bottom-up:}
\begin{itemize}
\item Non comporta l'overhead della ricorsione, poiché i sottoproblemi vengono risolti in ordine crescente senza chiamate ricorsive
\item È più efficiente in termini di memoria, in quanto non richiede lo stack di chiamate della ricorsione
\item È spesso più semplice da ottimizzare, poiché i sottoproblemi vengono risolti in modo iterativo e non richiedono una gestione complessa della ricorsione
\end{itemize}
\textbf{Svantaggi dell'approccio bottom-up:}
\begin{itemize}
\item Può essere meno intuitivo per problemi che sono naturalmente ricorsivi, rendendo il codice più difficile da scrivere e leggere
\item Potrebbe essere necessario allocare e gestire strutture dati di grandi dimensioni per risolvere il problema, il che può risultare inefficiente in termini di memoria per alcuni problemi
\item In alcuni casi, l'approccio bottom-up potrebbe richiedere una maggiore complessità nell'organizzazione dei sottoproblemi
\end{itemize}
\end{document}