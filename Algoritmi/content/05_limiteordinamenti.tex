\section{Dimostrazione che il limite degli ordinamenti è nlogn}
Il limite degli ordinamenti sfrutta l'albero di decisione, il quale è un albero che descrive in maniera astratta l'esecuzione di algoritmi su input di dimensione n in cui ogni foglia rappresenta una e una sola permutazione dell'input, e ogni permutazione è rappresentata da almeno una foglia. Se ciò non avviene vuol dire che l'agoritmo non è corretto\\
Bisogna comunque dire che $\Omega$(nlogn) si applica solo agli algoritmi che usano confronto tra elementi\\\\
\href{alberodecisione.pdf}{Foto dimostrazione}