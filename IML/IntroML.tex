\documentclass[10pt,oneside,a4paper]{article}
\usepackage{hyperref, amsmath}
\hypersetup{
	colorlinks=true,   % Abilita il colore dei link (senza box intorno)
	linkcolor=blue,    % Colore per i link interni (come quelli nell'indice)
	urlcolor=red,      % Colore per i link URL esterni
	filecolor=magenta, % Colore per i link ai file locali
	citecolor=green,   % Colore per i riferimenti bibliografici
	pdfborder={0 0 0}  % Disabilita i bordi intorno ai link
}
\title{Introduzione all'apprendimento automatico}
\author{Matteo Mazzaretto}
\date{2024/2025}
\begin{document}
\pagenumbering{gobble}
\maketitle
\begin{center}
%do un nuovo nome alla tabella degli indici e la inizializzo
\renewcommand{\contentsname}{Indice}
\tableofcontents
\end{center}
\newpage
%inizio l'effettivo conteggio delle pagine
\pagenumbering{arabic}
\setcounter{page}{1}
\section{Prima parte}
	\subsection{Quali sono i principali paradigmi del machine learning? Se ne riporti una descrizione
		sintetica – chiarendo quali siano le principali differenze – con particolare enfasi per il caso
		del supervised learning. Si distinguano in particolare classificazione e regressione}
 	I principali paradigmi sono:
	\begin{enumerate}
		\item Supervised learning:\\
			lo scopo di questo paradigma è quello di dare la risposta corretta ad ogni esempio, ovvero, dati gli x$^i$ e gli y$^i$, trovare la funzione $h\approx f:X \to Y$\\
			Si chiama "supervised" perché il supervisore fornisce i valori di $h(\cdot)$ alle varie istanze $x^i$\\
			Può essere utilizzato sia per i casi di classificazione (valori discreti per capire se un dato appartiene ad una classe o ad un'altra) e per i casi di regressione (trovare la vera funzione che mappa correttamente gli input e gli output)
		\item Unsupervised learning:\\
			in questo paradigma non esistono supervisori, l'obiettivo di questo paradigma è trovare la regolarità oppure i pattern, dunque dati esempio $x^i$ trovare le regolarità presenti in tutto il dominio
		\item Reinforncement learning:\\
			in questo paradigma si ha un agente che può essere in uno stato "s", esegue azione "a" (tra quelle ammissibili per lo stato s) ed opera in un ambiente "e" che in risposta all'azione "a" e lo stato "s" ritorna un nuovo stato e una ricompensa (positiva, negativa o neutrale)\\
			L'obiettivo dell'agente è massimizzare la funzione delle ricompense
	\end{enumerate}
	Altri paradigmi possono essere:
	\begin{enumerate}
		\item Weak-supervised learning:\\
			è una branca dell'apprendimento automatico in cui vengono utilizzati
			dati non organizzati o imprecisi per fornire indicazioni per etichettare una grande quantità di
			dati non supervisionati in modo che possa essere utilizzata
			nell'apprendimento automatico o nell'apprendimento supervisionato
		\item Self-supervised learning:\\
			consente ai sistemi di intelligenza artificiale di apprendere da ordini di
			grandezza maggiori di dati, il che è importante per riconoscere e comprendere modelli di
			rappresentazioni del mondo più sottili e meno comuni
		\item Federated learning:\\
			(noto anche come apprendimento collaborativo) è una tecnica di
			apprendimento automatico che addestra un algoritmo su più dispositivi edge decentralizzati o
			server che conservano campioni di dati locali, senza scambiarli
		\item Deep Learning:\\
			la cui traduzione letterale significa apprendimento profondo, è una
			sottocategoria del Machine Learning e indica quella branca dell’intelligenza artificiale che fa riferimento agli algoritmi
			ispirati alla struttura e alla funzione del cervello chiamate reti neurali artificiali
	\end{enumerate}
	
	
	
	\subsection{Si descrivano nel modo più accurato possibile i concetti di bias e variance, il loro rapporto e come nella pratica possano essere affrontati e ridotti i problemi. A tal fine si riportino anche esempi concreti che aiutino a chiarire i diversi aspetti coinvolti}
	Il bias rappresenta l'errore sistematico introdotto da un modello nel semplificare troppo il problema, portando a una rappresentazione inaccurata dei dati\\
	L'inductive bias rappresenta l'insieme di ipotesi che un algoritmo assume sulla funzione target per poter generalizzare dai dati di training ai dati nuovi\\
	Infatti, nel caso della regressione lineare, l'inductive bias principale è che l'insieme di input e output può essere rappresentato come una funzione lineare, oppure, nel caso dei "Nearest Neighbors", si assume che la maggior parte dei casi in uno spazio ravvicinato faccia parte della stessa classe \\
	Ci sono esattamente due tipi:
	\begin{enumerate}
		\item restriction: limita lo spazio delle ipotesi
		\item preference: impone l'ordine delle preferenze
	\end{enumerate}
	Il bias alto causa all'algoritmo la mancanza di relazioni rilevanti fra feature e target, ad esempio un modello di regressione lineare ha un bias alto se i dati reali seguono una relazione quadratica: la sua rigidità lo porterà a fare errori sistematici\\
	La variance misura quanto il modello è sensibile alle variazioni nei dati di training\\
	Un modello ha varianza alta quando è molto sensibile ai cambiamenti nei dati di training: piccole variazioni possono portare a grandi differenze nelle predizioni\\
	Questo accade quando il modello è troppo complesso e tende a sovradattarsi (overfitting)\\
	Il loro rapporto principale porta al bias-variance tradeoff, ovvero ciò che rappresenta il rapporto tra bias e varianza: un modello con bias alto semplifica troppo il problema (underfitting), mentre un modello con varianza alta si adatta troppo ai dati di training (overfitting)\\
	L'obiettivo è trovare un equilibrio tra i due per minimizzare l'errore totale\\
	La cosa migliore per un algoritmo sarebbe avere bias basso, varianza bassa\\
	Alcuni metodi pratici per bilanciare bias e varianza:\\
	1) Ridurre il bias: usare modelli più complessi (es. passare da regressione lineare a polinomiale)\\
	2) Ridurre la varianza: usare più dati di training\\
	Le learning curve (errori su training e validation) aiutano a diagnosticare questi problemi:\\
	bias elevato si manifesta con errori alti su entrambi, mentre la variance si riconosce da un errore basso sul training ma alto sulla validation\\

	
	
	\subsection{Si descriva in modo dettagliato il
		modello di logistic regression (con regolarizzazione), le sue principali caratteristiche ed il contributo dei diversi elementi presenti nella funzione
		di costo. Si riporti infine una descrizione accurata delle differenze e degli elementi in comune di tale modello rispetto
		ad un semplice classificatore lineare, anche mediante esempi qualitativi. Infine, si descriva chiaramente la procedura di addestramento
		mediante l’applicazione di gradient descent}
		
	\subsubsection{Introduzione e motivazione}
	La \textbf{logistic regression} è un modello di classificazione che evita i problemi della regressione lineare applicata a task di classificazione:
	\begin{itemize}
		\item La regressione lineare è sensibile a \textit{outliers} e può produrre valori al di fuori dell'intervallo $[0,1]$.
		\item La logistic regression garantisce invece output probabilistici in $[0,1]$ tramite la \textbf{funzione sigmoide}\\
		Comunque il suo scopo è quello di trovare una funzione $h \approx f: X \to Y$ dove Y è un insieme di classi (2) e non in $\mathcal{R}$ come nella regressione lineare
	\end{itemize}
	
	\subsubsection{Formulazione del modello}
	L'ipotesi è data da:
	\[
	h_\theta(\mathbf{x}) = g(\theta^T \mathbf{x}) = \frac{1}{1 + e^{-\theta^T \mathbf{x}}}, \quad \text{dove } g(z) = \frac{1}{1+e^{-z}}
	\]
	\subsubsection{Interpretazione probabilistica}
	\begin{itemize}
		\item $h_\theta(\mathbf{x}) = P(y=1 \mid \mathbf{x}; \theta)$
		\item Per due classi: $P(y=0 \mid \mathbf{x}; \theta) = 1 - h_\theta(\mathbf{x})$
		\item \textbf{Limite di decisione}: 
		\[
		y = 
		\begin{cases}
			1 & \text{se } h_\theta(\mathbf{x}) \geq 0.5 \\
			0 & \text{se } h_\theta(\mathbf{x}) < 0.5
		\end{cases}
		\]
	\end{itemize}
	
	\subsubsection{Funzione di costo con regolarizzazione}
	La \textbf{cross-entropy loss} con regolarizzazione:
	\[
	J(\theta) = -\frac{1}{m} \sum_{i=1}^m \left[ y^{(i)} \log(h_\theta(\mathbf{x}^{(i)})) + (1-y^{(i)}) \log(1 - h_\theta(\mathbf{x}^{(i)})) \right] + \frac{\lambda}{2m} \sum_{j=1}^n \theta_j^2
	\]
	Questa funzione serve per quando ci sono tantissime feature, però serve mantenere le due più importanti e le altre renderle molto piccole, dove quest'ultime sono molto poco influenti però danno il loro contributo per trovare y\\
	Come in tutti gli altri casi, bisogna trovare $min(J(\Theta))$
	
	\subsubsection{Confronto con il classificatore lineare}
	\begin{itemize}
		\item \textbf{Similarità}:
		\begin{itemize}
			\item Entrambi usano una funzione lineare $\theta^T \mathbf{x}$.
			\item Utilizzano ottimizzazione tramite gradient descent.
		\end{itemize}
		\item \textbf{Differenze}:
		\begin{itemize}
			\item La regressione lineare minimizza l'MSE, la logistic regression la cross-entropy.
			\item La logistic regression produce probabilità, mentre il classificatore lineare valori continui.
			\item \textit{Inductive bias}: La logistic regression assume una relazione logistica tra features e classi.
		\end{itemize}
		\item \textbf{Esempio}: Per dati con outlier, la logistic regression è robusta mentre la regressione lineare può produrre un iperpiano distorto.
	\end{itemize}
	
	L'addestramento con gradient descent usa la procedura iterativa aggiornando i parametri $\theta$ come:
	\[
	\theta_j := \theta_j - \alpha \frac{\partial J(\theta)}{\partial \theta_j} = \theta_j - \frac{\alpha}{m} (\sum_{i=1}^{m} (h_{\theta}(x^{(i)}) - y^i)x_j^i + \lambda \theta_j)
	\]
	dove $\alpha$ è il learning rate\\
	L'algoritmo termina quando le derivate sono vicine a zero (punto di minimo), si può notare che è molto simile al procedimento di "gradient descent" della regressione lineare
	
	
	
	\subsection{Si descriva dettagliatamente la procedura di crossvalidation, motivandone scopo ed utilità,
		e fornendo una chiara descrizione della (corretta) procedura di addestramento di un
		qualunque sistema di machine learning. Si descrivano inoltre i concetti di true error ed
		empirical error e se ne evidenzino le relazioni con la procedura di cross-validation}
	La \textbf{cross-validation} è una tecnica fondamentale per la selezione e valutazione di modelli nel machine learning, con tre obiettivi chiave:
	\begin{itemize}
		\item Stimare le prestazioni del modello su dati non visti (\textit{generalizzazione})
		\item Ottimizzare gli iperparametri evitando l'overfitting
		\item Utilizzare efficientemente dataset limitati
	\end{itemize}
	Un'implementazione particolarmente diffusa è la \textbf{k-fold cross-validation}, dove il dataset viene suddiviso in $k$ sottoinsiemi (\textit{fold}) di uguale dimensione, utilizzati iterativamente per training e validazione\\
	Facendo un esempio a 3 set tipicamente si divide in:\\
	il più grande, detto training set, da 70\%, gli altri due da 15\%, detti validation set, che servono a valutare la retta (o in generale la funzione h$\approx$f: X$\to$Y) ottenuta dallo studio sul validation set\\
	Questo serve a mantenere un learning molto più approfondito rispetto allo studio sul 100\% del dataset, dove si rischierebbe di andare incontro all'overfitting\\
	Inoltre, c'è la procedura di "leave-one-out", che è un particolare caso di k-fold dove si divide il dataset in k set di dimensione uguale disgiunti a due a due e volta per volta dei sottoinsiemi vengono considerati validation set e training set fra di loro\\
	Relativamente a questa procedura, si possono associare i concetti di "true error" ed "empirical error":\\
	\begin{enumerate}
		\item il "true error" di una particolare ipotesi \textit{h} che approssima \textit{f} rispetto al dataset D è la probabilità che
			  \textit{h} non classificherà correttamente un'istanza di D\\
			  \[ error_D(h) = Prob_{x \in D}\{ f(x) \neq h(x) \} \]
		\item il "empirical error" dell'ipotesi \textit{h} è dato da tutti i casi in cui h sbaglierà\\
			  \[ error_T(h) = \frac{\#\{ (x,f(x)) \in T : f(x) \neq h(x) \}, T \subseteq D}{\text{instances of T}} \]
	\end{enumerate}
	Questi due errori diversi servono ad identificare il bias (assunzione troppo "forte/debole" che facciamo sulla funzione target) e la variance (la dipendenza dai dati)\\
	Inoltre, serve ad identificare l'overfitting, il quale avviene quando:\\
	\[ \text{data } h \in H, \text{ overfitting avviene se } \exists h' \in H : error_T(h) < error_T(h') \text{ ma } error_D(h) > error_D (h') \]
	
	
	
	\subsection{Si descriva dettagliatamente la procedura di model selection (aiutandosi con un esempio
		concreto) e si fornisca una chiara giustificazione teorica/concettuale a tale procedura}
	La model selection è il processo grazie al quale si può scegliere il miglior modello/iperparametri dato un problema\\
	Uno dei tanti metodi per trovarli è la cross-validation, che fa parte della famiglia della model selection\\
	Facendo un esempio di model selection per la selezione di immagini, ipotizzo che si voglia cercare il miglior modello per la selezione di immagini (classificazione binaria) che identifica se un immagine rappresenta un cane oppure no\\
	Nell'esempio selezionato, la predizione si può classificare come:
	\begin{itemize}
		\item vero positivo (TP)
		\item falso positivo (FP)
		\item vero negativo (TN)
		\item falso negativo (FN)
	\end{itemize}
	L'accuratezza del modello è data da $\frac{TP+TN}{P+N} = \frac{\text{all correct}}{\text{all instances}}$, invece la performance si pasa su:
	\begin{itemize}
		\item precision: $\frac{TP}{TP+FP}$: frazione delle istanze recuperate che sono rilevanti
		\item recall: $\frac{TP}{TP+FN}$: frazione delle istanze rilevanti che sono recuperate
	\end{itemize}
	Il rapporto corretto, che purtroppo non può essere sempre preciso al 100\%, sopratutto su animali rarissimi di cui girano poche foto su Internet, si valuta tramite gli integrali fra le curve di precision e recall\\
	Si possono però stimare bias e variance, infatti:
	le learning curve rappresentano:\\
	bias elevato se manifesta con errori alti su entrambi, la variance si riconosce da un errore basso sul training ma alto sulla validation\\
	Inoltre, spesso si computa Prec@k e Rec@k nei top k risultati\\
	Infine, un'altra misurazione è l'average precision (AP) che è una misura che combina recall e precision per risultati ordinati, quindi restituisce ciò che ha una precisione maggiore, che si calcola sommando la precisione quando si trova un True positive identificato realmente
	\[ AP=\frac{\sum_{k=1}^{n}P(k) \cdot rel(k)}{\text{all groundtruth instances}} \]
	Quest'ultima sommatoria serve a trovare la corretta sequenza in cui i primi k risultati corretti compaiono, infatti se AP=1 vuol dire che tutti i risultati antecedenti al 1° sbagliato saranno corretti\\
	Grazie a quest'ultima, nel caso del nostro esempio ma anche per altri modelli, si può trovare chiaramente qual è il modello migliore per un determinato problema
	
	
	
	
	\subsection{Cosa si intende per “one learning algorithm hypothesis” e come tale ipotesi si relaziona
		con le reti neurali artificiali? Si fornisca inoltre una descrizione esaustiva degli elementi/ingredienti
		principali che permettono la definizione di una rete neurale multistrato}
	"One learning algorithm hypothesis" si riferisce all'idea che esista un unico algoritmo di apprendimento che possa essere utilizzato per affrontare una vasta gamma di compiti, indipendentemente dalla specificità dei dati o della struttura del problema\\
	In altre parole, questa ipotesi suggerisce che un solo algoritmo di apprendimento, se opportunamente configurato e allenato, possa essere in grado di risolvere una varietà di problemi, inclusi quelli complessi come la classificazione e la previsione\\
	Nell'ambito delle reti neurali artificiali, questa ipotesi si relaziona al fatto che, nonostante le reti neurali possano variare in complessità e architettura, un singolo algoritmo di apprendimento, come la retropropagazione (backpropagation), può essere utilizzato per addestrare una rete neurale su vari tipi di dati, risolvendo compiti complessi come la classificazione o la regressione
	
	
	
	\subsection{Spiegare in dettaglio gli elementi fondamentali del perceptron e, più in generale, delle reti
		neurali, illustrando chiaramente la fase di feed-forward\\
		Si riporti inoltre una breve descrizione di come tale modello possa essere esteso
		mediante la realizzazione di un’architettura a più strati, fornendo un esempio che evidenzi
		le differenze/vantaggi di tale architettura
		Si riporti inoltre un esempio di rete neurale per la realizzazione di un semplice operatore
		logico OR, AND, XOR, XNOR e NAND}
	Il perceptron è un modello semplice:\\
	un neurone che fa una trasformazione per combinazione lineare con una "funzione di attivazione" definita come f($\Theta^T$x)\\
	Questa funzione di attivazione può essere la sigmoide ($\frac{1}{1+e^-z}$), il tanh (tangente iperbolico), la ReLU ($max(0,x)$), Leaky ReLU: max(0.1x, x), Maxout: max($w_1^Tx+b_1, w_2^Tx+b_2$), ELU : x quando x$>=0$, $\alpha$($e^x$-1) quando $x<0$\\
	Ogni neurone implementa da solo la logistic regression (nel caso in cui usi la funzione sigmoide)\\
	Esso prende in input quindi tutti i pesi $\Theta$ e definisce, tramite la funzione scelta fra quelle descritte precedentemente, un output\\
	In questo caso, i pesi $\Theta$ sono definiti dalla fase di feed-forward dello strato precedente\\
	Questi percepetron sono la base per la costruzione di una rete neurale multistrato, infatti essa è formata da più strati, divisi in 3 macrocategorie:
	\begin{enumerate}
		\item input layer
		\item hidden layer ($>=1$)
		\item output layer
	\end{enumerate}
	Il passaggio da uno strato all'altro (feed-forward) si misura come
	\[
	a_i^{(l)} = f\left(\sum_{k=1}^{n^{(l-1)}} \Theta_{ik}^{(l)} a_k^{(l-1)} + b_i^{(l)}\right)
	\]
	
	\textbf{Spiegazione dei simboli:}
	\begin{itemize}
		\item $a_i^{(l)}$: Attivazione del neurone $i$-esimo nel layer $l$.
		\item $\Theta_{ik}^{(l)}$: Peso tra il neurone $k$ del layer $l-1$ e il neurone $i$ del layer $l$.
		\item $b_i^{(l)}$: \textbf{Bias} (spesso incluso come $\Theta_{i0}^{(l)}$ con $a_0^{(l-1)} = 1$).
		\item $f$: Funzione di attivazione (ReLU, sigmoide, ecc.).
		\item $n^{(l-1)}$: Numero di neuroni nel layer precedente.
	\end{itemize}
	Mentre i perceptron riescono a fare unicamente una combinazione lineare, il collegamento fra più perceptron permette architetture molto più complesse che permettono di definire operazioni più difficili che un unico perceptron non riuscirebbe a fare\\
	Questo permette la costruzione di un sistema che offre nettamente più vantaggi rispetto al perceptron singolo, ma anche un numero di collegamenti decisamente più elevato (per le reti che abbiamo visto fino ad ora ogni perceptron è collegato a tutti i perceptron dello stato successivo, portando la complessità computazionale ad essere molto elevata per sistemi formati da centinaia se non migliaia di strati)\\
	Queste reti neurali seguono una fase di feed-forward, ovvero una propagazione in avanti dei dati dove si porta ogni dato da uno strato a quello successivo\\
	L'architettura multistrato e la propagazione feed-forward permettono la costruzione di operatori logici OR, AND, XOR (XNOR $\to$ NOT AND NOT) e NAND (NOT AND)\\
	La differenza principale fra questi operatori logici è che OR e AND sono operatori logici linearmente separabili, quindi non richiedono strati nascosti, invece XOR e NAND non sono linearmente separabili e richiedono degli strati nascosti per arrivare al corretto risultato\\
	\textbf{Gli esempi sono nelle slide/negli appunti}	
		
	
	\newpage
	\section{Seconda parte}
	\subsection{Spiegare in dettaglio gli elementi fondamentali di SVM; in particolare:
		1) la sua interpretazione geometrica, 2) la funzione di costo, 3) le differenze/similitudini con altri modelli di ML.
		Infine, si introduca brevemente l’estensione di SVM basata sul kernel trick}
	g
	
	\subsection{Spiegare in dettaglio gli elementi fondamentali del perceptron e, più in generale, delle reti
		neurali. Si riporti inoltre una breve descrizione di come tale modello possa essere esteso
		mediante la realizzazione di un’architettura multistrato, fornendo un esempio che evidenzi
		le differenze ed i vantaggi di tale architettura}
	g
	
	\subsection{Spiegare in dettaglio gli elementi fondamentali del perceptron e, più in generale, delle reti
		neurali multistrato, illustrando chiaramente le due fasi di feedforward e backpropagation.}
	
	
	
	\subsection{Si descrivano nel modo più accurato possibile gli alberi di decisione, i loro vantaggi e svantaggi rispetto ad altri modelli (ad es. reti neurali)
		 e si evidenzi il principale inductive bias di tale algoritmo. Si fornisca inoltre un semplice esempio di albero di decisione, successivamente lo XOR. Infine, si illustri brevemente l’estensione di tale modello attraverso random forest}
	g
	
	
	\subsection{Si descriva in modo accurato l’algoritmo k-NN, illustrando il ruolo dei principali iperparametri, i vantaggi e le debolezze del modello nei 
		confronti di altri algoritmi affrontati nel corso, e si evidenzi il principale inductive bias di tale algoritmo}
	g
	
	
	
\end{document}