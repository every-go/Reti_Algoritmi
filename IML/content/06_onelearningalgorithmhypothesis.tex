\section{Cosa si intende per “one learning algorithm hypothesis” e come tale ipotesi si relaziona
		con le reti neurali artificiali? Si fornisca inoltre una descrizione esaustiva degli elementi/ingredienti
		principali che permettono la definizione di una rete neurale multistrato}
   
	"One learning algorithm hypothesis" si riferisce all'idea che esista un unico algoritmo di apprendimento che possa essere utilizzato per affrontare una vasta gamma di compiti, indipendentemente dalla specificità dei dati o della struttura del problema\\
	In altre parole, questa ipotesi suggerisce che un solo algoritmo di apprendimento, se opportunamente configurato e allenato, possa essere in grado di risolvere una varietà di problemi, inclusi quelli complessi come la classificazione e la previsione\\
	Nell'ambito delle reti neurali artificiali, questa ipotesi si relaziona al fatto che, nonostante le reti neurali possano variare in complessità e architettura, un singolo algoritmo di apprendimento, come la retropropagazione (backpropagation), può essere utilizzato per addestrare una rete neurale su vari tipi di dati, risolvendo compiti complessi come la classificazione o la regressione