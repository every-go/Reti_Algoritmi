\documentclass[12pt,oneside,a4paper]{article}
\usepackage{bussproofs}
\pagenumbering{arabic}
\title{Es 2.4 simulazione appello}
\author{Matteo Mazzaretto}
\date{23 gennaio 2023}
\begin{document}
\maketitle
\begin{center}{\bf 2.4}\end{center}
"Esiste uno a cui nessuno scrive se e solo se lui scrive a tutti oppure non esiste uno che, se
lui scrive a qualcuno tutti scrivono a qualcuno.”\newline
Si consiglia di usare:\newline
S(x, y)=x scrive a y\newline
La traduzione è:
\begin{center}$\vdash$$\exists$x($\neg$$\exists$yS(y,x)$\leftrightarrow$$\forall$yS(x,y))$\vee$($\neg$$\exists$x($\exists$zS(x,z)$\to$$\forall$y$\exists$zS(y,z)))\end{center}
Durante la correzione della simulazione, ci è stato esposto che questo sequente è un paradosso\newline
In realtà, si tratta di opinione, e ora lo dimostro\newline
\section{Formula negata}
Innanzitutto, dato che ci è stato detto che doveva essere paradosso, derivo immediatamente la negazione
\begin{prooftree}
\RightLabel{\(\)}
\noLine
\AxiomC{}
\UnaryInfC{problema}
\RightLabel{\(\)}
\noLine
\UnaryInfC{$\exists$x($\neg$$\exists$yS(y,x)$\leftrightarrow$$\forall$yS(x,y))$\vdash$}
\RightLabel{\(\)}
\noLine
\AxiomC{}
\UnaryInfC{ax-id}
\RightLabel{\(\)}
\noLine
\UnaryInfC{S(x,z),S(x1,x2)$\vdash$S(x1,x2),$\forall$y$\exists$zS(y,z)}
\RightLabel{$\exists$-Dv}
\UnaryInfC{S(x,z),S(x1,x2)$\vdash$$\exists$zS(x1,z),$\forall$y$\exists$zS(y,z)}
\RightLabel{SCdx}
\UnaryInfC{S(x,z),S(x1,x2)$\vdash$$\forall$y$\exists$zS(y,z),$\exists$zS(x1,z)}
\RightLabel{$\exists$-S x2$\notin$VL}
\UnaryInfC{S(x,z),$\exists$zS(x1,z)$\vdash$$\forall$y$\exists$zS(y,z),$\exists$zS(x1,z)}
\RightLabel{$\to$-D}
\UnaryInfC{S(x,z)$\vdash$$\exists$zS(x1,z)$\to$$\forall$y$\exists$zS(y,z),$\exists$zS(x1,z)}
\RightLabel{$\exists$-Dv}
\UnaryInfC{S(x,z)$\vdash$$\exists$x($\exists$zS(x,z)$\to$$\forall$y$\exists$zS(y,z)),$\exists$zS(x1,z)}
\RightLabel{Scdx}
\UnaryInfC{S(x,z)$\vdash$$\exists$zS(x1,z),$\exists$x($\exists$zS(x,z)$\to$$\forall$y$\exists$zS(y,z))}
\RightLabel{$\exists$-S z$\notin$VL}
\UnaryInfC{$\exists$zS(x,z)$\vdash$$\exists$zS(x1,z),$\exists$x($\exists$zS(x,z)$\to$$\forall$y$\exists$zS(y,z))}
\RightLabel{$\forall$-D x1$\notin$VL}
\UnaryInfC{$\exists$zS(x,z)$\vdash$$\forall$y$\exists$zS(y,z),$\exists$x($\exists$zS(x,z)$\to$$\forall$y$\exists$zS(y,z))}
\RightLabel{$\to$-D}
\UnaryInfC{$\vdash$$\exists$zS(x,z)$\to$$\forall$y$\exists$zS(y,z),$\exists$x($\exists$zS(x,z)$\to$$\forall$y$\exists$zS(y,z))}
\RightLabel{$\exists$-D}
\UnaryInfC{$\vdash$$\exists$x($\exists$zS(x,z)$\to$$\forall$y$\exists$zS(y,z))}
\RightLabel{$\neg$-S}
\UnaryInfC{$\neg$$\exists$x($\exists$zS(x,z)$\to$$\forall$y$\exists$zS(y,z))$\vdash$}
\RightLabel{$\vee$-S}
\BinaryInfC{$\exists$x($\neg$$\exists$yS(y,x)$\leftrightarrow$$\forall$yS(x,y))$\vee$($\neg$$\exists$x($\exists$zS(x,z)$\to$$\forall$y$\exists$zS(y,z))))$\vdash$}
\RightLabel{$\neg$-D}
\UnaryInfC{$\vdash$$\neg$($\exists$x($\neg$$\exists$yS(y,x)$\leftrightarrow$$\forall$yS(x,y))$\vee$($\neg$$\exists$x($\exists$zS(x,z)$\to$$\forall$y$\exists$zS(y,z))))}
\end{prooftree}
Grazie all'applicazione della regola $\vee$-D siamo riusciti a separare le due formule che compongono il sequente di partenza\newline
Abbiamo dimostrato che $\neg$$\exists$x($\exists$zS(x,z)$\to$$\forall$y$\exists$zS(y,z)) è un paradosso, in quanto in logica se la derivazione della negazione della formula è tautologia allora il sequente di partenza è paradosso\newline
Il problema è $\exists$x($\neg$$\exists$yS(y,x)$\leftrightarrow$$\forall$yS(x,y)), il quale causa il risultato di opinione al sequente di partenza
\section{Dimostrazione del perché è un problema}
Innanzitutto, dato che abbiamo scoperto che la seconda parte del sequente di partenza è paradosso, procedo a derivare unicamente la prima parte col metodo "classico", ovvero considerando $\exists$x($\neg$$\exists$yS(y,x)$\leftrightarrow$$\forall$yS(x,y))\newline
Nelle parti successive verrà indicato come "sequente fr"
\subsection{Derivazione di sequente fr}
Derivo il sequente fr per dimostrare che è possibile costruire un contromodello
\begin{prooftree}
\RightLabel{\(\)}
\noLine
\AxiomC{}
\UnaryInfC{continua sotto}
\RightLabel{\(\)}
\noLine
\UnaryInfC{$\vdash$$\neg$$\exists$yS(y,x)$\to$$\forall$yS(x,y),$\exists$x($\neg$$\exists$yS(y,x)$\leftrightarrow$$\forall$yS(x,y))}
\RightLabel{\(\)}
\noLine
\AxiomC{}
\UnaryInfC{$\vdash$$\forall$yS(x,y)$\to$$\neg$$\exists$yS(y,x),$\exists$x($\neg$$\exists$yS(y,x)$\leftrightarrow$$\forall$yS(x,y))}
\RightLabel{$\&$-D}
\BinaryInfC{$\vdash$$\neg$$\exists$yS(y,x)$\leftrightarrow$$\forall$yS(x,y),$\exists$x($\neg$$\exists$yS(y,x)$\leftrightarrow$$\forall$yS(x,y))}
\RightLabel{$\exists$-D}
\UnaryInfC{$\vdash$$\exists$x($\neg$$\exists$yS(y,x)$\leftrightarrow$$\forall$yS(x,y))}
\end{prooftree}
Non si vede nel pdf, ma ho applicato la regola $\&$-D per arrivare alle due foglie\newline
Continuo la derivazione della foglia di sinistra
\begin{prooftree}
\RightLabel{\(\)}
\noLine
\AxiomC{}
\UnaryInfC{loop}
\RightLabel{\(\)}
\noLine
\UnaryInfC{$\vdash$S(z,x),$\exists$yS(y,x),S(x,z),$\exists$x($\neg$$\exists$yS(y,x)$\leftrightarrow$$\forall$yS(x,y))}
\RightLabel{$\exists$-D}
\UnaryInfC{$\vdash$$\exists$yS(y,x),S(x,z),$\exists$x($\neg$$\exists$yS(y,x)$\leftrightarrow$$\forall$yS(x,y))}
\RightLabel{SCdx}
\UnaryInfC{$\vdash$S(x,z),$\exists$yS(y,x),$\exists$x($\neg$$\exists$yS(y,x)$\leftrightarrow$$\forall$yS(x,y))}
\RightLabel{$\forall$-D z$\notin$VL}
\UnaryInfC{$\vdash$$\forall$yS(x,y),$\exists$yS(y,x),$\exists$x($\neg$$\exists$yS(y,x)$\leftrightarrow$$\forall$yS(x,y))}
\RightLabel{SCdx}
\UnaryInfC{$\vdash$$\exists$yS(y,x),$\forall$yS(x,y),$\exists$x($\neg$$\exists$yS(y,x)$\leftrightarrow$$\forall$yS(x,y))}
\RightLabel{$\neg$-S}
\UnaryInfC{$\neg$$\exists$yS(y,x)$\vdash$$\forall$yS(x,y),$\exists$x($\neg$$\exists$yS(y,x)$\leftrightarrow$$\forall$yS(x,y))}
\RightLabel{$\to$-D}
\UnaryInfC{$\vdash$$\neg$$\exists$yS(y,x)$\to$$\forall$yS(x,y),$\exists$x($\neg$$\exists$yS(y,x)$\leftrightarrow$$\forall$yS(x,y))}
\end{prooftree}
Ho trovato un loop perché, indipendetemente da quante volte applico la regola dell'$\exists$-D sulla formula, nel ramo di sinistra sviluppato dopo l'applicazione della regola $\&$-D non si dimostra niente in quanto tutte le formule (fr1, fr2,...,fri) sono a destra del sequente e non è possibile dimostrare il $tt$ non avendo neanche a nostra disposizione dei predicati con la negazione\newline
Quindi trovo un contromodello\newline
D=\{Minni, Topolino\}\newline
(S(z,x))$^{D}$(Minni,Topolino)=0\newline
(S(x,z))$^{D}$(Topolino,Minni)=0\newline
Questo contromodello risulta falso nel sequente di partenza $tt$$\vdash$$\exists$x($\neg$$\exists$yS(y,x)$\leftrightarrow$$\forall$yS(x,y)) perché: \newline
$\exists$yS(y,x)$^{D}$=0  per ogni d $\in$ D in quanto né Minni scrive a Topolino né Topolino scrive a Minni\newline
quindi $\neg$$\exists$yS(y,x)$^{D}$=1 per ogni d $\in$ D\newline
inoltre $\forall$yS(x,y)$^{D}$=0 per ogni d $\in$ D perché basta che uno degli due elementi del dominio non scriva all'altro, e in questo caso non si scrivono a vicenda\newline
Quindi questo contro contromodello è valido perché:\newline
$tt$$\vdash$$\exists$x($\neg$$\exists$yS(y,x)$\leftrightarrow$$\forall$yS(x,y))=\newline
$tt$$\vdash$$\exists$x(1$\leftrightarrow$0)=\newline
$tt$$\vdash$$\exists$x(0)=\newline
$tt$$\vdash$0=0 per definizione di implica\newline
Ora che abbiamo trovato un contromodello, posso dimostrare che è opinione tramite la derivazione della negazione del sequente fr
\section{Derivazione negazione}
\begin{prooftree}
\RightLabel{\(\)}
\noLine
\AxiomC{}
\UnaryInfC{$\neg$$\exists$yS(y,w)$\to$$\forall$yS(w,y)$\vdash$$\forall$yS(w,y)}
\RightLabel{\(\)}
\noLine
\AxiomC{}
\UnaryInfC{$\neg$$\exists$yS(y,w)$\to$$\forall$yS(w,y),$\neg$$\exists$yS(y,w)$\vdash$}
\RightLabel{$\to$-S}
\BinaryInfC{$\neg$$\exists$yS(y,w)$\to$$\forall$yS(w,y),$\forall$yS(w,y)$\to$$\neg$$\exists$yS(y,w)$\vdash$}
\RightLabel{$\&$-S}
\UnaryInfC{$\neg$$\exists$yS(y,w)$\leftrightarrow$$\forall$yS(w,y)$\vdash$}
\RightLabel{$\exists$-S w$\notin$VL}
\UnaryInfC{$\exists$x($\neg$$\exists$yS(y,x)$\leftrightarrow$$\forall$yS(x,y))$\vdash$}
\RightLabel{$\neg$-D}
\UnaryInfC{$\vdash$$\neg$($\exists$x($\neg$$\exists$yS(y,x)$\leftrightarrow$$\forall$yS(x,y)))}
\end{prooftree}
\subsection{Derivazione foglia destra: $\neg$$\exists$yS(y,w)$\to$$\forall$yS(w,y),$\neg$$\exists$yS(y,w)$\vdash$}
\begin{prooftree}
\RightLabel{\(\)}
\noLine
\AxiomC{}
\UnaryInfC{ax-id}
\RightLabel{\(\)}
\noLine
\UnaryInfC{S(x,w)$\vdash$S(x,w)}
\RightLabel{$\exists$-Dv}
\UnaryInfC{S(x,w)$\vdash$$\exists$yS(y,w)}
\RightLabel{$\exists$-Sx$\notin$VL}
\UnaryInfC{$\exists$yS(y,w)$\vdash$$\exists$yS(y,w)}
\RightLabel{$\neg$-D}
\UnaryInfC{$\vdash$$\neg$$\exists$yS(y,w),$\exists$yS(y,w)}
\RightLabel{\(\)}
\noLine
\AxiomC{}
\UnaryInfC{ax-id}
\RightLabel{\(\)}
\noLine
\UnaryInfC{S(w,w)$\vdash$S(w,w)}
\RightLabel{$\exists$-Dv}
\UnaryInfC{S(w,w)$\vdash$$\exists$yS(y,w)}
\RightLabel{$\forall$-SV}
\UnaryInfC{$\forall$yS(w,y)$\vdash$$\exists$yS(y,w)}
\RightLabel{$\to$-S}
\BinaryInfC{$\neg$$\exists$yS(y,w)$\to$$\forall$yS(w,y)$\vdash$$\exists$yS(y,w)}
\RightLabel{$\neg$-D}
\UnaryInfC{$\neg$$\exists$yS(y,w)$\to$$\forall$yS(w,y),$\neg$$\exists$yS(y,w)$\vdash$}
\end{prooftree}
La derivazione di questa foglia non causa problemi; deriviamo l'altra
\subsection{Derivazione foglia sinistra: $\neg$$\exists$yS(y,w)$\to$$\forall$yS(w,y)$\vdash$$\forall$yS(w,y)}
\begin{prooftree}
\RightLabel{\(\)}
\noLine
\AxiomC{}
\UnaryInfC{problema}
\RightLabel{\(\)}
\noLine
\UnaryInfC{S(x,w)$\vdash$S(w,z)}
\RightLabel{$\exists$-Sx$\notin$VL}
\UnaryInfC{$\exists$yS(y,w)$\vdash$S(w,z)}
\RightLabel{$\forall$-D z$\notin$VL}
\UnaryInfC{$\exists$yS(y,w)$\vdash$$\forall$yS(w,y)}
\RightLabel{$\neg$-D}
\UnaryInfC{$\vdash$$\neg$$\exists$yS(y,w),$\forall$yS(w,y)}
\RightLabel{\(\)}
\noLine
\AxiomC{}
\UnaryInfC{ax-id}
\RightLabel{\(\)}
\noLine
\UnaryInfC{S(w,z)$\vdash$S(w,z)}
\RightLabel{$\forall$-Sv}
\UnaryInfC{$\forall$yS(w,y)$\vdash$S(w,z)}
\RightLabel{$\forall$-D z$\notin$VL}
\UnaryInfC{$\forall$yS(w,y)$\vdash$$\forall$yS(w,y)}
\RightLabel{$\to$-S}
\BinaryInfC{$\neg$$\exists$yS(y,w)$\to$$\forall$yS(w,y)$\vdash$$\forall$yS(w,y)}
\end{prooftree}
Una foglia non si può dimostrare, dimostro un contromodello\newline
D=\{Minni,Topolino\}\newline
(S(x,w))$^{D}$(Minni,Topolino)=1\newline
(S(w,z))$^{D}$(Topolino,Minni)=0\newline
Questo contromodello risulta falso nel sequente fr perché: \newline
$\exists$yS(y,x)$^{D}$=1  perché è verificato nel contromodello il fatto che Minni scrive a Topolino\newline
Quindi $\neg$$\exists$yS(y,x)$^{D}$=0\newline
Inoltre $\forall$yS(x,y)$^{D}$=0 perché non tutti gli elementi del dominio scrivono all'altro, come nel caso di Topolino che $non$ scrive a Minni\newline
Quindi questo contro contromodello è valido perché:\newline
$tt$$\vdash$$\neg$($\exists$x($\neg$$\exists$yS(y,x)$\leftrightarrow$$\forall$yS(x,y)))=\newline
$tt$$\vdash$$\neg$($\exists$x(0$\leftrightarrow$0))=\newline
$tt$$\vdash$$\neg$($\exists$x(1))=\newline
$tt$$\vdash$$\neg$(1)=\newline
$tt$$\vdash$0=0 per definizione di implica\newline
\subsection{Conclusione}
Avendo trovato un contromodello sia per il sequente fr sia per la sua derivazione 'classica' che per la sua negazione, si può affermare che il sequente fr è opinione\newline
Essendo il sequente fr opinione anche il sequente di partenza\newline
$\vdash$$\exists$x($\neg$$\exists$yS(y,x)$\leftrightarrow$$\forall$yS(x,y))$\vee$($\neg$$\exists$x($\exists$zS(x,z)$\to$$\forall$y$\exists$zS(y,z))) lo è
\end{document}