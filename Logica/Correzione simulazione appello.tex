\documentclass[12pt,oneside,a4paper]{article}
\usepackage{bussproofs}
\pagenumbering{arabic}
\title{Correzione simulazione appello}
\author{Matteo Mazzaretto}
\date{18 dicembre 2023}
\begin{document}
\maketitle
\section{Esercizio 1}
Mostrare se i sequenti elencati sotto sono tautologie, opinioni o paradossi in logica classica con
uguaglianza motivando la risposta: nel caso di sequente proposizionale non valido si indichi la riga
della tabella di verità in cui il sequente è falso e nel caso di sequente predicativo non valido si
mostri un contromodello (nel caso di non validità si assegna il doppio dei punti indicati)
\subsection{(3 punti): $\neg$$\neg$(C$\vee$A)$\vdash$$\neg$C$\&$A}
\begin{prooftree}
\RightLabel{\(\)}
\noLine
\AxiomC{}
\UnaryInfC{C,C$\vdash$}
\RightLabel{$\neg$-D}
\UnaryInfC{C$\vdash$$\neg$C}
\RightLabel{\(\)}
\noLine
\AxiomC{}
\UnaryInfC{A,C$\vdash$}
\RightLabel{$\neg$-D}
\UnaryInfC{A$\vdash$$\neg$C}
\RightLabel{$\vee$-S}
\BinaryInfC{C$\vee$A$\vdash$$\neg$C}
\RightLabel{$\neg$-S}
\UnaryInfC{$\vdash$$\neg$(C$\vee$A),$\neg$C}
\RightLabel{$\neg$-S}
\UnaryInfC{$\neg$$\neg$(C$\vee$A)$\vdash$$\neg$C}
\RightLabel{\(\)}
\noLine
\AxiomC{}
\UnaryInfC{riga falsaria}
\RightLabel{\(\)}
\noLine
\UnaryInfC{C$\vdash$A}
\RightLabel{\(\)}
\noLine
\AxiomC{}
\UnaryInfC{ax-id}
\RightLabel{\(\)}
\noLine
\UnaryInfC{A$\vdash$A}
\BinaryInfC{C$\vee$A$\vdash$A}
\RightLabel{$\neg$-S}
\UnaryInfC{$\vdash$$\neg$(C$\vee$A),A}
\RightLabel{$\neg$-S}
\UnaryInfC{$\neg$$\neg$(C$\vee$A)$\vdash$A}
\RightLabel{$\&$-D}
\BinaryInfC{$\neg$$\neg$(C$\vee$A)$\vdash$$\neg$C$\&$A}
\end{prooftree}
Il sequente di partenza è falso nella riga falsaria con C=1,A=0\newline
Derivo la negazione del sequente per capire se è opinione o paradosso
\begin{prooftree}
\RightLabel{\(\)}
\noLine
\AxiomC{}
\UnaryInfC{$\vdash$C,A}
\RightLabel{$\vee$-D}
\UnaryInfC{$\vdash$C$\vee$A}
\RightLabel{$\neg$$\neg$-D}
\UnaryInfC{$\vdash$$\neg$$\neg$(C$\vee$A)}
\RightLabel{\(\)}
\noLine
\AxiomC{}
\UnaryInfC{riga falsaria}
\RightLabel{\(\)}
\noLine
\UnaryInfC{A$\vdash$C}
\RightLabel{$\neg$-S}
\UnaryInfC{A,$\neg$C$\vdash$}
\RightLabel{Scsx}
\UnaryInfC{$\neg$C,A$\vdash$}
\RightLabel{$\&$-S}
\UnaryInfC{$\neg$C$\&$A$\vdash$}
\RightLabel{$\to$-S}
\BinaryInfC{$\neg$$\neg$(C$\vee$A)$\to$$\neg$C$\&$A$\vdash$}
\RightLabel{$\neg$-D}
\UnaryInfC{$\vdash$$\neg$($\neg$$\neg$(C$\vee$A)$\to$$\neg$C$\&$A)}
\end{prooftree}
La negazione del sequente è falso nella riga falsaria A=1, C=0\newline
Conclusione: il sequente di partenza è opinione, falso nella riga falsaria del sequente di partenza C=1, A=0, vero nella riga falsaria della negazione A=1, C=0
\subsection{(++) (6 punti): $\vdash$$\forall$y y$\Relbar$b $\&$ $\neg$$\neg$$\exists$x x$\neq$a}
\begin{prooftree}
\RightLabel{\(\)}
\noLine
\AxiomC{}
\UnaryInfC{$\vdash$y$\Relbar$b}
\RightLabel{$\forall$-D y$\notin$VL}
\UnaryInfC{$\vdash$$\forall$y y$\Relbar$b}
\noLine
\AxiomC{}
\UnaryInfC{$\vdash$$\neg$$\neg$$\exists$x x$\neq$a}
\RightLabel{$\&$-D}
\BinaryInfC{$\vdash$$\forall$y y$\Relbar$b $\&$ $\neg$$\neg$$\exists$x x$\neq$a}
\end{prooftree}
Dcontra=\{Topolino\}\newline
(Ipotizzando Topolino$\neq$b)\newline
(y=b)$^{Dcontra}$(Topolino)=0\newline
E quindi (tt$\vdash$$\forall$y y$\Relbar$b $\&$ $\neg$$\neg$$\exists$x x$\neq$a)$^{Dcontra}$=0\newline
Perché ($\forall$y y$\Relbar$b)$^{Dcontra}$(Topolino)=0\newline
$\neg$$\neg$$\exists$x x$\neq$a)$^{Dcontra}(Topolino)$=????\newline
tt$\vdash$0$\&$???=tt$\vdash$0=0\newline
Avendo trovato un contromodello, posso derivare la negazione
\begin{prooftree}
\RightLabel{\(\)}
\noLine
\AxiomC{}
\UnaryInfC{ax-id}
\RightLabel{\(\)}
\noLine
\UnaryInfC{a$\Relbar$b,x$\Relbar$a$\vdash$x$\Relbar$a}
\RightLabel{$\Relbar$-S}
\UnaryInfC{x$\Relbar$b,a$\Relbar$b$\vdash$x$\Relbar$a}
\RightLabel{$\forall$-Sv}
\UnaryInfC{x$\Relbar$b,$\forall$yy$\Relbar$b$\vdash$x$\Relbar$a}
\RightLabel{Scsx}
\UnaryInfC{$\forall$yy$\Relbar$b,x$\Relbar$b$\vdash$x$\Relbar$a}
\RightLabel{$\forall$-S}
\UnaryInfC{$\forall$yy$\Relbar$b$\vdash$x$\Relbar$a}
\RightLabel{$\neg$-S}
\UnaryInfC{$\forall$yy$\Relbar$b,x$\neq$a$\vdash$}
\RightLabel{$\exists$-Sx$\notin$VL}
\UnaryInfC{$\forall$yy$\Relbar$b,$\exists$xx$\neq$a$\vdash$}
\RightLabel{$\neg$$\neg$-S}
\UnaryInfC{$\forall$yy$\Relbar$b,$\neg$$\neg$$\exists$xx$\neq$a$\vdash$}
\RightLabel{$\&$-S}
\UnaryInfC{$\forall$yy$\Relbar$b$\&$$\neg$$\neg$$\exists$xx$\neq$a$\vdash$}
\RightLabel{$\neg$-D}
\UnaryInfC{$\vdash$$\neg$($\forall$yy$\Relbar$b$\&$$\neg$$\neg$$\exists$xx$\neq$a)}
\end{prooftree}
Conclusione: la negazione del sequente di partenza è derivabile, quindi il sequente di partenza è paradosso
\subsection{(5 punti): $\neg$B(w), $\neg$$\forall$z B(z) $\vdash$ $\exists$y $\neg$B(y)}
\begin{prooftree}
\RightLabel{\(\)}
\noLine
\AxiomC{}
\UnaryInfC{$\neg$-axdx2}
\RightLabel{\(\)}
\noLine
\UnaryInfC{$\vdash$$\neg$B(w),B(w),B(x)}
\RightLabel{$\exists$-Dv}
\UnaryInfC{$\vdash$$\exists$y$\neg$B(y),B(w),B(x)}
\RightLabel{Scdx}
\UnaryInfC{$\vdash$B(w),B(x),$\exists$y$\neg$B(y)}
\RightLabel{$\neg$-S}
\UnaryInfC{$\neg$B(w)$\vdash$B(x),$\exists$y$\neg$B(y)}
\RightLabel{$\forall$-D x$\notin$VL}
\UnaryInfC{$\neg$B(w)$\vdash$$\forall$zB(z),$\exists$y$\neg$B(y)}
\RightLabel{$\neg$-S}
\UnaryInfC{$\neg$B(w),$\neg$$\forall$zB(z)$\vdash$$\exists$y$\neg$B(y)}
\end{prooftree}
Conclusione: il sequente di partenza è derivabile quindi è tautologia
\section{Esercizio 2}
Formalizzare le seguenti asserzioni e stabilire se i sequenti ottenuti sono tautologie, opinioni o paradossi nella logica classica con uguaglianza motivando la risposta: nel caso di sequente predicativo non valido si mostri un contromodello (nel caso di non validità si assegna il doppio dei punti indicati)
\subsection{6 punti}
\begin{prooftree}
\RightLabel{\(\)}
\noLine
\AxiomC{}
\UnaryInfC{Chi ama è felice}
\RightLabel{\(\)}
\noLine
\UnaryInfC{Tutti quelli che non amano sono infelici}
\RightLabel{\(\)}
\UnaryInfC{Ognuno o è felice o non ama}
\end{prooftree}
Si consiglia di usare\newline
A(y) =“y ama” \newline
F(x) = “x è felice”
\setlength{\parindent}{0pt}\begin{prooftree}
\RightLabel{\(\)}
\noLine
\AxiomC{}
\UnaryInfC{$\neg$-axdx1}
\RightLabel{\(\)}
\noLine
\UnaryInfC{$\vdash$A(z),$\neg$A(z),F(z),$\neg$A(z)}
\RightLabel{\(\)}
\noLine
\AxiomC{}
\UnaryInfC{ax-id}
\RightLabel{\(\)}
\noLine
\UnaryInfC{F(z)$\vdash$$\neg$A(z),F(z),$\neg$A(z)}
\RightLabel{$\to$-S}
\BinaryInfC{A(z)$\to$F(z)$\vdash$$\neg$A(z),F(z),$\neg$A(z)}
\RightLabel{$\forall$-Sv}
\UnaryInfC{$\forall$x(A(x)$\to$F(x))$\vdash$$\neg$A(z),F(z),$\neg$A(z)}
\RightLabel{\(\)}
\noLine
\AxiomC{}
\UnaryInfC{(continua sotto)}
\RightLabel{\(\)}
\UnaryInfC{$\forall$x(A(x)$\to$F(x)),$\neg$F(z)$\vdash$F(z),$\neg$A(z)}
\RightLabel{$\to$-S}
\BinaryInfC{$\forall$x(A(x)$\to$F(x)),$\neg$A(z)$\to$$\neg$F(z)$\vdash$F(z),$\neg$A(z)}
\RightLabel{$\forall$-Sv}
\UnaryInfC{$\forall$x(A(x)$\to$F(x)),$\forall$y($\neg$A(y)$\to$$\neg$F(y))$\vdash$F(z),$\neg$A(z)}
\RightLabel{$\vee$-D}
\UnaryInfC{$\forall$x(A(x)$\to$F(x)),$\forall$y($\neg$A(y)$\to$$\neg$F(y))$\vdash$F(z)$\vee$$\neg$A(z)}
\RightLabel{$\forall$-D z $\notin$VL}
\UnaryInfC{$\forall$x(A(x)$\to$F(x)),$\forall$y($\neg$A(y)$\to$$\neg$F(y))$\vdash$$\forall$z(F(z)$\vee$$\neg$A(z)))}
\end{prooftree}
Continuazione di $\forall$x(A(x)$\to$F(x)),$\neg$F(z)$\vdash$F(z),$\neg$A(z)
\begin{prooftree}
\RightLabel{\(\)}
\noLine
\AxiomC{}
\UnaryInfC{$\neg$-axdx1}
\RightLabel{\(\)}
\noLine
\UnaryInfC{$\neg$F(z)$\vdash$A(z),F(z),$\neg$A(z)}
\RightLabel{\(\)}
\noLine
\AxiomC{}
\UnaryInfC{ax-id}
\RightLabel{\(\)}
\noLine
\UnaryInfC{$\neg$F(z),F(z)$\vdash$F(z),$\neg$A(z)}
\RightLabel{$\to$-S}
\BinaryInfC{$\neg$F(z),A(z)$\to$F(z)$\vdash$F(z),$\neg$A(z)}
\RightLabel{Scsx}
\UnaryInfC{A(z)$\to$F(z),$\neg$F(z)$\vdash$F(z),$\neg$A(z)}
\RightLabel{$\forall$-Sv}
\UnaryInfC{$\forall$x(A(x)$\to$F(x)),$\neg$F(z)$\vdash$F(z),$\neg$A(z)}
\end{prooftree}
Conclusione: il sequente di partenza è tautologia
\subsection{6 punti}
\begin{prooftree}
\RightLabel{\(\)}
\noLine
\AxiomC{}
\UnaryInfC{Qualcuno cerca e trova}
\RightLabel{\(\)}
\UnaryInfC{Qualcuno non trova ma neanche cerca}
\end{prooftree}
Si consiglia di usare:\newline
C(y) =“y cerca”\newline
T(x) = “x trova”\newline
Traduzione:
\begin{center}$\exists$x(C(x)$\&$T(x))$\vdash$$\exists$z($\neg$C(z)$\&$$\neg$T(z))\end{center}
\begin{prooftree}
\RightLabel{\(\)}
\noLine{}
\AxiomC{}
\UnaryInfC{loop}
\RightLabel{\(\)}
\noLine
\UnaryInfC{C(w),T(w),C(w)$\vdash$$\exists$z($\neg$C(z)$\&$$\neg$T(z))}
\RightLabel{$\neg$D}
\UnaryInfC{C(w),T(w)$\vdash$$\neg$C(w),$\exists$z($\neg$C(z)$\&$$\neg$T(z))}
\RightLabel{\(\)}
\noLine
\AxiomC{}
\UnaryInfC{loop}
\RightLabel{\(\)}
\noLine
\UnaryInfC{C(w),T(w),T(w)$\vdash$$\exists$z($\neg$C(z)$\&$$\neg$T(z))}
\RightLabel{$\neg$-D}
\UnaryInfC{C(w),T(w)$\vdash$$\neg$T(w),$\exists$z($\neg$C(z)$\&$$\neg$T(z))}
\RightLabel{$\&$-D}
\BinaryInfC{C(w),T(w)$\vdash$$\neg$C(w)$\&$$\neg$T(w),$\exists$z($\neg$C(z)$\&$$\neg$T(z))}
\RightLabel{$\exists$-D}
\UnaryInfC{C(w),T(w)$\vdash$$\exists$z($\neg$C(z)$\&$$\neg$T(z))}
\RightLabel{$\&$-S}
\UnaryInfC{C(w)$\&$T(w)$\vdash$$\exists$z($\neg$C(z)$\&$$\neg$T(z))}
\RightLabel{$\exists$-S w$\notin$VL}
\UnaryInfC{$\exists$x(C(x)$\&$T(x))$\vdash$$\exists$z($\neg$C(z)$\&$$\neg$T(z))}
\end{prooftree}
Dato che abbiamo trovato un loop, bisogna trovare un contromodello\newline
$D_{contra}$=\{Minni\} \newline
$C(w)^{Dcontra}$(Minni)=1\newline
$T(w)^{Dcontra}$(Minni)=1\newline
($\exists$z($\neg$C(z)$\&$$\neg$T(z)))$^{Dcontra}$=0\newline
vero perché ($\neg$T(z))(Minni)=0\newline
($\exists$x(C(x)$\&$T(x)))$^{Dcontra}$=1\newline
perché (C(x)$\&$T(x))$^{Dcontra}$(Minni)=1\newline
Il sequente radice non vale in tale modello perché \newline
$\exists$x(C(x)$\&$T(x))$\vdash$$\exists$z($\neg$C(z)$\&$$\neg$T(z)) = 1$\vdash$0=0 \newline
Provo a negare il sequente radice: \newline
$\vdash$$\neg$($\exists$x(C(x)$\&$T(x))$\to$$\exists$z($\neg$C(z)$\&$$\neg$T(z)))
\begin{prooftree}
\RightLabel{\(\)}
\noLine
\AxiomC{}
\UnaryInfC{loop}
\RightLabel{\(\)}
\noLine
\UnaryInfC{$\vdash$C(w),$\exists$x(C(x)$\&$T(x))}
\RightLabel{\(\)}
\noLine
\AxiomC{}
\UnaryInfC{loop}
\RightLabel{\(\)}
\noLine
\UnaryInfC{$\vdash$T(w),$\exists$x(C(x)$\&$T(x))}
\RightLabel{$\&$-D}
\BinaryInfC{$\vdash$C(w)$\&$T(w),$\exists$x(C(x)$\&$T(x))}
\RightLabel{$\exists$-D}
\UnaryInfC{$\vdash$$\exists$x(C(x)$\&$T(x))}
\RightLabel{\(\)}
\noLine
\AxiomC{}
\UnaryInfC{$\vdash$C(w),T(w)}
\RightLabel{$\neg$-S}
\UnaryInfC{$\neg$C(w)$\vdash$T(w)}
\RightLabel{$\neg$-S}
\UnaryInfC{$\neg$C(w),$\neg$T(w)$\vdash$}
\RightLabel{$\&$-S}
\UnaryInfC{$\neg$C(w)$\&$$\neg$T(w)$\vdash$}
\RightLabel{$\exists$-S w$\notin$VL}
\UnaryInfC{$\exists$z($\neg$C(z)$\&$$\neg$T(z))$\vdash$}
\RightLabel{$\to$-S}
\BinaryInfC{$\exists$x(C(x)$\&$T(x))$\to$$\exists$z($\neg$C(z)$\&$$\neg$T(z))$\vdash$}
\RightLabel{$\neg$-D}
\UnaryInfC{$\vdash$$\neg$($\exists$x(C(x)$\&$T(x))$\to$$\exists$z($\neg$C(z)$\&$$\neg$T(z)))}
\end{prooftree}
$D_{contraneg}$=\{Minni\} \newline
(T(w))$^{Dcontraneg}$ a piacere\newline
(C(w))$^{Dcontraneg}$(Minni)=0\newline
È un contromodello perché\newline
($\exists$x(C(x)$\&$T(x)))$^{Dcontraneg}$=0\newline
($\exists$z($\neg$C(z)$\&$$\neg$T(z)))$^{Dcontraneg}$=??\newline
$\neg$(0$\to$??)=$\neg$1=0\newline
Conclusione: il sequente di partenza è opinione con modello il contromodello della negazione e contromodello il contromodello del sequente di partenza
\subsection{(++) (22 punti)}
\begin{prooftree}
\RightLabel{\(\)}
\noLine
\AxiomC{}
\UnaryInfC{Charlotte è l'unica studentessa straniera}
\RightLabel{\(\)}
\noLine
\UnaryInfC{Giulia è diversa da Charlotte}
\RightLabel{\(\)}
\UnaryInfC{Giulia non è una studentessa straniera}
\end{prooftree}
Si consiglia di usare:\newline
S(x)= x è una studentessa straniera\newline
c=Charlotte, g=Giulia
\begin{prooftree}
\noLine
\RightLabel{\(\)}
\AxiomC{}
\noLine
\UnaryInfC{ax-id}
\RightLabel{\(\)}
\noLine
\UnaryInfC{S(c),S(g)$\vdash$S(g),g$\Relbar$c}
\noLine
\RightLabel{\(\)}
\AxiomC{}
\UnaryInfC{ax-id}
\noLine
\RightLabel{\(\)}
\noLine
\UnaryInfC{S(c),S(g),g$\Relbar$c$\vdash$g$\Relbar$c}
\RightLabel{$\to$-S}
\BinaryInfC{S(c),S(g),S(g)$\to$g$\Relbar$c$\vdash$g$\Relbar$c}
\RightLabel{$\forall$-Sv}
\UnaryInfC{S(c),S(g),$\forall$x(S(x)$\to$x=c)$\vdash$g$\Relbar$c}
\RightLabel{SCsx}
\UnaryInfC{S(c),$\forall$x(S(x)$\to$x=c),S(g)$\vdash$g$\Relbar$c}
\RightLabel{$\neg$-D}
\UnaryInfC{S(c),$\forall$x(S(x)$\to$x=c)$\vdash$$\neg$S(g),g$\Relbar$c}
\RightLabel{SCdx}
\UnaryInfC{S(c),$\forall$x(S(x)$\to$x=c)$\vdash$g$\Relbar$c,$\neg$S(g)}
\RightLabel{$\&$-S}
\UnaryInfC{S(c)$\&$$\forall$x(S(x)$\to$x=c)$\vdash$g$\Relbar$c,$\neg$S(g)}
\RightLabel{$\neg$-S}
\UnaryInfC{S(c)$\&$$\forall$x(S(x)$\to$x=c),g$\neq$c$\vdash$$\neg$S(g)}
\end{prooftree}
Conclusione: il sequente di partenza è tautologia
\subsection{(++) (20 punti)}
\begin{center}{\bf Attenzione: esercizio da non guardare perché non si capisce bene se fosse opinione o paradosso, in quanto la prof diceva che venisse paradosso ma a me ed altri ragazzi veniva opinione, quindi non metto la correzione venendo risultati diversi ma lascio la traduzione}\end{center}
"Esiste uno a cui nessuno scrive se e solo se lui scrive a tutti oppure non esiste uno che, se
lui scrive a qualcuno tutti scrivono a qualcuno.”\newline
Si consiglia di usare:\newline
S(x, y)=x scrive a y\newline
La traduzione è:
\begin{center}$\vdash$$\exists$x($\neg$$\exists$yS(y,x)$\leftrightarrow$$\forall$yS(x,y))$\vee$($\neg$$\exists$x($\exists$zS(x,z)$\to$$\forall$y$\exists$zS(y,z)))\end{center}
\section{Teoria 1}
(32 punti) Sia Tmon la teoria ottenuta estendendo LC= con la formalizzazione dei seguenti assiomi:\newline
1. Valerio non va in montagna se nevica.\newline
2. Nevica se Reinhold non va in montagna.\newline
3. Valerio non va in montagna solo se neanche Reinhold va in montagna.\newline
4. Nevica se e solo se, o Valerio non va in montagna o Reinhold non ci va.\newline
5. Se c'è uno che va in montagna, Reinhold non ci va e tutti vanno in montagna.\newline
6. Solo se Reinhold va in montagna, nevica.\newline
Si consiglia di usare:\newline
M(x)= x va in montagna\newline
N= Nevica\newline
v=Valerio r=Reinhold\newline
Formalizzare le seguenti affermazioni e dedurne la validità in Tmon:\newline
- (4 punti) Reinhold non va in montagna solo se, o nevica oppure non nevica.\newline
- (6 punti) Reinhold va in montagna.\newline
- (5 punti) Valerio va in montagna.\newline
- (5 punti ) Non nevica.\newline
- (5 punti) Non nevica e almeno uno non va in montagna.
\subsection{Traduzione assiomi}
Ax1: N$\to$$\neg$M(v)\newline
Ax2: $\neg$M(r)$\to$N\newline
Ax3: $\neg$M(v)$\to$$\neg$M(r)\newline
Ax4: N$\leftrightarrow$$\neg$M(v)$\vee$$\neg$M(r)\newline
Ax5: $\exists$xM(x)$\to$($\neg$M(r)$\&$$\forall$xM(x))\newline
Ax6: N$\to$M(r)
\subsection{Traduzione teoremi}
Th1: $\neg$M(r)$\to$N$\vee$$\neg$N\newline
Th2: M(r)\newline
Th3: M(v)\newline
Th4: $\neg$N\newline
Th5: $\neg$N$\&$$\exists$x$\neg$M(x)
\subsection{Dimostrazione teoremi}
\subsubsection{Dimostrazione teorema 1}
\begin{prooftree}
\RightLabel{\(\)}
\noLine
\AxiomC{}
\UnaryInfC{$\vdash$ax1}
\RightLabel{\(\)}
\noLine
\AxiomC{}
\UnaryInfC{$\neg$-axdx1}
\RightLabel{\(\)}
\noLine
\UnaryInfC{N$\to$$\neg$M(v),$\neg$M(v)$\vdash$N,$\neg$N}
\RightLabel{$\vee$-D}
\UnaryInfC{N$\to$$\neg$M(v),$\neg$M(v)$\vdash$N$\vee$$\neg$N}
\RightLabel{$\to$-D}
\UnaryInfC{N$\to$$\neg$M(v)$\vdash$$\neg$M(v)$\to$N$\vee$$\neg$N}
\RightLabel{comp}
\BinaryInfC{$\vdash$$\neg$M(r)$\to$N$\vee$$\neg$N}
\end{prooftree}
Teorema 1 dimostrato
\subsubsection{Dimostrazione teorema 2}
\begin{prooftree}
\RightLabel{\(\)}
\noLine
\AxiomC{}
\UnaryInfC{$\vdash$ax6}
\RightLabel{\(\)}
\noLine
\AxiomC{}
\UnaryInfC{$\vdash$ax2}
\RightLabel{\(\)}
\noLine
\AxiomC{}
\UnaryInfC{$\neg$-axdx2}
\RightLabel{\(\)}
\noLine
\UnaryInfC{$\vdash$$\neg$M(r),N,M(r)}
\RightLabel{\(\)}
\noLine
\AxiomC{}
\UnaryInfC{ax-id}
\RightLabel{\(\)}
\noLine
\UnaryInfC{N$\vdash$N,M(r)}
\RightLabel{$\to$-S}
\BinaryInfC{$\neg$M(r)$\to$N$\vdash$N,M(r)}
\RightLabel{comp}
\BinaryInfC{$\vdash$N,M(r)}
\RightLabel{\(\)}
\noLine
\AxiomC{}
\UnaryInfC{ax-id}
\RightLabel{\(\)}
\noLine
\UnaryInfC{M(r)$\vdash$M(r)}
\RightLabel{$\to$-S}
\BinaryInfC{N$\to$M(r)$\vdash$M(r)}
\RightLabel{comp}
\BinaryInfC{$\vdash$M(r)}
\end{prooftree}
Teorema 2 dimostrato
\subsubsection{Dimostrazione teorema 3}
\begin{prooftree}
\RightLabel{\(\)}
\noLine
\AxiomC{}
\UnaryInfC{$\vdash$Th2,ax3}
\RightLabel{\(\)}
\noLine
\AxiomC{}
\UnaryInfC{$\neg$-axdx2}
\RightLabel{\(\)}
\noLine
\UnaryInfC{M(r)$\vdash$$\neg$M(v),M(v)}
\RightLabel{\(\)}
\noLine
\AxiomC{}
\UnaryInfC{$\neg$-axsx1}
\RightLabel{\(\)}
\noLine
\UnaryInfC{M(r),$\neg$M(r)$\vdash$M(v)}
\RightLabel{$\to$-S}
\BinaryInfC{M(r),$\neg$M(v)$\to$$\neg$M(r)$\vdash$M(v)}
\RightLabel{comp}
\BinaryInfC{$\vdash$M(v)}
\end{prooftree}
Teorema 3 dimostrato
\subsubsection{Dimostrazione teorema 4}
\begin{prooftree}
\RightLabel{\(\)}
\noLine
\AxiomC{}
\UnaryInfC{$\vdash$ax1}
\RightLabel{\(\)}
\noLine
\AxiomC{}
\UnaryInfC{$\neg$-axdx1}
\RightLabel{\(\)}
\noLine
\UnaryInfC{$\vdash$N,$\neg$N}
\RightLabel{\(\)}
\noLine
\AxiomC{}
\UnaryInfC{$\vdash$th3}
\RightLabel{\(\)}
\noLine
\AxiomC{}
\UnaryInfC{$\neg$-axsx1}
\RightLabel{\(\)}
\noLine
\UnaryInfC{M(v),$\neg$M(v)$\vdash$$\neg$N}
\RightLabel{comp}
\BinaryInfC{$\neg$M(v)$\vdash$$\neg$N}
\RightLabel{$\to$-S}
\BinaryInfC{N$\to$$\neg$M(v)$\vdash$$\neg$N}
\RightLabel{comp}
\BinaryInfC{$\vdash$$\neg$N}
\end{prooftree}
Teorema 4 dimostrato
\subsubsection{Dimostrazione teorema 5}
\begin{prooftree}
\noLine
\RightLabel{\(\)}
\AxiomC{}
\UnaryInfC{$\vdash$ax5}
\RightLabel{\(\)}
\noLine
\AxiomC{}
\UnaryInfC{$\vdash$th4}
\RightLabel{\(\)}
\noLine
\AxiomC{}
\UnaryInfC{ax-id}
\RightLabel{\(\)}
\noLine
\UnaryInfC{$\neg$N$\vdash$$\neg$N,$\exists$xM(x)}
\RightLabel{comp}
\BinaryInfC{$\vdash$$\neg$N,$\exists$xM(x)}
\RightLabel{\(\)}
\noLine
\AxiomC{}
\UnaryInfC{ax-id}
\RightLabel{\(\)}
\noLine
\UnaryInfC{M(z)$\vdash$M(z)}
\RightLabel{$\exists$-Dv}
\UnaryInfC{M(z)$\vdash$$\exists$xM(x)}
\RightLabel{$\neg$-D}
\UnaryInfC{$\vdash$$\neg$M(z),$\exists$xM(x)}
\RightLabel{$\exists$-Dv}
\UnaryInfC{$\vdash$$\exists$x$\neg$M(x),$\exists$xM(x)}
\RightLabel{$\&$-D}
\BinaryInfC{$\vdash$$\neg$N$\&$$\exists$x$\neg$M(x),$\exists$xM(x)}
\RightLabel{Scdx}
\UnaryInfC{$\vdash$$\exists$xM(x),$\neg$N$\&$$\exists$x$\neg$M(x)}
\RightLabel{\(\)}
\noLine
\AxiomC{}
\UnaryInfC{Sviluppo sotto}
\RightLabel{$\to$-S}
\BinaryInfC{$\exists$xM(x)$\to$($\neg$M(r)$\&$$\forall$xM(x))$\vdash$$\neg$N$\&$$\exists$x$\neg$M(x)}
\RightLabel{comp}
\BinaryInfC{$\vdash$$\neg$N$\&$$\exists$x$\neg$M(x)}
\end{prooftree}
Continuo sviluppo foglia di destra a seguito di $\to$-S\newline
Da sviluppare $\neg$M(r)$\&$$\forall$xM(x)$\vdash$$\neg$N$\&$$\exists$x$\neg$M(x)
\begin{prooftree}
\RightLabel{\(\)}
\noLine
\AxiomC{}
\UnaryInfC{$\neg$-axsx2}
\RightLabel{\(\)}
\noLine
\UnaryInfC{$\neg$M(r),M(r)$\vdash$$\neg$N$\&$$\exists$x$\neg$M(x)}
\RightLabel{$\forall$-Sv}
\UnaryInfC{$\neg$M(r),$\forall$xM(x)$\vdash$$\neg$N$\&$$\exists$x$\neg$M(x)}
\RightLabel{$\&$-S}
\UnaryInfC{$\neg$M(r)$\&$$\forall$xM(x)$\vdash$$\neg$N$\&$$\exists$x$\neg$M(x)}
\end{prooftree}
Teorema 5 dimostrato
\section{Teoria 2}
 (++ 60 punti) Sia Tamm la teoria ottenuta estendendo LC= con la formalizzazione dei seguenti
assiomi:\newline
- (1 punto) Beatrice non è Rosa.\newline
- (3 punti) Uno ammira un altro qualsiasi se e solo se quest’altro non ammira il primo.\newline
- (2 punti) Non si dà il caso che qualcuno non ammiri Ester.\newline
- (2 punti) Gino è ammirato da tutti.\newline
- (3 punti) Non si dà il caso che qualcuno non ammiri sé stesso.\newline
- (3 punti) Beatrice ammira Rosa e soltanto lei.\newline
Si consiglia di usare:\newline
A(x, y)= x ammira y\newline
g=“Gino” r=“Rosa” b=“Beatrice” e=“Ester”\newline
Dopo aver formalizzato le frase seguenti mostrarne una derivazione nella teoria in Tamm:\newline
- (5 punti) Rosa ammira Ester.\newline
- (5 punti) Rosa non si ammira.\newline
- (12 punti) Beatrice non si ammira e non si dà il caso che non esista nessuno diverso da Rosa.\newline
- (12 punti) Ester non ammira Rosa.\newline
- (12 punti) Gino non ammira nessuno.
\subsection{Traduzione assiomi}
Ax1: b$\neq$r\newline
Ax2: $\forall$x$\forall$y(A(x,y)$\leftrightarrow$$\neg$A(y,x))\newline
Ax3: $\neg$$\exists$x$\neg$A(x,e)\newline
Ax4: $\forall$y A(y,g)\newline
Ax5: $\neg$$\exists$x$\neg$A(x,x)\newline
Ax6: A(b,r)$\&$$\forall$y(A(b,y)$\to$y$\Relbar$r)
\subsection{Traduzione teoremi}
Th1: A(r,e) \newline
Th2: $\neg$A(r,r)\newline
Th3: $\neg$A(b,b)$\&$ $\neg$$\neg$$\exists$x x$\neq$r \newline
Th4: $\neg$A(e,r)\newline
Th5: $\neg$$\exists$xA(g,x)
\subsection{Dimostrazione teoremi}
\subsubsection{Dimostrazione teorema 1}
\begin{prooftree}
\RightLabel{\(\)}
\noLine
\AxiomC{}
\UnaryInfC{$\vdash$ax3}
\RightLabel{\(\)}
\noLine
\AxiomC{}
\UnaryInfC{$\neg$-axdx2}
\RightLabel{\(\)}
\noLine
\UnaryInfC{$\vdash$$\neg$A(x,e),A(r,e)}
\RightLabel{$\exists$-Dv}
\UnaryInfC{$\vdash$$\exists$x$\neg$A(x,e),A(r,e)}
\RightLabel{$\neg$-S}
\UnaryInfC{$\neg$$\exists$x$\neg$A(x,e)$\vdash$A(r,e)}
\RightLabel{comp}
\BinaryInfC{$\vdash$A(r,e)}
\end{prooftree}
Teorema 1 dimostrato
\subsubsection{Dimostrazione teorema 2}
\begin{prooftree}
\RightLabel{\(\)}
\noLine
\AxiomC{}
\UnaryInfC{$\vdash$ax2}
\RightLabel{\(\)}
\noLine
\AxiomC{}
\UnaryInfC{$\neg$-axdx1}
\RightLabel{\(\)}
\noLine
\UnaryInfC{$\vdash$A(r,r),$\neg$A(r,r),$\neg$A(r,r)}
\RightLabel{\(\)}
\noLine
\AxiomC{}
\UnaryInfC{ax-id}
\RightLabel{\(\)}
\noLine
\UnaryInfC{$\neg$A(r,r)$\vdash$$\neg$A(r,r),$\neg$A(r,r)}
\RightLabel{$\to$-S}
\BinaryInfC{A(r,r)$\to$$\neg$A(r,r)$\vdash$$\neg$A(r,r),$\neg$A(r,r)}
\noLine
\AxiomC{}
\RightLabel{}
\UnaryInfC{continua sotto}
\RightLabel{$\to$-S}
\BinaryInfC{A(r,r)$\to$$\neg$A(r,r),$\neg$A(r,r)$\to$A(r,r)$\vdash$$\neg$A(r,r)}
\RightLabel{$\&$-S}
\UnaryInfC{(A(r,r)$\leftrightarrow$$\neg$A(y,r))$\vdash$$\neg$A(r,r)}
\RightLabel{$\forall$-Sv}
\UnaryInfC{$\forall$y(A(r,y)$\leftrightarrow$$\neg$A(y,r))$\vdash$$\neg$A(r,r)}
\RightLabel{$\forall$-Sv}
\UnaryInfC{$\forall$x$\forall$y(A(x,y)$\leftrightarrow$$\neg$A(y,x))$\vdash$$\neg$A(r,r)}
\RightLabel{comp}
\BinaryInfC{$\vdash$$\neg$A(r,r)}
\end{prooftree}
Continuazione della foglia destra dell'implica a sinistra\newline
A(r,r)$\to$$\neg$A(r,r),A(r,r)$\vdash$$\neg$A(r,r)
\begin{prooftree}
\RightLabel{\(\)}
\noLine
\AxiomC{}
\UnaryInfC{ax-id}
\RightLabel{\(\)}
\noLine
\UnaryInfC{A(r,r)$\vdash$A(r,r),$\neg$A(r,r)}
\RightLabel{\(\)}
\noLine
\AxiomC{}
\UnaryInfC{$\neg$axsx1}
\RightLabel{\(\)}
\noLine
\UnaryInfC{A(r,r),$\neg$A(r,r)$\vdash$$\neg$A(r,r)}
\RightLabel{$\to$-S}
\BinaryInfC{A(r,r),A(r,r)$\to$$\neg$A(r,r)$\vdash$$\neg$A(r,r)}
\RightLabel{Scsx}
\UnaryInfC{A(r,r)$\to$$\neg$A(r,r),A(r,r)$\vdash$$\neg$A(r,r)}
\end{prooftree}
Teorema 2 dimostrato
\subsubsection{Dimostrazione teorema 3}
\begin{prooftree}
\RightLabel{\(\)}
\noLine
\AxiomC{}
\UnaryInfC{$\vdash$ax2}
\RightLabel{\(\)}
\noLine
\AxiomC{}
\UnaryInfC{$\forall$x$\forall$y(A(x,y)$\leftrightarrow$$\neg$A(y,x))$\vdash$$\neg$A(b,b)}
\RightLabel{\(\)}
\noLine
\AxiomC{}
\UnaryInfC{$\forall$x$\forall$y(A(x,y)$\leftrightarrow$$\neg$A(y,x))$\vdash$$\neg$$\neg$$\exists$xx$\neq$r}
\RightLabel{$\&$-D}
\BinaryInfC{$\forall$x$\forall$y(A(x,y)$\leftrightarrow$$\neg$A(y,x))$\vdash$$\neg$A(b,b)$\&$$\neg$$\neg$$\exists$xx$\neq$r}
\RightLabel{comp}
\BinaryInfC{$\vdash$$\neg$A(b,b)$\&$$\neg$$\neg$$\exists$xx$\neq$r}
\end{prooftree}
Derivo separatamente le due foglie ottenute dopo l'applicazione della regola $\&$-D\newline
Faccio prima $\forall$x$\forall$y(A(x,y)$\leftrightarrow$$\neg$A(y,x))$\vdash$$\neg$A(b,b)
\begin{prooftree}
\RightLabel{\(\)}
\noLine
\AxiomC{}
\UnaryInfC{$\neg$-axdx1}
\RightLabel{\(\)}
\noLine
\UnaryInfC{$\vdash$A(b,b),$\neg$A(b,b),$\neg$A(b,b)}
\RightLabel{\(\)}
\noLine
\AxiomC{}
\UnaryInfC{ax-id}
\RightLabel{\(\)}
\noLine
\UnaryInfC{$\neg$A(b,b)$\vdash$$\neg$A(b,b),$\neg$A(b,b)}
\RightLabel{$\to$-S}
\BinaryInfC{A(b,b)$\to$$\neg$A(b,b)$\vdash$$\neg$A(b,b),$\neg$A(b,b)}
\RightLabel{\(\)}
\noLine
\AxiomC{}
\UnaryInfC{continua sotto}
\RightLabel{$\to$-S}
\BinaryInfC{A(b,b)$\to$$\neg$A(b,b),$\neg$A(b,b)$\to$A(b,b)$\vdash$$\neg$A(b,b)}
\RightLabel{$\&$-S}
\UnaryInfC{A(b,b)$\leftrightarrow$$\neg$A(b,b)$\vdash$$\neg$A(b,b)}
\RightLabel{$\forall$-Sv}
\UnaryInfC{$\forall$y(A(b,y)$\leftrightarrow$$\neg$A(y,b))$\vdash$$\neg$A(b,b)}
\RightLabel{$\forall$-Sv}
\UnaryInfC{$\forall$x$\forall$y(A(x,y)$\leftrightarrow$$\neg$A(y,x))$\vdash$$\neg$A(b,b)}
\end{prooftree}
Continuo la foglia di destra dopo l'implica a sinistra:\newline
A(b,b)$\to$$\neg$A(b,b),A(b,b)$\vdash$$\neg$A(b,b)
\begin{prooftree}
\RightLabel{\(\)}
\noLine
\AxiomC{}
\UnaryInfC{ax-id}
\RightLabel{\(\)}
\noLine
\UnaryInfC{A(b,b)$\vdash$A(b,b)$\neg$A(b,b)}
\RightLabel{\(\)}
\noLine
\AxiomC{}
\UnaryInfC{$\neg$-axsx1}
\RightLabel{\(\)}
\noLine
\UnaryInfC{A(b,b),$\neg$A(b,b)$\vdash$$\neg$A(b,b)}
\RightLabel{$\to$-S}
\BinaryInfC{A(b,b),A(b,b)$\to$$\neg$A(b,b)$\vdash$$\neg$A(b,b)}
\RightLabel{Scsx}
\UnaryInfC{A(b,b)$\to$$\neg$A(b,b),A(b,b)$\vdash$$\neg$A(b,b)}
\end{prooftree}
Adesso derivo la foglia di destra dopo l'applicazione della regola $\&$-D\newline
$\forall$x$\forall$y(A(x,y)$\leftrightarrow$$\neg$A(y,x))$\vdash$$\neg$$\neg$$\exists$xx$\neq$r
\begin{prooftree}
\RightLabel{\(\)}
\noLine
\AxiomC{}
\UnaryInfC{$\neg$-axdx1}
\RightLabel{\(\)}
\noLine
\UnaryInfC{$\vdash$A(b,b),$\neg$A(b,b),$\exists$xx$\neq$r}
\RightLabel{\(\)}
\noLine
\AxiomC{}
\UnaryInfC{ax-id}
\RightLabel{\(\)}
\noLine
\UnaryInfC{$\neg$A(b,b)$\vdash$$\neg$A(b,b),$\exists$xx$\neq$r}
\RightLabel{$\to$-S}
\BinaryInfC{A(b,b)$\to$$\neg$A(b,b)$\vdash$$\neg$A(b,b),$\exists$xx$\neq$r}
\RightLabel{\(\)}
\noLine
\AxiomC{}
\UnaryInfC{continua sotto}
\RightLabel{$\to$-S}
\BinaryInfC{A(b,b)$\to$$\neg$A(b,b),$\neg$A(b,b)$\to$A(b,b)$\vdash$$\exists$xx$\neq$r}
\RightLabel{$\&$-S}
\UnaryInfC{A(b,b)$\leftrightarrow$$\neg$A(b,b)$\vdash$$\exists$xx$\neq$r}
\RightLabel{$\forall$-Sv}
\UnaryInfC{$\forall$y(A(b,y)$\leftrightarrow$$\neg$A(y,b))$\vdash$$\exists$xx$\neq$r}
\RightLabel{$\forall$-Sv}
\UnaryInfC{$\forall$x$\forall$y(A(x,y)$\leftrightarrow$$\neg$A(y,x))$\vdash$$\exists$xx$\neq$r}
\RightLabel{$\neg$$\neg$-D}
\UnaryInfC{$\forall$x$\forall$y(A(x,y)$\leftrightarrow$$\neg$A(y,x))$\vdash$$\neg$$\neg$$\exists$xx$\neq$r}
\end{prooftree}
Continuo la derivazione dell'altra foglia\newline
A(b,b)$\to$$\neg$A(b,b),A(b,b)$\vdash$$\exists$xx$\neq$r
\begin{prooftree}
\RightLabel{\(\)}
\noLine
\AxiomC{}
\UnaryInfC{ax-id}
\RightLabel{\(\)}
\noLine
\UnaryInfC{A(b,b)$\vdash$A(b,b),$\exists$xx$\neq$r}
\RightLabel{\(\)}
\noLine
\AxiomC{}
\UnaryInfC{$\neg$-axsx1}
\RightLabel{\(\)}
\noLine
\UnaryInfC{A(b,b),$\neg$A(b,b)$\vdash$$\exists$xx$\neq$r}
\RightLabel{$\to$-S}
\BinaryInfC{A(b,b),A(b,b)$\to$$\neg$A(b,b)$\vdash$$\exists$xx$\neq$r}
\RightLabel{Scsx}
\UnaryInfC{A(b,b)$\to$$\neg$A(b,b),A(b,b)$\vdash$$\exists$xx$\neq$r}
\end{prooftree}
Teorema 3 dimostrato
\subsubsection{Dimostrazione teorema 4}
\begin{prooftree}
\RightLabel{\(\)}
\noLine
\AxiomC{}
\UnaryInfC{$\vdash$ax6}
\RightLabel{\(\)}
\noLine
\AxiomC{}
\UnaryInfC{ax-id}
\RightLabel{\(\)}
\noLine
\UnaryInfC{A(b,r)$\vdash$A(b,r),$\neg$A(e,r)}
\AxiomC{}
\noLine
\UnaryInfC{$\vdash$ax2}
\noLine
\AxiomC{}
\UnaryInfC{continua sotto}
\noLine
\UnaryInfC{$\forall$x$\forall$y(A(x,y)$\leftrightarrow$$\neg$A(y,x)),A(b,r),r$\Relbar$r,A(e,r)$\vdash$}
\RightLabel{comp}
\BinaryInfC{A(b,r),r$\Relbar$r,A(e,r)$\vdash$}
\RightLabel{$\neg$-D}
\UnaryInfC{A(b,r),r$\Relbar$r$\vdash$$\neg$A(e,r)}
\RightLabel{$\to$-S}
\BinaryInfC{A(b,r),A(b,r)$\to$r$\Relbar$r$\vdash$$\neg$A(e,r)}
\RightLabel{$\forall$-Sv}
\UnaryInfC{A(b,r),$\forall$y(A(b,y)$\to$y$\Relbar$r)$\vdash$$\neg$A(e,r)}
\RightLabel{$\&$-S}
\UnaryInfC{A(b,r)$\&$$\forall$y(A(b,y)$\to$y$\Relbar$r)$\vdash$$\neg$A(e,r)}
\RightLabel{comp}
\BinaryInfC{$\vdash$$\neg$A(e,r)}
\end{prooftree}
Continuazione di $\forall$x$\forall$y(A(x,y)$\leftrightarrow$$\neg$A(y,x)),A(b,r),r$\Relbar$r,A(e,r)$\vdash$
\begin{prooftree}
\RightLabel{\(\)}
\noLine
\AxiomC{}
\UnaryInfC{continua sotto}
\RightLabel{$\to$-S}
\UnaryInfC{A(b,r),r$\Relbar$r,A(e,r),A(r,r)$\to$$\neg$A(r,r),$\neg$A(r,r)$\to$A(r,r)$\vdash$}
\RightLabel{$\&$-S}
\UnaryInfC{A(b,r),r$\Relbar$r,A(e,r),A(r,r)$\leftrightarrow$$\neg$A(r,r)$\vdash$}
\RightLabel{$\forall$-Sv}
\UnaryInfC{A(b,r),r$\Relbar$r,A(e,r),$\forall$y(A(r,y)$\leftrightarrow$$\neg$A(y,r))$\vdash$}
\RightLabel{$\forall$-Sv}
\UnaryInfC{A(b,r),r$\Relbar$r,A(e,r),$\forall$x$\forall$y(A(x,y)$\leftrightarrow$$\neg$A(y,x))$\vdash$}
\RightLabel{SCsx}
\UnaryInfC{$\forall$x$\forall$y(A(x,y)$\leftrightarrow$$\neg$A(y,x)),A(b,r),r$\Relbar$r,A(e,r)$\vdash$}
\end{prooftree}
Foglia sinistra a seguito dell'applicazione della regola $\to$-S:\newline
Derivazione di A(b,r),r$\Relbar$r,A(e,r),A(r,r)$\to$$\neg$A(r,r)$\vdash$$\neg$A(r,r)
\begin{prooftree}
\RightLabel{\(\)}
\noLine
\AxiomC{}
\UnaryInfC{$\neg$-axdx1}
\RightLabel{\(\)}
\noLine
\UnaryInfC{A(b,r),r$\Relbar$r,A(e,r)$\vdash$A(r,r),$\neg$A(r,r)}
\RightLabel{\(\)}
\noLine
\AxiomC{}
\UnaryInfC{ax-id}
\RightLabel{\(\)}
\noLine
\UnaryInfC{A(b,r),r$\Relbar$r,A(e,r),$\neg$A(r,r)$\vdash$$\neg$A(r,r)}
\RightLabel{$\to$-S}
\BinaryInfC{A(b,r),r$\Relbar$r,A(e,r),A(r,r)$\to$$\neg$A(r,r)$\vdash$$\neg$A(r,r)}
\end{prooftree}
Foglia destra a seguito dell'applicazione della regola $\to$-S:\newline
Derivazione di: A(b,r),r$\Relbar$r,A(e,r),A(r,r)$\to$$\neg$A(r,r),A(r,r)$\vdash$
\begin{prooftree}
\RightLabel{\(\)}
\noLine
\AxiomC{}
\UnaryInfC{ax-id}
\RightLabel{\(\)}
\noLine
\UnaryInfC{A(b,r),r$\Relbar$r,A(e,r),A(r,r)$\vdash$A(r,r)}
\RightLabel{\(\)}
\noLine
\AxiomC{}
\UnaryInfC{$\neg$-axsx1}
\RightLabel{\(\)}
\noLine
\UnaryInfC{A(b,r),r$\Relbar$r,A(e,r),A(r,r),$\neg$A(r,r)$\vdash$}
\RightLabel{$\to$-S}
\BinaryInfC{A(b,r),r$\Relbar$r,A(e,r),A(r,r),A(r,r)$\to$$\neg$A(r,r)$\vdash$}
\RightLabel{SCsx}
\UnaryInfC{A(b,r),r$\Relbar$r,A(e,r),A(r,r)$\to$$\neg$A(r,r),A(r,r)$\vdash$}
\end{prooftree}
Teorema 4 dimostrato
\subsubsection{Dimostrazione teorema 5}
\begin{prooftree}
\RightLabel{\(\)}
\noLine
\AxiomC{}
\UnaryInfC{$\vdash$ax4}
\RightLabel{\(\)}
\noLine
\AxiomC{}
\UnaryInfC{$\vdash$ax2}
\RightLabel{\(\)}
\noLine
\AxiomC{}
\UnaryInfC{continua sotto}
\RightLabel{}
\noLine
\UnaryInfC{A(y,g),A(g,y),A(g,y)$\to$$\neg$A(y,g),$\neg$A(y,g)$\to$A(g,y)$\vdash$}
\RightLabel{$\&$-S}
\UnaryInfC{A(y,g),A(g,y),A(g,y)$\leftrightarrow$$\neg$A(y,g)$\vdash$}
\RightLabel{$\forall$-Sv}
\UnaryInfC{A(y,g),A(g,y),$\forall$y(A(g,y)$\leftrightarrow$$\neg$A(y,g))$\vdash$}
\RightLabel{$\forall$-Sv}
\UnaryInfC{A(y,g),A(g,y),$\forall$x$\forall$y(A(x,y)$\leftrightarrow$$\neg$A(y,x))$\vdash$}
\RightLabel{SCsx}
\UnaryInfC{$\forall$x$\forall$y(A(x,y)$\leftrightarrow$$\neg$A(y,x)),A(y,g),A(g,y)$\vdash$}
\RightLabel{comp}
\BinaryInfC{A(y,g),A(g,y)$\vdash$}
\RightLabel{$\forall$-Sv}
\UnaryInfC{$\forall$yA(y,g),A(g,y)$\vdash$}
\RightLabel{$\exists$-S y$\notin$VL}
\UnaryInfC{$\forall$yA(y,g),$\exists$xA(g,x)$\vdash$}
\RightLabel{$\neg$-D}
\UnaryInfC{$\forall$yA(y,g)$\vdash$$\neg$$\exists$xA(g,x)}
\RightLabel{comp}
\BinaryInfC{$\vdash$$\neg$$\exists$xA(g,x)}
\end{prooftree}
Continuo la derivazione
\begin{prooftree}
\RightLabel{\(\)}
\noLine
\AxiomC{}
\UnaryInfC{ax-id}
\RightLabel{\(\)}
\noLine
\UnaryInfC{A(y,g),A(g,y)$\vdash$A(g,y),$\neg$A(y,g)}
\RightLabel{\(\)}
\noLine
\AxiomC{}
\UnaryInfC{$\neg$-axsx1}
\RightLabel{\(\)}
\noLine
\UnaryInfC{A(y,g),A(g,y),$\neg$A(y,g)$\vdash$$\neg$A(y,g)}
\RightLabel{$\to$-S}
\BinaryInfC{A(y,g),A(g,y),A(g,y)$\to$$\neg$A(y,g)$\vdash$$\neg$A(y,g)}
\RightLabel{\(\)}
\noLine
\AxiomC{}
\UnaryInfC{continua sotto}
\RightLabel{$\to$-S}
\BinaryInfC{A(y,g),A(g,y),A(g,y)$\to$$\neg$A(y,g),$\neg$A(y,g)$\to$A(g,y)$\vdash$}
\end{prooftree}
Continuazione di A(y,g),A(g,y),A(g,y)$\to$$\neg$A(y,g),A(g,y)$\vdash$
\begin{prooftree}
\RightLabel{\(\)}
\noLine
\AxiomC{}
\UnaryInfC{ax-id}
\RightLabel{\(\)}
\noLine
\UnaryInfC{A(y,g),A(g,y),A(g,y)$\vdash$A(g,y)}
\RightLabel{\(\)}
\noLine
\AxiomC{}
\UnaryInfC{$\neg$-axsx1}
\RightLabel{\(\)}
\noLine
\UnaryInfC{A(y,g),A(g,y),A(g,y),$\neg$A(y,g)$\vdash$}
\RightLabel{$\to$-S}
\BinaryInfC{A(y,g),A(g,y),A(g,y),A(g,y)$\to$$\neg$A(y,g)$\vdash$}
\RightLabel{SCsx}
\UnaryInfC{A(y,g),A(g,y),A(g,y)$\to$$\neg$A(y,g),A(g,y)$\vdash$}
\end{prooftree}
Teorema 5 dimostrato
\subsection{Alternativa}
Si nota che l'assioma 2 dimostra da solo il falso, quindi procediamo alla derivazione del falso
\begin{prooftree}
\RightLabel{\(\)}
\noLine
\AxiomC{}
\UnaryInfC{$\vdash$ax2}
\RightLabel{\(\)}
\noLine
\AxiomC{}
\UnaryInfC{$\neg$axdx1}
\RightLabel{\(\)}
\noLine
\UnaryInfC{$\vdash$A(x,x),$\neg$A(x,x),$\bot$}
\RightLabel{\(\)}
\noLine
\AxiomC{}
\UnaryInfC{ax-id}
\RightLabel{\(\)}
\noLine
\UnaryInfC{$\neg$A(x,x)$\vdash$$\neg$A(x,x),$\bot$}
\RightLabel{$\to$-S}
\BinaryInfC{A(x,x)$\to$$\neg$A(x,x)$\vdash$$\neg$A(x,x),$\bot$}
\RightLabel{\(\)}
\noLine
\AxiomC{}
\UnaryInfC{continua sotto}
\RightLabel{$\to$-S}
\BinaryInfC{A(x,x)$\to$$\neg$A(x,x),$\neg$A(x,x)$\to$A(x,x)$\vdash$$\bot$}
\RightLabel{$\&$-S}
\UnaryInfC{(A(x,x)$\leftrightarrow$$\neg$A(x,x))$\vdash$$\bot$}
\RightLabel{$\forall$-Sv}
\UnaryInfC{$\forall$y(A(x,y)$\leftrightarrow$$\neg$A(y,x))$\vdash$$\bot$}
\RightLabel{$\forall$-Sv}
\UnaryInfC{$\forall$x$\forall$y(A(x,y)$\leftrightarrow$$\neg$A(y,x))$\vdash$$\bot$}
\RightLabel{comp}
\BinaryInfC{$\vdash$$\bot$}
\end{prooftree}
Continuo la derivazione dell'altra foglia\newline
A(x,x)$\to$$\neg$A(x,x),A(x,x)$\vdash$$\bot$
\begin{prooftree}
\RightLabel{\(\)}
\noLine
\AxiomC{}
\UnaryInfC{ax-id}
\RightLabel{\(\)}
\noLine
\UnaryInfC{A(x,x)$\vdash$A(x,x),$\bot$}
\RightLabel{\(\)}
\noLine
\AxiomC{}
\UnaryInfC{$\neg$-axsx1}
\RightLabel{\(\)}
\noLine
\UnaryInfC{A(x,x),$\neg$A(x,x)$\vdash$$\bot$}
\RightLabel{$\to$-S}
\BinaryInfC{A(x,x),A(x,x)$\to$$\neg$A(x,x)$\vdash$$\bot$}
\RightLabel{Scsx}
\UnaryInfC{A(x,x)$\to$$\neg$A(x,x),A(x,x)$\vdash$$\bot$}
\end{prooftree}
Ora possiamo dimostrare che tutti i teoremi in Tamm sono tautologie derivandoli come segue:
\begin{prooftree}
\RightLabel{\(\)}
\noLine
\AxiomC{}
\UnaryInfC{$\vdash$$\bot$}
\RightLabel{\(\)}
\noLine
\AxiomC{}
\UnaryInfC{$\bot$-ax}
\RightLabel{\(\)}
\noLine
\UnaryInfC{$\bot$$\vdash$Ti}
\RightLabel{comp}
\BinaryInfC{$\vdash$Ti}
\end{prooftree}
per i=1,2,3,4,5
\section{Esercizio 5}
(++): Dall'affermazione
\begin{center}{\bf Ip   Di sera non tutti non escono}\end{center}
Si dica quali delle seguenti affermazioni si possono dedurre (la classificazione di ciascuna vale 8
punti se è deducibile e 14 punti se NON lo è):
\begin{center}A {\bf Qualcuno non esce oppure è sera}\end{center}
\begin{center}B {\bf Qualcuno esce oppure non è sera}\end{center}
\begin{center}C {\bf Se nessuno esce non è sera}\end{center}
Si giustifichi la risposta producendo una sua derivazione nella teoria predicativa
\begin{center}{\bf  TIp = LC= + Ip}\end{center}
dopo aver formalizzato ciascuna affermazione utilizzando:\newline
{\bf E(x)= “x esce ”}\newline
{\bf S=“È sera”}\newline
In particolare si giustifichi le risposte “affermazione X” classificando in {\bf LC=} il sequente {\bf Ip}  {$\vdash$} “affermazione X” .\newline
\subsection{Traduzione ipotesi}
Di sera non tutti non escono\newline
S$\to$$\exists$xE(x)
\subsubsection{Affermazione A}
Qualcuno non esce oppure è sera\newline
$\exists$x$\neg$E(x)$\vee$S\newline
Provo a vedere se l'affermazione A è deducibile dall'ipotesi
\begin{prooftree}
\RightLabel{\(\)}
\noLine
\AxiomC{}
\UnaryInfC{$\vdash$Ip}
\RightLabel{\(\)}
\noLine
\AxiomC{}
\UnaryInfC{loop}
\RightLabel{\(\)}
\noLine
\UnaryInfC{$\vdash$$\neg$E(w),$\exists$x$\neg$E(x),S,S}
\RightLabel{$\exists$-D}
\UnaryInfC{$\vdash$$\exists$x$\neg$E(x),S,S}
\RightLabel{Scdx}
\UnaryInfC{$\vdash$S,$\exists$x$\neg$E(x),S}
\RightLabel{\(\)}
\noLine
\AxiomC{}
\UnaryInfC{loop}
\RightLabel{\(\)}
\noLine
\UnaryInfC{E(w)$\vdash$$\neg$E(w),$\exists$x$\neg$E(x),S}
\RightLabel{$\exists$-D}
\UnaryInfC{E(w)$\vdash$$\exists$x$\neg$E(x),S}
\RightLabel{$\exists$-S w$\notin$VL}
\UnaryInfC{$\exists$xE(x)$\vdash$$\exists$x$\neg$E(x),S}
\RightLabel{$\to$-S}
\BinaryInfC{S$\to$$\exists$xE(x)$\vdash$$\exists$x$\neg$E(x),S}
\RightLabel{$\vee$-D}
\UnaryInfC{S$\to$$\exists$xE(x)$\vdash$$\exists$x$\neg$E(x)$\vee$S}
\RightLabel{comp}
\BinaryInfC{$\vdash$$\exists$x$\neg$E(x)$\vee$S}
\end{prooftree}
Contromodello di Ip$\vdash$AffA\newline
La foglia che non si riesce a derivare\newline
$\neg$E(w),$\exists$x$\neg$E(x),S,S\newline
Suggerisce il seguente contromodello\newline
$D_{contra}$=\{Minni\}\newline
(S)$^{Dcontra}$=0\newline
($\neg$E(w))$^{Dcontra}$(Minni)=0\newline
In questo modello\newline
($\exists$xE(x))$^{Dcontra}$=1 perché nel dominio c'è solo Minni che esce (perché il suo non uscire in questo contromodello ha il valore di 0, quindi esce)\newline
($\exists$x$\neg$E(x))$^{Dcontra}$=0 perché nel dominio c'è solo Minni ed ($\neg$E(x))$^{Dcontra}$=0\newline
Quindi ipotesi ed affermazione assumono questi significati:\newline
(Ip)$^{Dcontra}$=(S$\to$$\exists$xE(x))$^{Dcontra}$=0$\to$1=1\newline
(Aff.A)$^{Dcontra}$=($\exists$x$\neg$E(x)$\vee$S)$^{Dcontra}$=0$\vee$0=0\newline
Quindi (Ip$\vdash$Aff.A)$^{Dcontra}$=1$\vdash$0=0\newline
Ho trovato un contromodello, quindi l'affermazione A non è deducibile dall'ipotesi
\subsubsection{Affermazione B}
Qualcuno esce oppure non è sera\newline
$\exists$xE(x)$\vee$$\neg$S
\begin{prooftree}
\RightLabel{\(\)}
\noLine
\AxiomC{}
\UnaryInfC{$\vdash$Ip}
\RightLabel{\(\)}
\noLine
\AxiomC{}
\UnaryInfC{$\neg$-axdx1}
\RightLabel{\(\)}
\noLine
\UnaryInfC{$\vdash$S,$\exists$xE(x),$\neg$S}
\RightLabel{\(\)}
\noLine
\AxiomC{}
\UnaryInfC{ax-id}
\RightLabel{\(\)}
\noLine
\UnaryInfC{E(w)$\vdash$E(w),$\neg$S}
\RightLabel{$\exists$-Dv}
\UnaryInfC{E(w)$\vdash$$\exists$xE(x),$\neg$S}
\RightLabel{$\exists$-S w$\notin$VL}
\UnaryInfC{$\exists$xE(x)$\vdash$$\exists$xE(x),$\neg$S}
\RightLabel{$\to$-S}
\BinaryInfC{S$\to$$\exists$xE(x)$\vdash$$\exists$xE(x),$\neg$S}
\RightLabel{$\vee$-D}
\UnaryInfC{S$\to$$\exists$xE(x)$\vdash$$\exists$xE(x)$\vee$$\neg$S}
\RightLabel{comp}
\BinaryInfC{$\vdash$$\exists$xE(x)$\vee$$\neg$S}
\end{prooftree}
L'affermazione B è deducibile dall'ipotesi
\subsubsection{Affermazione C}
Se nessuno esce non è sera\newline
$\neg$$\exists$xE(x)$\to$$\neg$S
\begin{prooftree}
\RightLabel{\(\)}
\noLine
\AxiomC{}
\UnaryInfC{$\vdash$Ip}
\RightLabel{\(\)}
\noLine
\AxiomC{}
\UnaryInfC{$\neg$-axdx1}
\RightLabel{\(\)}
\noLine
\UnaryInfC{$\vdash$S,$\exists$xE(x),$\neg$S}
\RightLabel{\(\)}
\noLine
\AxiomC{}
\UnaryInfC{ax-id}
\RightLabel{\(\)}
\noLine
\UnaryInfC{E(w)$\vdash$E(w),$\neg$S}
\RightLabel{$\exists$-Dv}
\UnaryInfC{E(w)$\vdash$$\exists$xE(x),$\neg$S}
\RightLabel{$\exists$-S w$\notin$VL}
\UnaryInfC{$\exists$xE(x)$\vdash$$\exists$xE(x),$\neg$S}
\RightLabel{$\to$-S}
\BinaryInfC{S$\to$$\exists$xE(x)$\vdash$$\exists$xE(x),$\neg$S}
\RightLabel{$\neg$-S}
\UnaryInfC{S$\to$$\exists$xE(x),$\neg$$\exists$xE(x)$\vdash$$\neg$S}
\RightLabel{$\to$-D}
\UnaryInfC{S$\to$$\exists$xE(x)$\vdash$$\neg$$\exists$xE(x)$\to$$\neg$S}
\RightLabel{comp}
\BinaryInfC{$\vdash$$\neg$$\exists$xE(x)$\to$$\neg$S}
\end{prooftree}
L'affermazione C è deducibile dall'ipotesi
\section{Esercizio 6}
Stabilire se la seguente regola è sicura rispetto alla semantica classica (nel caso di regola non sicura
si analizzi entrambe le inverse):\newline
-(++ solo sicurezza della regola) (15 punti)\newline
\subsection{Validità regola}
\begin{prooftree}
\RightLabel{\(\)}
\noLine
\AxiomC{}
\UnaryInfC{M$\vdash$ $\neg$$\neg$F}
\RightLabel{\(\)}
\noLine
\AxiomC{}
\UnaryInfC{$\neg$$\neg$F$\vdash$ $\neg$M}
\RightLabel{1}
\BinaryInfC{M$\vdash$ $\neg$M}
\end{prooftree}
Sia r riga fissata sulla tabella di verità\newline
Ip.1: M$\to$$\neg$$\neg$F=1 su r\newline
IP.2: $\neg$$\neg$F$\to$$\neg$M=1 su r\newline
Ip.3: M=1 su r\newline
Tesi:\newline
$\neg$M=1 su r\newline
Dall'ip. 3 abiamo M=1.\newline
Applicato all'ip.1 abbiamo  M$\to$$\neg$$\neg$F=1$\to$$\neg$$\neg$F che per essere vero deve avere come valore F=1 in modo che, per definizione di implica, 1$\to$1=1\newline
Ma M=1 ed F=1 non rendono vera l'ip.2 perché $\neg$$\neg$F$\to$$\neg$M=1$\to$0=0\newline
Essendo una foglia della regola a valore 0 date le ipotesi, allora sicuramente la regola è valida perché\newline
1$\&$0$\vdash$1$\to$$\neg$1=0$\vdash$0=1\newline
\subsection{Prima regola inversa}
Essendo per l'appunto la regola è valida, analizzo intanto la prima inversa
\begin{prooftree}
\RightLabel{\(\)}
\noLine
\AxiomC{}
\UnaryInfC{M$\vdash$$\neg$M}
\RightLabel{inv-1-1}
\UnaryInfC{M$\vdash$$\neg$$\neg$F}
\end{prooftree}
Sia r riga fissata sulla tabella di verità\newline
Ip.1: M$\to$$\neg$M=1 su r\newline
Ip.2: M=1 su r\newline
Tesi\newline
$\neg$$\neg$F=1 su r\newline
L'ip.2 M=1 applicata all'ip.1 porta ad avere 1$\to$$\neg$1=1$\to$0=0\newline
Dato che una delle ipotesi avrà valore 0, per definizione di implica quando lo 0 è antecedente dell'implicazione essa avrà sempre valore 1\newline
Quindi la prima regola inversa è valida\newline
\subsection{Seconda regola inversa}
Analizzo la seconda inversa
\begin{prooftree}
\RightLabel{\(\)}
\noLine
\AxiomC{}
\UnaryInfC{M$\vdash$$\neg$M}
\RightLabel{inv-1-2}
\UnaryInfC{$\neg$$\neg$F$\vdash$$\neg$M}
\end{prooftree}
Sia r riga fissata sulla tabella di verità\newline
Ip.1: M$\to$$\neg$M=1 su r\newline
Ip.2: $\neg$$\neg$F=1 su r\newline
Tesi\newline
$\neg$M=1 su r\newline
L'ip.2 porta ad avere F=1.\newline
L'ip.1 è verificata con M=0 poiché 0$\to$$\neg$0=0$\to$1=1.\newline
La tesi è verificata poiché $\neg$0=1.\newline
Seconda regola inversa valida\newline
\subsection{Conclusione}
Dato che entrambe le regole inverse sono valide, la regola 1 è sicura
\section{Esercizio 7}
(++) (32 punti)\newline
In un gioco due amiche fanno un’affermazione, che è vera o falsa.\newline
Un’affermazione è mancante e l’altra è riportata sotto:\newline
{\bf Celeste: ....}\newline
{\bf Morgana: Almeno una di noi dice la verità.}\newline
Si può dedurre, anche se non si conosce l’affermazione di Celeste, quante affermazioni sono vere?\newline
a) No\newline
b) Sì, sono vere tutte e due le affermazioni.\newline
c) Sì, è vera solo l’affermazione di Morgana.\newline
d) Sì, è vera solo l’affermazione di Celeste.\newline
e) Nessuna affermazione è vera.\newline
Si analizzino le varie affermazioni nella teoria proposizionale TMorgana ottenuta estendendo LCp con la formalizzazione di ciò che dice Morgana (formalizzazione 2 punti) tramite:\newline
{\bf M= l’affermazione di Morgana è vera}\newline
{\bf C= l’affermazione di Celeste è vera}
\subsection{Formalizzazione Morgana}
Almeno una di noi dice la verità\newline
ax.Morgana=M$\leftrightarrow$M$\vee$C che diventa\newline
(M$\to$M$\vee$C)$\&$(M$\vee$C$\to$M)
\subsubsection{Verifica risposta b}
Provo a vedere se entrambe dicono il vero
\begin{prooftree}
\RightLabel{\(\)}
\noLine
\AxiomC{}
\UnaryInfC{$\vdash$axMorgana}
\RightLabel{\(\)}
\noLine
\AxiomC{}
\UnaryInfC{(M$\to$M$\vee$C)$\&$(M$\vee$C$\to$M)$\vdash$M}
\RightLabel{\(\)}
\noLine
\AxiomC{}
\UnaryInfC{(M$\to$M$\vee$C)$\&$(M$\vee$C$\to$M)$\vdash$C}
\RightLabel{$\&$-D}
\BinaryInfC{(M$\to$M$\vee$C)$\&$(M$\vee$C$\to$M)$\vdash$M$\&$C}
\RightLabel{comp}
\BinaryInfC{$\vdash$M$\&$C}
\end{prooftree}
Grazie alla regola del $\&$-D, posso verificare anche separatamente se Morgana dice il vero o Celeste dice il vero (od eventualmente entrambe)
\subsubsection{Morgana dice il vero?}
\begin{prooftree}
\RightLabel{\(\)}
\noLine
\AxiomC{}
\UnaryInfC{$\vdash$M,M,C,M}
\RightLabel{\(\)}
\noLine
\AxiomC{}
\UnaryInfC{ax-id}
\RightLabel{\(\)}
\noLine
\UnaryInfC{M$\vdash$M,C,M}
\RightLabel{\(\)}
\noLine
\AxiomC{}
\UnaryInfC{ax-id}
\RightLabel{\(\)}
\noLine
\UnaryInfC{C$\vdash$M,C,M}
\RightLabel{\(\)}
\BinaryInfC{M$\vee$C$\vdash$M,C,M}
\RightLabel{$\to$-S}
\BinaryInfC{M$\to$M$\vee$C$\vdash$M,C,M}
\RightLabel{$\vee$-D}
\UnaryInfC{M$\to$M$\vee$C$\vdash$M$\vee$C,M}
\RightLabel{\(\)}
\noLine
\AxiomC{}
\UnaryInfC{ax-id}
\RightLabel{\(\)}
\noLine
\UnaryInfC{M$\to$M$\vee$C,M$\vdash$M}
\RightLabel{$\to$-S}
\BinaryInfC{M$\to$M$\vee$C,M$\vee$C$\to$M$\vdash$M}
\RightLabel{$\&$-S}
\UnaryInfC{(M$\to$M$\vee$C)$\&$(M$\vee$C$\to$M)$\vdash$M}
\end{prooftree}
Il sequente è falso per M=0, C=0, quindi Morgana non dice il vero, escludendo così anche la risposta c, ma controlliamo se Celeste dice il vero
\subsubsection{Celeste dice il vero? (risposta d)}
Celeste dice il vero se la seguente foglia è derivabile \newline
(M$\to$M$\vee$C)$\&$(M$\vee$C$\to$M)$\vdash$C
\begin{prooftree}
\RightLabel{\(\)}
\noLine
\AxiomC{}
\UnaryInfC{$\vdash$M,C,C}
\RightLabel{\(\)}
\noLine
\AxiomC{}
\UnaryInfC{ax-id}
\RightLabel{\(\)}
\noLine
\UnaryInfC{M$\vdash$M,C,C}
\RightLabel{\(\)}
\noLine
\AxiomC{}
\UnaryInfC{ax-id}
\RightLabel{\(\)}
\noLine
\UnaryInfC{C$\vdash$M,C,C}
\RightLabel{$\vee$-S}
\BinaryInfC{M$\vee$C$\vdash$M,C,C}
\RightLabel{$\to$-S}
\BinaryInfC{M$\to$M$\vee$C$\vdash$M$\vee$C,C}
\RightLabel{\(\)}
\noLine
\AxiomC{}
\UnaryInfC{M,M$\vdash$C}
\RightLabel{\(\)}
\noLine
\AxiomC{}
\UnaryInfC{ax-id}
\RightLabel{\(\)}
\noLine
\UnaryInfC{M,C$\vdash$C}
\RightLabel{$\vee$-S}
\BinaryInfC{M,M$\vee$C$\vdash$C}
\RightLabel{\(\)}
\noLine
\AxiomC{}
\UnaryInfC{ax-id}
\RightLabel{\(\)}
\noLine
\UnaryInfC{M$\vdash$M,C}
\RightLabel{$\to$-S}
\BinaryInfC{M$\to$M$\vee$C,M$\vdash$C}
\RightLabel{$\to$-S}
\BinaryInfC{M$\to$M$\vee$C,M$\vee$C$\to$M$\vdash$C}
\RightLabel{$\&$-S}
\UnaryInfC{(M$\to$M$\vee$C)$\&$(M$\vee$C$\to$M)$\vdash$C}
\end{prooftree}
Il sequente per verificare se Celeste dice il vero è falso per M=1,C=0\newline
La risposta d non è possibile
\subsection{Verifica risposta e}
In quanto Morgana e Celeste non dicono il vero, vediamo se è derivabile l'ultima risposta per comprendere se è la risposta corretta o non  è deducibile chi ha ragione
\begin{prooftree}
\RightLabel{\(\)}
\noLine
\AxiomC{}
\UnaryInfC{$\vdash$ax.Morgana}
\RightLabel{\(\)}
\noLine
\AxiomC{}
\UnaryInfC{M$\to$M$\vee$C,M$\vee$C$\to$M$\vdash$$\neg$M}
\RightLabel{\(\)}
\noLine
\AxiomC{}
\UnaryInfC{M$\to$M$\vee$C,M$\vee$C$\to$M$\vdash$$\neg$C}
\RightLabel{$\&$-D}
\BinaryInfC{M$\to$M$\vee$C,M$\vee$C$\to$M$\vdash$$\neg$M$\&$$\neg$C}
\RightLabel{$\&$-S}
\UnaryInfC{(M$\to$M$\vee$C)$\&$(M$\vee$C$\to$M)$\vdash$$\neg$M$\&$$\neg$C}
\RightLabel{comp}
\BinaryInfC{$\vdash$$\neg$M$\&$$\neg$C}
\end{prooftree}
Posso applicare lo stesso ragionamento di prima, ovvero verificare prima se Morgana non dice il vero e se sì verificare anche se Celeste non dice il vero
\subsubsection{Morgana non dice il vero?}
\begin{prooftree}
\RightLabel{\(\)}
\noLine
\AxiomC{}
\UnaryInfC{$\neg$-axdx1}
\RightLabel{\(\)}
\noLine
\UnaryInfC{$\vdash$M,M,C,$\neg$M}
\RightLabel{\(\)}
\noLine
\AxiomC{}
\UnaryInfC{ax-id}
\RightLabel{\(\)}
\noLine
\UnaryInfC{M$\vdash$M,C,$\neg$M}
\RightLabel{\(\)}
\noLine
\AxiomC{}
\UnaryInfC{ax-id}
\RightLabel{\(\)}
\noLine
\UnaryInfC{C$\vdash$M,C,$\neg$M}
\RightLabel{$\vee$-S}
\BinaryInfC{M$\vee$C$\vdash$M,C,$\neg$M}
\RightLabel{$\to$-S}
\BinaryInfC{M$\to$M$\vee$C$\vdash$M,C,$\neg$M}
\RightLabel{$\vee$-D}
\UnaryInfC{M$\to$M$\vee$C$\vdash$M$\vee$C,$\neg$M}
\RightLabel{\(\)}
\noLine
\AxiomC{}
\UnaryInfC{M,M$\vdash$$\neg$M}
\RightLabel{\(\)}
\noLine
\AxiomC{}
\UnaryInfC{M,C$\vdash$$\neg$M}
\RightLabel{$\vee$-S}
\BinaryInfC{M,M$\vee$C$\vdash$$\neg$M}
\RightLabel{\(\)}
\noLine
\AxiomC{}
\UnaryInfC{ax-id}
\RightLabel{\(\)}
\noLine
\UnaryInfC{M$\vdash$M,$\neg$M}
\RightLabel{$\to$-S}
\BinaryInfC{M,M$\to$M$\vee$C$\vdash$$\neg$M}
\RightLabel{Scsx}
\noLine
\UnaryInfC{M$\to$M$\vee$C,M$\vdash$$\neg$M}
\RightLabel{$\to$-S}
\BinaryInfC{M$\to$M$\vee$C,M$\vee$C$\to$M$\vdash$$\neg$M}
\end{prooftree}
Il sequente è falso nella riga falsaria M=1,C=1\newline
Di conseguenza non si dimostra il sequente radice ottenuto dopo la regola di composizione, il che permette di escludere la risposta e come possibile risposta 
\subsection{Conclusione}
La risposta corretta è: no\newline
Non è possibile dedurre chi ha ragione
\end{document}