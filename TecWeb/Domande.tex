\documentclass[10pt,oneside,a4paper]{article}
\usepackage{hyperref, forest, listings}
\hypersetup{
	colorlinks=true,   % Abilita il colore dei link (senza box intorno)
	linkcolor=blue,    % Colore per i link interni (come quelli nell'indice)
	urlcolor=red,      % Colore per i link URL esterni
	filecolor=magenta, % Colore per i link ai file locali
	citecolor=green,   % Colore per i riferimenti bibliografici
	pdfborder={0 0 0}  % Disabilita i bordi intorno ai link
}

\definecolor{lightgray}{rgb}{0.83, 0.83, 0.83}
\lstdefinestyle{pseudocodice}{
	language=HTML,  % Non importa quale linguaggio di base, lo utilizziamo per la sintassi
	basicstyle=\ttfamily\footnotesize,  % Font di base
	keywordstyle=\color{blue}\bfseries,  % Parole chiave in blu e in grassetto, comprende parole chiave c++ e quelle aggiunte
	commentstyle=\color{red},  % Commenti in rosso, seguiti da //
	stringstyle=\color{purple},  % Stringhe in viola, racchiuse fra virgolette, es: "ciao"
	identifierstyle=\color{black},  % Variabili in nero (in generale tutto ciò che è scritto in maniera normale)
	backgroundcolor=\color{lightgray},  % Colore di sfondo
	numbers=left,  % Disabilita la numerazione delle righe
	numberstyle=\tiny\color{black},%imposta stile dei numeri
	frame=single,  % Bordo intorno al codice
	breaklines=true,  % Rientro automatico delle linee troppo lunghe
	morekeywords={then, begin, end, endif, endwhile, elif, endfor, and, or},  % Parole chiave personalizzate per pseudocodice
	tabsize=3
}

\title{Tecnologie Web}
\author{Matteo Mazzaretto}
\date{2025/2026}
\begin{document}
	\pagenumbering{gobble}
	\maketitle
	\begin{center}
		%do un nuovo nome alla tabella degli indici e la inizializzo
		\renewcommand{\contentsname}{Indice}
		\tableofcontents
	\end{center}
	\newpage
	%inizio l'effettivo conteggio delle pagine
	\pagenumbering{arabic}
	\setcounter{page}{1}
	
	\section{ID e Class}

id e class sono attributi HTML utilizzati come selettori nel CSS.\\
La class definisce un gruppo di appartenenza, può essere assegnata a più elementi HTML e lo stesso elemento può avere più classi.\\
Nel CSS si seleziona con .nomeClasse\{ \}, dove tra le parentesi sono definiti gli attributi CSS con i relativi valori\\
L’id identifica un elemento in modo univoco all’interno del documento HTML.\\
I principali utilizzi sono:
\begin{itemize}
	\item selettore in un foglio di stile
	\item riferimento all’interno di script
	\item ancora di destinazione di un link
	\item identificatore generico per il trattamento dei dati
\end{itemize}
Un id deve essere unico nel documento e deve iniziare con una lettera o con il carattere di sottolineatura (\_).\\
Nel CSS si seleziona con \#nomeId\{ \}, dove tra le parentesi sono definiti gli attributi CSS con i relativi valori.
	
	\section{Procedimento a cascata}

Il “procedimento a cascata” definisce le priorità con cui le regole CSS vengono applicate:
\begin{enumerate}
	\item Impostazioni personali dell’utente
	\item Impostazioni di stile inline definite dall’autore della pagina
	\item Fogli di stile embedded definiti dall’autore
	\item Fogli di stile esterni definiti dall’autore
	\item Impostazioni di stile predefinite del browser
	\item Impostazioni utilizzate quando qualcosa non è definito o se il browser non supporta i CSS
\end{enumerate}
Le impostazioni di default cambiano da browser a browser

	
	\section{Significato di PURO}

PURO vuol dire "Percepibile, Comprensibile, Robusto, Utilizzabile".\\
Se qualcosa è PURO allora è anche accessibile, serve ad avere un metro di riferimento per capire se si è raggiunta l'accessibilità
	
	\section{Doppia A o tripla A per l'accessibilità?}
	
	\section{Indicatori di specificità}

Un figlio eredita le impostazioni del padre.\\
Con più regole a stessa importanza viene applicata l'ultima definita.\\
L'importanza si calcola con gli indicatori di specificità.\\
La regola più specifica sarà quella di cui viene applicata la regola.\\
Date le regole, si calcolano così:\\
(num !important, num id, num attributi, num tag html).\\
È molto importante la virgola fra un indicatore e l'altro per permettere la divisione degli elementi.\\
Esempio:

\begin{lstlisting}[style=pseudocodice]
	#nav a{
		color:orange;	
	}
	a{
		color:blue;
	}
\end{lstlisting}
La specificità per il primo è (0, 1, 0, 1).\\
La specificità per il secondo è (0, 0, 0, 1).\\
Quindi viene applicata la prima regola in caso di contrasti.
	
	\section{Rapporto fra meta tag description e la SERP}
	
	\section{Globalizzazione vs localizzazione}

Nella progettazione di un sito web è fondamentale identificare l’audience principale.\\
Occorre stabilire se il sito è destinato a una singola nazione o a più culture.\\
Le opzioni principali sono:
\begin{itemize}
	\item Sito nazionale: accessibile solo da una specifica nazione.
	\item Sito globale uniforme: stesso contenuto in tutti i paesi (es. GitHub).
	\item Siti localizzati: versioni diverse del sito per ciascun paese o cultura (es. siti di automobili).
\end{itemize}
	
	\section{Descrivi bisogni utente con metodo della pesca}
	
	\section{Struttura ottimale gerarchia}

La struttura gerarchica è la base per una buona struttura informativa, rende gli utenti in grado di sviluppare facilmente un modello mentale della struttura del sito e la loro localizzazione in questa struttura.\\
Deve rispettare i requisiti indicati nella domanda \ref{domanda11}
	
	\section{Area sicura per la visualizzazione}

L’area sicura per la visualizzazione è la porzione di pagina web visibile immediatamente senza bisogno di scorrere.\\
Deve contenere le informazioni principali e di primo impatto, tra cui:

\begin{itemize}
	\item Elementi informativi fondamentali
	\item Elementi essenziali per l’interazione
	\item Elementi grafici per l’identità del sito
\end{itemize}
	
	\section{Ampiezza e profondità per un sito web}\label{domanda11}

Nella progettazione di una struttura gerarchica è necessario trovare un buon
equilibrio tra ampiezza (numero di opzioni ad ogni livello) e profondità della
gerarchia (numero di livelli):\\
\begin{itemize}
	\item gerarchie troppo ampie portano al sovraccarico cognitivo: per il web è
	consigliabile non andare oltre le 10 opzioni nel menù principale
	\item gerarchie troppo profonde rendono eccessivo il numero di click necessari per reperire l’informazione: è buona regola non superare i 4 o 5 livelli
\end{itemize}
I siti in evoluzione possono richiedere una riprogettazione della gerarchia, per questo gerarchie ampie e poco profonde sono facilmente aggiornabili.\\
Se è possibile organizzare un menù secondo un ordine conveniente per l'utente, le strutture ampie sono preferibili a quelle profonde.
	
	\section{Emotional design}

L'emotional design si basa sul fatto che le esperienze legate alle emozioni hanno un forte impatto nella memoria a lungo termine e sono richiamate con maggiore accuratezza.\\
Un design deve tenere conto delle emozioni per trasformare un utente casuale in un accanito lettore/cliente.\\
Uno dei modi è l'utilizzo del \textbf{contrasto}, ovvero l'emozione che rappresenta ciò che l'utente non si aspetta, l'interruzione di un pattern conosciuto.\\
Il contrasto può essere:

\begin{itemize}
	\item visuale (differenze nella forma, aspetto, colori)
	\item cognitivo (ricordi)
\end{itemize}
Altre tecniche sono:
\begin{itemize}
	\item l'umanizzazione del prodotto
	\item l'utilizzo di una personalità
	\item l'utilizzo delle emozioni come:
	\begin{itemize}
		\item sorpresa
		\item piacere
		\item anticipazione
		\item status/esclusività
		\item rewards
	\end{itemize}
\end{itemize} 
	
	\section{Differenze fra metodo GET e POST}

Metodo GET: è il predefinito, serve a recuperare informazioni e passa le stringhe in chiaro.\\
Il browser allega la stringa di query all’url:
\begin{itemize}
	\item http://server/path/file.php?parametro=valore
	\item limite alla lunghezza della stringa (256 caratteri)
	\item vulnerabilità dell’accesso
\end{itemize}
Metodo POST: la stringa di query viene passato come input standard, il che vuol dire maggiore facilità di gestione.\\
Serve a passare stringhe importanti che non possono essere catturate direttamente, come le password.\\
In PHP i parametri vengono memorizzati in variabili superglobali:

\begin{itemize}
	\item Con GET: i dati sono accessibili tramite \texttt{\$\_GET}
	\item Con POST: i dati sono accessibili tramite \texttt{\$\_POST}
	\item Indipendentemente dal metodo, i dati sono sempre presenti in \texttt{\$\_REQUEST}, permettendo allo script di non dover distinguere il metodo di invio
\end{itemize}
\texttt{\$\_REQUEST} è indipendente da \texttt{\$\_GET} e \texttt{\$\_POST}: modificarne i valori non influisce sulle altre variabili e viceversa.\\
Il metodo che serve a passare file è \textbf{POST}.\\
Il metodo POST consente l’invio di dati binari tramite form con l’attributo \texttt{enctype="multipart/form-data"}, necessario per l’upload di file. GET non supporta l’invio di file perché i parametri vengono passati nell’URL.
	
	\section{Come costruire una tabella accessibile}
	
	\section{Schema o struttura organizzativa}

Gli schemi esatti sono i migliori quando l'utente sa quello che sta cercando.\\
Gli schemi ambigui sono i migliori per la navigazione e l'apprendimento associativo, quando l'utente ha una vaga idea di quello che sta cercando.\\
Quando possibile è opportuno utilizzare entrambi i tipi di schema.\\
Le principali strutture organizzative sono:
\begin{itemize}
	\item Sequenza
	\item Gerarchia
	\item A faccette (a matrice)
	\item Ipertesto
\end{itemize}

\subsection{Sequenza}
La sequenza rappresenta il modo più semplice di organizzare l'informazione.\\
È adatto per siti didattici, perché impone un ordine al materiale da consultare.
\subsection{Gerarchia}
Una gerarchia ben progettata è la base di una buona architettura informativa.\\
Le suddivisioni mutuamente esclusive e le relazione padre-figlio sono semplici e familiari.\\
Se usate per un sito web rendono gli utenti in grado di sviluppare facilmente un modello mentale della struttura del sito e della loro localizzazione in questa struttura.
In questa struttura è necessario trovare l'equilibro tra ampiezza e profondità, maggiori dettagli nella domanda \ref{domanda11}.
\subsection{Faccette o Matrici}
Ogni informazione può essere vista da diverse angolature, perché ha molti attributi attraverso cui può essere conosciuta e usata.\\
Ad esempio una maglia può essere catalogata per taglia, tipologia, materiale, colore, ...
\subsection{Ipertesto}
Struttura organizzativa innovativa.\\
Le unità informative possono essere collegate attraverso link gerarchicamente o non gerarchicamente, oppure seguendo entrambe le modalità.\\
È una struttura non lineare, molto flessibile ma è un ostacolo per l'utente per la formazione di un modello mentale del sito.\\
Non si adatta bene per la navigazione primaria.\\
	
	\section{Si può capire con solo test automatici se un sito è accessibile}

Non  è possibile in quanto la maggior parte delle cose non sono controllabili tramite test automatici ma solo semiautomici.\\
Alcuni esempi di cose non controllabili sono il layout corretto e visibile del sito oppure la correttezza del testo alternativo dentro l'attributo alt.\\
Relativamente a quest'ultimo si può però controllare la presenza o la mancanza.
	
	\section{Distinzione fra presentazione, struttura e comportamento}

Struttura = (o contenuto) è il testo. \\
Presentazione = come appare il sito. \\
Si sviluppa con il CSS, decisamente meglio se in un file separato ma si può fare
anche nello stesso file. \\
Comportamento è JavaScript, sarebbe meglio dividerlo, più mantenibile.
	
	\section{Nascondere, anche solo parzialmente, il contenuto dei menù tramite menù a tendina è corretto?}

L'utilizzo del menu hamburger su desktop è generalmente sconsigliato perché compromette sia l'usabilità che l'accessibilità. \\
Nascondendo le opzioni di navigazione primaria dietro un'icona, si riduce drasticamente la loro visibilità e scopribilità, costringendo l'utente a un click aggiuntivo solo per visualizzare le possibili destinazioni.\\
Questo design peggiora l'esperienza di navigazione da tastiera, richiede maggiore precisione di puntamento e contraddice le aspettative consolidate degli utenti desktop, che si attendono una barra di navigazione sempre visibile e immediatamente utilizzabile.
	
	\section{Differenza tra convenzioni interne ed esterne}

Le convenzioni esterne sono gli standard universali del web che gli utenti conoscono (esempio: il logo in alto a sinistra rimanda alla home). Non vanno infrante se non per un beneficio davvero straordinario, per non disorientare l'utente. \\
Le convenzioni interne sono le regole di coerenza che stabiliamo all'interno del nostro sito (ad esempio, tutti i pulsanti di conferma sono verdi). Vanno sempre rispettate per aiutare l'utente a costruire una mappa mentale affidabile dell'interfaccia e navigare con sicurezza.
	
	\section{Mettere in ordine di applicazione i seguenti stili CSS}

\begin{enumerate}
	\item Impostazioni personali utente
	\item Dichiarazioni definite con !important
	\item Impostazioni inline definite da autore pagina
	\item Fogli di stile embedded
	\item Fogli di stile esterni definiti dall'autore
	\item Impostazioni predefinite del browser
\end{enumerate}
	
	\section{Qual è il media più accessibile?}

Il testo, in quanto può essere letto da screen reader, tradotto e ingrandito.
	
	\section{Quali sono le tra domande più importanti alle quali devo saper rispondere, pena il fenomeno del disorientamento?}

\begin{itemize}
	\item Dove sono? In titolo e intestazione
	\item Dove posso andare? In menù di navigazione
	\item Di cosa si tratta? Nel contenuto della pagina
\end{itemize}
	
\end{document}