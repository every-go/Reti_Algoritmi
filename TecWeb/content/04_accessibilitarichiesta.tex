\section{Doppia A o tripla A per l'accessibilità?}
Il livello di priorità è determinato dall’impatto che ciascun punto di controllo ha sull’accessibilità:

\begin{itemize}
	\item Priorità 1 (A):
	lo sviluppatore deve conformarsi al presente punto di controllo, pena la preclusione di una o più categorie di utenti dall’accesso alle informazioni.
	Costituisce un requisito base.
	\item Priorità 2 (AA):
	lo sviluppatore dovrebbe conformarsi al presente punto di controllo, altrimenti ad una o più categorie diutenti risulterà difficile accedere alle informazioni.
	Consente di rimuovere barriere significative.
	\item Priorità 3 (AAA):
	lo sviluppatore può tenere in considerazione questo punto di controllo, altrimenti ad una o più categorie di utenti sarà in qualche modo ostacolata nell’accedere alle informazioni.
	Migliora l’accesso ai documenti web.
\end{itemize}
Per l’accessibilità è necessario rispettare tutti i requisiti dei livelli 1 (A) e 2 (AA).\\
Il livello 3 (AAA) è opzionale e dedicato a categorie specifiche di utenti; applicarlo in modo rigoroso potrebbe limitare l’usabilità per altri utenti.