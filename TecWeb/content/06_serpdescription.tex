\section{Rapporto fra meta tag description e la SERP, definizione SEO}

Ogni pagina web contiene diversi meta tag, tra cui \texttt{title}, \texttt{keywords} e \texttt{description}:

\begin{itemize}
	\item \textbf{Title:} deve essere dal particolare al generale, con un massimo di circa 60 caratteri. Rappresenta il titolo visualizzato nei risultati dei motori di ricerca.
	\item \textbf{Keywords:} deve contenere parole chiave presenti nella pagina; la maggior parte dovrebbe comparire anche nelle intestazioni per migliorare il ranking.
	\item \textbf{Description:} non influisce direttamente sul ranking, ma fornisce la descrizione visualizzata nei risultati di ricerca. Deve riassumere accuratamente il contenuto della pagina, evitando clickbait, e aiutare l’utente a decidere se aprire il sito.
\end{itemize}

La SERP (Search Engine Response Page) è la pagina di risposta che si visualizza dopo una ricerca in un motore di ricerca. \\

La SEO (Search Engine Optimization) indica tutte le attività di ottimizzazione di un sito web volte a migliorarne il posizionamento nelle pagine dei risultati d0ei motori di ricerca.