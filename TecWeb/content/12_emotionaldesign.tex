\section{Emotional design}

L'emotional design si basa sul fatto che le esperienze legate alle emozioni hanno un forte impatto nella memoria a lungo termine e sono richiamate con maggiore accuratezza.\\
Un design deve tenere conto delle emozioni per trasformare un utente casuale in un accanito lettore/cliente.\\
Uno dei modi è l'utilizzo del \textbf{contrasto}, ovvero l'emozione che rappresenta ciò che l'utente non si aspetta, l'interruzione di un pattern conosciuto.\\
Il contrasto può essere:

\begin{itemize}
	\item visuale (differenze nella forma, aspetto, colori)
	\item cognitivo (ricordi)
\end{itemize}
Altre tecniche sono:
\begin{itemize}
	\item l'umanizzazione del prodotto
	\item l'utilizzo di una personalità
	\item l'utilizzo delle emozioni come:
	\begin{itemize}
		\item sorpresa
		\item piacere
		\item anticipazione
		\item status/esclusività
		\item rewards
	\end{itemize}
\end{itemize} 