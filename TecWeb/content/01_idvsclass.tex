\section{ID e Class}

id e class sono attributi HTML utilizzati come selettori nel CSS.\\
La class definisce un gruppo di appartenenza, può essere assegnata a più elementi HTML e lo stesso elemento può avere più classi.\\
Nel CSS si seleziona con .nomeClasse\{ \}, dove tra le parentesi sono definiti gli attributi CSS con i relativi valori\\
L’id identifica un elemento in modo univoco all’interno del documento HTML.\\
I principali utilizzi sono:
\begin{itemize}
	\item selettore in un foglio di stile
	\item riferimento all’interno di script
	\item ancora di destinazione di un link
	\item identificatore generico per il trattamento dei dati
\end{itemize}
Un id deve essere unico nel documento e deve iniziare con una lettera o con il carattere di sottolineatura (\_).\\
Nel CSS si seleziona con \#nomeId\{ \}, dove tra le parentesi sono definiti gli attributi CSS con i relativi valori.