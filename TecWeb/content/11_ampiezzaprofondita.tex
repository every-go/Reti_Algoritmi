\section{Ampiezza e profondità per un sito web}\label{domanda11}

Nella progettazione di una struttura gerarchica è necessario trovare un buon
equilibrio tra ampiezza (numero di opzioni ad ogni livello) e profondità della
gerarchia (numero di livelli):\\
\begin{itemize}
	\item gerarchie troppo ampie portano al sovraccarico cognitivo: per il web è
	consigliabile non andare oltre le 10 opzioni nel menù principale
	\item gerarchie troppo profonde rendono eccessivo il numero di click necessari per reperire l’informazione: è buona regola non superare i 4 o 5 livelli
\end{itemize}
I siti in evoluzione possono richiedere una riprogettazione della gerarchia, per questo gerarchie ampie e poco profonde sono facilmente aggiornabili.\\
Se è possibile organizzare un menù secondo un ordine conveniente per l'utente, le strutture ampie sono preferibili a quelle profonde.