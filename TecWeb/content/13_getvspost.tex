\section{Differenze fra metodo GET e POST}

Metodo GET: è il predefinito, serve a recuperare informazioni e passa le stringhe in chiaro.\\
Il browser allega la stringa di query all’url:
\begin{itemize}
	\item http://server/path/file.php?parametro=valore
	\item limite alla lunghezza della stringa (256 caratteri)
	\item vulnerabilità dell’accesso
\end{itemize}
Metodo POST: la stringa di query viene passato come input standard, il che vuol dire maggiore facilità di gestione.\\
Serve a passare stringhe importanti che non possono essere catturate direttamente, come le password.\\
In PHP i parametri vengono memorizzati in variabili superglobali:

\begin{itemize}
	\item Con GET: i dati sono accessibili tramite \texttt{\$\_GET}
	\item Con POST: i dati sono accessibili tramite \texttt{\$\_POST}
	\item Indipendentemente dal metodo, i dati sono sempre presenti in \texttt{\$\_REQUEST}, permettendo allo script di non dover distinguere il metodo di invio
\end{itemize}
\texttt{\$\_REQUEST} è indipendente da \texttt{\$\_GET} e \texttt{\$\_POST}: modificarne i valori non influisce sulle altre variabili e viceversa.\\
Il metodo che serve a passare file è \textbf{POST}.\\
Il metodo POST consente l’invio di dati binari tramite form con l’attributo \texttt{enctype="multipart/form-data"}, necessario per l’upload di file. GET non supporta l’invio di file perché i parametri vengono passati nell’URL.