\section{Schema o struttura organizzativa}

Gli schemi esatti sono i migliori quando l'utente sa quello che sta cercando.\\
Gli schemi ambigui sono i migliori per la navigazione e l'apprendimento associativo, quando l'utente ha una vaga idea di quello che sta cercando.\\
Quando possibile è opportuno utilizzare entrambi i tipi di schema.\\
Le principali strutture organizzative sono:
\begin{itemize}
	\item Sequenza
	\item Gerarchia
	\item A faccette (a matrice)
	\item Ipertesto
\end{itemize}

\subsection{Sequenza}
La sequenza rappresenta il modo più semplice di organizzare l'informazione.\\
È adatto per siti didattici, perché impone un ordine al materiale da consultare.
\subsection{Gerarchia}
Una gerarchia ben progettata è la base di una buona architettura informativa.\\
Le suddivisioni mutuamente esclusive e le relazione padre-figlio sono semplici e familiari.\\
Se usate per un sito web rendono gli utenti in grado di sviluppare facilmente un modello mentale della struttura del sito e della loro localizzazione in questa struttura.
In questa struttura è necessario trovare l'equilibro tra ampiezza e profondità, maggiori dettagli nella domanda \ref{domanda11}.
\subsection{Faccette o Matrici}
Ogni informazione può essere vista da diverse angolature, perché ha molti attributi attraverso cui può essere conosciuta e usata.\\
Ad esempio una maglia può essere catalogata per taglia, tipologia, materiale, colore, ...
\subsection{Ipertesto}
Struttura organizzativa innovativa.\\
Le unità informative possono essere collegate attraverso link gerarchicamente o non gerarchicamente, oppure seguendo entrambe le modalità.\\
È una struttura non lineare, molto flessibile ma è un ostacolo per l'utente per la formazione di un modello mentale del sito.\\
Non si adatta bene per la navigazione primaria.\\